\documentclass{ximera}

%\usepackage{todonotes}
%\usepackage{mathtools} %% Required for wide table Curl and Greens
%\usepackage{cuted} %% Required for wide table Curl and Greens
\newcommand{\todo}{}

\usepackage{esint} % for \oiint
\ifxake%%https://math.meta.stackexchange.com/questions/9973/how-do-you-render-a-closed-surface-double-integral
\renewcommand{\oiint}{{\large\bigcirc}\kern-1.56em\iint}
\fi


\graphicspath{
  {./}
  {ximeraTutorial/}
  {basicPhilosophy/}
  {functionsOfSeveralVariables/}
  {normalVectors/}
  {lagrangeMultipliers/}
  {vectorFields/}
  {greensTheorem/}
  {shapeOfThingsToCome/}
  {dotProducts/}
  {partialDerivativesAndTheGradientVector/}
  {../productAndQuotientRules/exercises/}
  {../motionAndPathsInSpace/exercises/}
  {../normalVectors/exercisesParametricPlots/}
  {../continuityOfFunctionsOfSeveralVariables/exercises/}
  {../partialDerivativesAndTheGradientVector/exercises/}
  {../directionalDerivativeAndChainRule/exercises/}
  {../commonCoordinates/exercisesCylindricalCoordinates/}
  {../commonCoordinates/exercisesSphericalCoordinates/}
  {../greensTheorem/exercisesCurlAndLineIntegrals/}
  {../greensTheorem/exercisesDivergenceAndLineIntegrals/}
  {../shapeOfThingsToCome/exercisesDivergenceTheorem/}
  {../greensTheorem/}
  {../shapeOfThingsToCome/}
  {../separableDifferentialEquations/exercises/}
  {vectorFields/}
}

\newcommand{\mooculus}{\textsf{\textbf{MOOC}\textnormal{\textsf{ULUS}}}}

\usepackage{tkz-euclide}\usepackage{tikz}
\usepackage{tikz-cd}
\usetikzlibrary{arrows}
\tikzset{>=stealth,commutative diagrams/.cd,
  arrow style=tikz,diagrams={>=stealth}} %% cool arrow head
\tikzset{shorten <>/.style={ shorten >=#1, shorten <=#1 } } %% allows shorter vectors

\usetikzlibrary{backgrounds} %% for boxes around graphs
\usetikzlibrary{shapes,positioning}  %% Clouds and stars
\usetikzlibrary{matrix} %% for matrix
\usepgfplotslibrary{polar} %% for polar plots
\usepgfplotslibrary{fillbetween} %% to shade area between curves in TikZ
%\usetkzobj{all} obsolete

\usepackage[makeroom]{cancel} %% for strike outs
%\usepackage{mathtools} %% for pretty underbrace % Breaks Ximera
%\usepackage{multicol}
\usepackage{pgffor} %% required for integral for loops



%% http://tex.stackexchange.com/questions/66490/drawing-a-tikz-arc-specifying-the-center
%% Draws beach ball
\tikzset{pics/carc/.style args={#1:#2:#3}{code={\draw[pic actions] (#1:#3) arc(#1:#2:#3);}}}



\usepackage{array}
\setlength{\extrarowheight}{+.1cm}
\newdimen\digitwidth
\settowidth\digitwidth{9}
\def\divrule#1#2{
\noalign{\moveright#1\digitwidth
\vbox{\hrule width#2\digitwidth}}}





\newcommand{\RR}{\mathbb R}
\newcommand{\R}{\mathbb R}
\newcommand{\N}{\mathbb N}
\newcommand{\Z}{\mathbb Z}

\newcommand{\sagemath}{\textsf{SageMath}}


%\renewcommand{\d}{\,d\!}
\renewcommand{\d}{\mathop{}\!d}
\newcommand{\dd}[2][]{\frac{\d #1}{\d #2}}
\newcommand{\pp}[2][]{\frac{\partial #1}{\partial #2}}
\renewcommand{\l}{\ell}
\newcommand{\ddx}{\frac{d}{\d x}}

\newcommand{\zeroOverZero}{\ensuremath{\boldsymbol{\tfrac{0}{0}}}}
\newcommand{\inftyOverInfty}{\ensuremath{\boldsymbol{\tfrac{\infty}{\infty}}}}
\newcommand{\zeroOverInfty}{\ensuremath{\boldsymbol{\tfrac{0}{\infty}}}}
\newcommand{\zeroTimesInfty}{\ensuremath{\small\boldsymbol{0\cdot \infty}}}
\newcommand{\inftyMinusInfty}{\ensuremath{\small\boldsymbol{\infty - \infty}}}
\newcommand{\oneToInfty}{\ensuremath{\boldsymbol{1^\infty}}}
\newcommand{\zeroToZero}{\ensuremath{\boldsymbol{0^0}}}
\newcommand{\inftyToZero}{\ensuremath{\boldsymbol{\infty^0}}}



\newcommand{\numOverZero}{\ensuremath{\boldsymbol{\tfrac{\#}{0}}}}
\newcommand{\dfn}{\textbf}
%\newcommand{\unit}{\,\mathrm}
\newcommand{\unit}{\mathop{}\!\mathrm}
\newcommand{\eval}[1]{\bigg[ #1 \bigg]}
\newcommand{\seq}[1]{\left( #1 \right)}
\renewcommand{\epsilon}{\varepsilon}
\renewcommand{\phi}{\varphi}


\renewcommand{\iff}{\Leftrightarrow}

\DeclareMathOperator{\arccot}{arccot}
\DeclareMathOperator{\arcsec}{arcsec}
\DeclareMathOperator{\arccsc}{arccsc}
\DeclareMathOperator{\si}{Si}
\DeclareMathOperator{\scal}{scal}
\DeclareMathOperator{\sign}{sign}


%% \newcommand{\tightoverset}[2]{% for arrow vec
%%   \mathop{#2}\limits^{\vbox to -.5ex{\kern-0.75ex\hbox{$#1$}\vss}}}
\newcommand{\arrowvec}[1]{{\overset{\rightharpoonup}{#1}}}
%\renewcommand{\vec}[1]{\arrowvec{\mathbf{#1}}}
\renewcommand{\vec}[1]{{\overset{\boldsymbol{\rightharpoonup}}{\mathbf{#1}}}\hspace{0in}}

\newcommand{\point}[1]{\left(#1\right)} %this allows \vector{ to be changed to \vector{ with a quick find and replace
\newcommand{\pt}[1]{\mathbf{#1}} %this allows \vec{ to be changed to \vec{ with a quick find and replace
\newcommand{\Lim}[2]{\lim_{\point{#1} \to \point{#2}}} %Bart, I changed this to point since I want to use it.  It runs through both of the exercise and exerciseE files in limits section, which is why it was in each document to start with.

\DeclareMathOperator{\proj}{\mathbf{proj}}
\newcommand{\veci}{{\boldsymbol{\hat{\imath}}}}
\newcommand{\vecj}{{\boldsymbol{\hat{\jmath}}}}
\newcommand{\veck}{{\boldsymbol{\hat{k}}}}
\newcommand{\vecl}{\vec{\boldsymbol{\l}}}
\newcommand{\uvec}[1]{\mathbf{\hat{#1}}}
\newcommand{\utan}{\mathbf{\hat{t}}}
\newcommand{\unormal}{\mathbf{\hat{n}}}
\newcommand{\ubinormal}{\mathbf{\hat{b}}}

\newcommand{\dotp}{\bullet}
\newcommand{\cross}{\boldsymbol\times}
\newcommand{\grad}{\boldsymbol\nabla}
\newcommand{\divergence}{\grad\dotp}
\newcommand{\curl}{\grad\cross}
%\DeclareMathOperator{\divergence}{divergence}
%\DeclareMathOperator{\curl}[1]{\grad\cross #1}
\newcommand{\lto}{\mathop{\longrightarrow\,}\limits}

\renewcommand{\bar}{\overline}

\colorlet{textColor}{black}
\colorlet{background}{white}
\colorlet{penColor}{blue!50!black} % Color of a curve in a plot
\colorlet{penColor2}{red!50!black}% Color of a curve in a plot
\colorlet{penColor3}{red!50!blue} % Color of a curve in a plot
\colorlet{penColor4}{green!50!black} % Color of a curve in a plot
\colorlet{penColor5}{orange!80!black} % Color of a curve in a plot
\colorlet{penColor6}{yellow!70!black} % Color of a curve in a plot
\colorlet{fill1}{penColor!20} % Color of fill in a plot
\colorlet{fill2}{penColor2!20} % Color of fill in a plot
\colorlet{fillp}{fill1} % Color of positive area
\colorlet{filln}{penColor2!20} % Color of negative area
\colorlet{fill3}{penColor3!20} % Fill
\colorlet{fill4}{penColor4!20} % Fill
\colorlet{fill5}{penColor5!20} % Fill
\colorlet{gridColor}{gray!50} % Color of grid in a plot

\newcommand{\surfaceColor}{violet}
\newcommand{\surfaceColorTwo}{redyellow}
\newcommand{\sliceColor}{greenyellow}




\pgfmathdeclarefunction{gauss}{2}{% gives gaussian
  \pgfmathparse{1/(#2*sqrt(2*pi))*exp(-((x-#1)^2)/(2*#2^2))}%
}


%%%%%%%%%%%%%
%% Vectors
%%%%%%%%%%%%%

%% Simple horiz vectors
\renewcommand{\vector}[1]{\left\langle #1\right\rangle}


%% %% Complex Horiz Vectors with angle brackets
%% \makeatletter
%% \renewcommand{\vector}[2][ , ]{\left\langle%
%%   \def\nextitem{\def\nextitem{#1}}%
%%   \@for \el:=#2\do{\nextitem\el}\right\rangle%
%% }
%% \makeatother

%% %% Vertical Vectors
%% \def\vector#1{\begin{bmatrix}\vecListA#1,,\end{bmatrix}}
%% \def\vecListA#1,{\if,#1,\else #1\cr \expandafter \vecListA \fi}

%%%%%%%%%%%%%
%% End of vectors
%%%%%%%%%%%%%

%\newcommand{\fullwidth}{}
%\newcommand{\normalwidth}{}



%% makes a snazzy t-chart for evaluating functions
%\newenvironment{tchart}{\rowcolors{2}{}{background!90!textColor}\array}{\endarray}

%%This is to help with formatting on future title pages.
\newenvironment{sectionOutcomes}{}{}



%% Flowchart stuff
%\tikzstyle{startstop} = [rectangle, rounded corners, minimum width=3cm, minimum height=1cm,text centered, draw=black]
%\tikzstyle{question} = [rectangle, minimum width=3cm, minimum height=1cm, text centered, draw=black]
%\tikzstyle{decision} = [trapezium, trapezium left angle=70, trapezium right angle=110, minimum width=3cm, minimum height=1cm, text centered, draw=black]
%\tikzstyle{question} = [rectangle, rounded corners, minimum width=3cm, minimum height=1cm,text centered, draw=black]
%\tikzstyle{process} = [rectangle, minimum width=3cm, minimum height=1cm, text centered, draw=black]
%\tikzstyle{decision} = [trapezium, trapezium left angle=70, trapezium right angle=110, minimum width=3cm, minimum height=1cm, text centered, draw=black]


\author{Bart Snapp}

\outcome{Evaluate limits of functions of several variables.}
\outcome{Determine continuity of functions of several variables.}
\outcome{Use different paths to show that a limit does not exist.}

\title[Dig-In:]{Continuity}

\begin{document}
\begin{abstract}
We investigate what continuity means for real-valued functions of several variables.
\end{abstract}
\maketitle


This section investigates what it means for real -valued functions of $n$-variables to be ``continuous'' .

%Assume $D is a set in $\R^n$
%F:D\to \R
%\]
We begin with a series of definitions. We are
used to ``open intervals'' such as $(1,3)$, which represents the set
of all $x$ such that $1<x<3$, and ``closed intervals'' such as
$[1,3]$, which represents the set of all $x$ such that $1\leq x\leq
3$. We need analogous definitions for open and closed sets in $\R^n$.

\begin{definition}
  We give these definitions in general, for when one is working in
  $\R^n$:
  \begin{itemize}
  \item An \dfn{open ball} $B$ in $\R^n$ centered at $\vec{a} =
    \vector{a_1,a_2,\dots,a_n}$ with radius $r$ is the set of all
    vectors $\vec{x}=\vector{x_1,x_2,\dots,x_n}$ such that
    $|\vec{x}-\vec{a}| < r$. In $\R^2$ an open ball is often called an
    \dfn{open disk}.
    \begin{image}
      \begin{tikzpicture}
        \draw[ultra thick,dashed,penColor,fill=fill1] (0,0) circle (2cm);
        \draw[draw=none,fill=black] (0,0) circle (.05cm);
         \draw[draw=none,fill=black] (1.5,0) circle (.05cm);
        \draw (0,0)--(1.5,0);
        \node[above] at (0.7,0) {\tiny{$|\vec{x}-\vec{a}|<r$}};
        \node[below ] at (0,0) {$\vec{a}$};
        \node[below ] at (1.5,0) {$\vec{x}$};
            \node[below ] at (-1,1.6) {$B$};
      \end{tikzpicture}
    \end{image}
    
  \item A point $\vec{p}$ (denoted by a vector) in $S$ is an
    \dfn{interior point} of $S$ if there is an open ball $B$ centered
    at $\vec{p}$ that contains only points in $S$. We can write this
    in symbols as
    \[
    \vec{p}\in B\subseteq S
    \]
    \begin{image}
      \begin{tikzpicture}
        \draw[ thick, penColor,fill=fill1] plot [smooth cycle] coordinates {(-1.5,.5) (.5,1) (1,2.5) (-1,2.5) (-2,1.5)};
        \draw[draw=none,fill=black] (-1,1) circle (.05cm);
        \draw[dashed] (-1,1) circle (.4cm);
        \node[penColor] at (-.5,2) {\small{$S$}};   
          \node[penColor]  at (-1.18,1.03) {\tiny{$\vec{p}$}};
        \node[above ] at (-0.8,1.02) {\tiny{$B$}};   
    \end{tikzpicture}
  \end{image}

  \item Let $S$ be a set of points in $\R^n$. A point $\vec{p}$ (denoted by a vector) in $\R^n$ is
    a \dfn{boundary point} of $S$ if all open balls centered at $\vec{p}$
    contain both points in $S$ and points not in $S$.
    \begin{image}
      \begin{tikzpicture}
        \draw[ thick, penColor,fill=fill1] plot [smooth cycle] coordinates {(-1.5,.5) (.5,1) (1,2.5) (-1,2.5) (-2,1.5)};
        \draw[draw=none,fill=black] (.89,1.6) circle (.05cm);
        \draw[dashed] (.89,1.6) circle (.4cm);
        \node[penColor] at (-.5,2) {\small{$S$}}; 
          \node[penColor]  at (1.1,1.5) {\tiny{$\vec{p}$}};     
    \end{tikzpicture}
  \end{image}
  \item A set $O$ is \dfn{open} if every point in $O$ is an interior
    point.
    \begin{image}
      \begin{tikzpicture}
        \draw[ thick, dashed, penColor,fill=fill1] plot [smooth cycle] coordinates {(-1.5,.5) (.5,1) (1,2.5) (-1,2.5) (-2,1.5)};
        \node[penColor] at (-.5,2) {$O$};     
      \end{tikzpicture}
    \end{image}
  \item A set $C$ is \dfn{closed} if it contains all of its boundary
    points.
    \begin{image}
      \begin{tikzpicture}
        \draw[ thick, penColor,fill=fill1] plot [smooth cycle] coordinates {(-1.5,.5) (.5,1) (1,2.5) (-1,2.5) (-2,1.5)};
        \node[penColor] at (-.5,2) {$C$};     
      \end{tikzpicture}
    \end{image}
  \item A set $S$ is \dfn{bounded} if there is an open ball $B$
    centered at the origin of radius $M$ such that
    \[
    S\subseteq B.
    \]
    A set that is not bounded is \dfn{unbounded}.
  \end{itemize}
  \item Given a set $S$, we denote the \dfn{boundary} of $S$ by
    $\partial S$.
\end{definition}


\begin{example}
  Consider a closed disk $D$ in $\R^2$. Describe $\partial D$.% and $\partial \partial D$.
  \begin{explanation}
    Since $\partial D$ is the boundary of a closed disk in $\R^2$, $\partial D$ is \wordChoice{
      \choice{a disk}
      \choice[correct]{a circle}
      \choice{a ball}
      \choice{a line}}.
    %% Since $\partial\partial D$ is the boundary of the boundary, and a circle has no boundary, $\partial\partial D$ is \wordChoice{
    %%   \choice[correct]{empty}
    %%   \choice{a line}
    %%   \choice{a circle}
    %%   }.
  \end{explanation}
\end{example}



\begin{example}
  Determine if the domain of the function
  $F(x,y)=\sqrt{1-\frac{x^2}9-\frac{y^2}4}$ is open, closed, or
  neither, and if it is bounded.
  \begin{explanation}
    We've already found the domain of this function to be
    \[
    D = \{(x,y): \answer[given]{x^2/9+y^2/4}\leq 1\}.
    \]
    This is the region \textit{bounded} by the ellipse
    $\frac{x^2}9+\frac{y^2}4=1$. Since the region includes the
    boundary (indicated by the use of ``$\leq$''), the set
    \wordChoice{\choice[correct]{contains}\choice{does not contain}}
    all of its boundary points and hence is closed. The region is
    \wordChoice{\choice[correct]{bounded}\choice{unbounded}} as a disk
    of radius $4$, centered at the origin, contains $D$.
    
    \begin{image}
            \begin{tikzpicture}
            	\begin{axis}[
            		domain=-10:10, ymax=4.6,xmax=4.6, ymin=-4.6, xmin=-4.6,
            		axis lines =center, xlabel=$x$, xtick={-4,-3,3,4}, ylabel=$y$, ytick={-4,-2,2,4},
            		every axis y label/.style={at=(current axis.above origin),anchor=south},
            		every axis x label/.style={at=(current axis.right of origin),anchor=west},
            		axis on top,
            		]
                      

		
                 %ellipse
                  \addplot [draw=penColor,domain=-2.8:2.8, ultra thick,smooth] {sqrt(4- 4/9*x^2)};
                  \addplot [draw=penColor,domain=-3:-2.8,ultra  thick,smooth,samples=200] {sqrt(4- 4/9*x^2)};
                  \addplot [draw=penColor,domain=2.8:3,ultra thick,smooth,samples=200] {sqrt(4- 4/9*x^2)};
                  \addplot [draw=penColor,domain=-2.8:2.8,ultra thick,smooth] {-sqrt(4- 4/9*x^2)};
                  \addplot [draw=penColor,domain=-3:-2.8,ultra thick,smooth,samples=200] {-sqrt(4- 4/9*x^2)};
                  \addplot [draw=penColor,domain=2.8:3, ultra thick,smooth,samples=200] {-sqrt(4- 4/9*x^2)};
 \addplot [draw=none,fill=fillp,domain=-4:4, smooth,samples=200] {sqrt(4- 4/9*x^2)} \closedcycle; 
  \addplot [draw=none,fill=fillp,domain=-4:4, smooth,samples=200] {-sqrt(4- 4/9*x^2)} \closedcycle; 
               	\node at (axis cs:1,1) [penColor] {\large{D}};
            	 \node at (axis cs:-1.2,0.6) [penColor] {\small{$\frac{x^2}{9}+\frac{y^2}{4} \le1$}};
		%circle
	     %	\addplot [draw=penColor3,domain=-3.8:3.8,thick,smooth] {sqrt(16-x^2)};
                 % \addplot [draw=penColor3,domain=-4:-3.8, thick,smooth,samples=200] {sqrt(16- x^2)};
                %  \addplot [draw=penColor3,domain=3.8:4,thick,smooth,samples=200] {sqrt(16- x^2)};
                 % \addplot [draw=penColor3,domain=-3.8:3.8, thick,smooth] {-sqrt(16- x^2)};
                %  \addplot [draw=penColor3,domain=-4:-3.8, thick,smooth,samples=200] {-sqrt(16- x^2)};
                 % \addplot [draw=penColor3,domain=3.8:4,thick,smooth,samples=200] {-sqrt(16- x^2)};
	
		
		 \node at (axis cs:2.6,2.6) [penColor2] {\small{$\frac{x^2}{9}+\frac{y^2}{4} >1$}};
	    
	      \end{axis}
            \end{tikzpicture}
            \end{image}

  \end{explanation}
\end{example}

\begin{example}
  Determine if the domain of $F(x,y) = \frac{1}{x-y}$ is open, closed,
  or neither, and if it is bounded.
  \begin{explanation}
    As we cannot divide by $0$, we find the domain to be
    \[
    D = \{(x,y):x-y\neq \answer[given]{0}\}.
    \]
    In other words, the domain is the set of all points $(x,y)$
    \textit{not} on the line $y=x$. For your viewing pleasure, we have
    included a graph:
    \begin{image}
      \begin{tikzpicture}
        \begin{axis}[
            tick label style={font=\scriptsize},axis y line=middle,axis x line=middle,name=myplot,axis on top,%
            xtick=\empty,
            ytick=\empty,
            ymin=-1,ymax=1,%
            xmin=-1,xmax=1%
          ]
          \filldraw [fill1,fill=fill1] (axis cs:-1,-1) rectangle (axis cs: 1,1);          
          \addplot [ultra thick,white]coordinates {(-1.,-1.)(1,1)};
        \end{axis}
        \node [right] at (myplot.right of origin) {\scriptsize $x$};
        \node [above] at (myplot.above origin) {\scriptsize $y$};
      \end{tikzpicture}
    \end{image}
    Note how we can draw an open disk around any point in the domain
    that lies entirely inside the domain, and also note how the only
    boundary points of the domain are the points on the line $y=x$. We
    conclude the domain is \wordChoice{\choice[correct]{an open
        set}\choice{a closed set}\choice{neither open nor closed
        set}}. Moreover, the set is \wordChoice{\choice{bounded}\choice[correct]{unbounded}}.
  \end{explanation}
\end{example}

\section{Limits}

On to the definition of a limit!
Recall that for functions of a single variable, we say that $\Lim{x}{a} f(x) = L$ if the value of $f(x)$ can be made arbitrarily close to $L$ for \emph{all} $x$ sufficiently close, but not equal to, $x=a$.

This easily allows us to make a similar definition for functions of several variables.

\begin{definition}
 Suppose that $F$ is a real-valued function of $n$ variables. Assume that the domain of $F$ contains a small $n$-dimensional ball centered at a point $\vec{a}$, except, possibly, the point $\vec{a}$.
  The \dfn{limit} of $F$ as $\vec{x}$ approaches $\vec{a}$ is $L$ if the value of $F(\vec{x})$ can be made as close as one wishes to $L$ for all $\vec{x}$ sufficiently close, but not equal to, $\vec{a}$.
 
 When this occurs, we write 
 \[
 \lim_{\vec{x}\to \vec{a}} F(\vec{x}) = L
 \]  
\end{definition}

 \begin{remark}   
  Suppose that $F$ is a real-valued function of two variables, $\vec{x} = \vector{x,y}$, and $\vec{a} =
  \vector{a,b}$. 
  Then the statement $ \lim_{\vec{x}\to \vec{a}} F(\vec{x}) = L$ can be also written as
    \[
    \lim_{\point{x,y}\to \point{a,b} }F\left(x,y\right) = L
    \]
\end{remark}
  \begin{remark}
    Suppose that $F$ is a function of three variables, $\vec{x} = \vector{x,y,z}$, and
    $\vec{a} = \vector{a,b,c}$. Similarly, the statement 
    $ \lim_{\vec{x}\to \vec{a}} F(\vec{x}) = L$ can be written equivalently as

      \[
      \lim_{\point{x,y,z}\to \point{a,b,c}} F\left(x,y,z\right) = L
      \]


\end{remark}

While the intuitive idea behind limits seems to remain unchanged, something interesting is worth observing.  One of the most important ideas for limits of a function of a single variable is the notion of a sided limit.  For functions of a single variable, there were really only two natural ways for $x$ to become close to $a$; we could take $x$ to approach the point $a$ from the left or the right.  For instance,
\[
\Lim{x}{a^-}f(x) 
\]
tells us to consider the inputs $x<a$ only.  In fact, there's a theorem that guarantees that $\Lim{x}{a} f(x) = L$ if and only if $\Lim{x}{a^-}f(x) =L$ and $\Lim{x}{a^+}f(x) =L$, meaning that the function must approach the same value as the input approaches $a$ from both the left and the right.

On the other hand, there are now \emph{infinitely many} ways for, e.g., $(x,y)\to (a,b)$; we can approach along a straight line path parallel to the $x$-axis or $y$-axis, other straight line paths, or even other types of curves.  
\begin{image}
\begin{tikzpicture}

\begin{axis}
	[
	domain=-2:8, ymax=2.9,xmax=6.9, ymin=-2.9, xmin=-.5,
	axis lines=center, xlabel=$x$, ylabel=$y$,
	every axis y label/.style={at=(current axis.above origin),anchor=south},
	every axis x label/.style={at=(current axis.right of origin),anchor=west},
	axis on top,
	typeset ticklabels with strut,
	]


	  \addplot[draw=none,penColor4,fill=fill1] (3.8,1.2) circle (0.6cm);
	\addplot [name path=A,domain=-3:3,draw=none] {10};   
	\addplot [name path=B,domain=3:8,draw=none] {10};
	\addplot [name path=C,domain=-3:3,draw=none] {-10};
	\addplot [name path=D,domain=3:8,draw=none] {-10};
	%\addplot [fill=penColor!40] fill between[of=A and C];
	%\addplot [fill=penColor2!40] fill between[of=B and D];
	
	%\node at (axis cs:5,-1.2) [penColor2] {\footnotesize $F(x,y) = 2x+2y$};
	%\node at (axis cs:4,1.8) [penColor2] {\footnotesize $(x,2), x>3$};

	%\node at (axis cs:1.5,-1.2) [penColor] {\footnotesize $F(x,y) = 3x-y$};
%\node at (axis cs:2,1.8) [penColor] {\footnotesize $(x,2), x<3$};
	
	\addplot[color=penColor,fill=penColor,only marks,mark=*] coordinates{(3.8,1.2)};
	\node at (axis cs:4.3,0.9) [penColor] { $(a,b)$};	
	
	\addplot [draw=penColor2,very thick, smooth,->] coordinates {(2.8,1.2)(3.7,1.2)};
	\addplot [draw=penColor2,very thick, smooth,->] coordinates {(4.8,1.2)(3.9,1.2)};
	\addplot [draw=penColor2,very thick, smooth,->] coordinates {(3.8,2.2)(3.8,1.3)};
	\addplot [draw=penColor2,very thick, smooth,->] coordinates {(3.8,0.2)(3.8,1.1)};
	  \addplot[ very thick, penColor2,smooth,->] plot coordinates { (4.4,2) (4.2,1.8) (4,1.5) (3.85,1.25)};
	   \addplot[very thick, penColor2,smooth,->] plot coordinates { (2.89,2) (3.48,1.8) (3.65,1.5) (3.75,1.25)};
	    \addplot[ very thick, penColor2,smooth,->] plot coordinates { (2.89,0.6) (3.48,0.8) (3.6,1.06) (3.75,1.15)};
	    
\end{axis}
\end{tikzpicture}
\end{image}




%To generalize the theorem for functions of a single variable requires that the function approach the same value along \emph{every} path leading to $\point{a,b}$.

%\begin{theorem}
%Given a function $F:\R^2\to\R$, $\Lim{\pt{x}}{\pt{a}}F(\pt{x}) = L$ if and only if for every curve $C$ in the domain of $F$, $F(\pt{x}) \to L$ as $\pt{x} \to \pt{a}$ along $C$.
%\end{theorem}

In order to check whether a limit exists, do we have to verify that the function tends to the same value along infinitely many different paths?

  While this may seem problematic, there is some good news; many of the limit laws from before still do hold now.

\begin{theorem}[Limit Laws]
  Let $F$ and $G$ be real-valued functions of two
  variables, and $b$, $L$ and $M$ be real numbers, where
  \[
  \lim_{\pt{x}\to\pt{a}}F(\pt{x}) = L \quad \text{and}\quad \lim_{\pt{x}\to\pt{a}} G(\pt{x}) = M.
  \]
Let $\pt{x} = \point{x_{1},x_{2}}$, and $\pt{a} = \point{a_{1},a_{2}}$. We  also assume that  the domain of $F$ and the domain of $G$
 both  contain a small disk with the center at $\pt{a}$, except possibly the point $\pt{a}$.
\begin{description}
\item[Constant Law] $\lim_{\pt{x}\to \pt{a}} b = b$
\item[Identity Law] $\lim_{\pt{x}\to \pt{a}} x_i = a_i$, where $i=1,2.$
\item[Sum/Difference Law] $\lim_{\pt{x}\to \pt{a}}\big(F(\pt{x})\pm G(\pt{x})\big) = L\pm M$
\item[Scalar Multiple Law] $\lim_{\pt{x}\to \pt{a}} b\cdot F(\pt{x}) = bL$
\item[Product Law] $\lim_{\pt{x}\to \pt{a}} \left(F(\pt{x})\cdot G(\pt{x})\right) = LM$
\item[Quotient Law] $\lim_{\pt{x}\to \pt{a}} \frac{F(\pt{x})}{G(\pt{x})} = \frac{L}{M}$, if $M\neq 0$
\end{description}
\end{theorem}

In practice, this allows us to compute many limits in a similar fashion as before.

\begin{example}
Compute $\Lim{\point{x,y}}{\point{1,2}} \frac{2x+4y}{x-3y}$.  

\begin{explanation}
We show how the above properties are used quite explicitly.

\begin{align*}
\Lim{\point{x,y}}{\point{1,2}} \frac{2x+4y}{x-3y} & = \frac{\Lim{\point{x,y}}{\point{1,2}}(2x+4y)}{\Lim{\point{x,y}}{\point{1,2}}(x-3y)} \textrm{ by the quotient law. } \\
&=  \frac{\Lim{\point{x,y}}{\point{1,2}}2x+\Lim{\point{x,y}}{\point{1,2}} 4y}{\Lim{\point{x,y}}{\point{1,2}}x-\Lim{\point{x,y}}{\point{1,2}}3y}  \textrm{ by the sum/difference law. } \\
&=  \frac{2\left[\Lim{\point{x,y}}{\point{1,2}}x\right]+4\left[\Lim{\point{x,y}}{\point{1,2}} y\right]}{\left[\Lim{\point{x,y}}{\point{1,2}}x\right]-3\left[\Lim{\point{x,y}}{\point{1,2}}y\right]}  \textrm{ by the scalar multiple law. } \\
&= \frac{2[1]+4[2]}{[1]-3[2]} \textrm{ by the identity law. } \\
&= -2
\end{align*}
\end{explanation}
\end{example}

Essentially, the above laws allow us to evaluate limits by directly substituting values into the given function, provided the end result is a constant.  Henceforth, when a limit can be evaluated by direct substitution, we will not show the details.  

As it turns out, another old technique works well too.

%%%\begin{explanation}
   %To compute this limit, note that direct substitution leads us to the indeterminate form $\frac{0}{0}$.  Also, note that the domain of $ \frac{x^2y-xy^3}{x^2-y^4}$ is $\left\{\point{x,y} \in \R^2 \big| x^2-y^4 \neq 0 \right\}$, so all of our analysis is done away from this curve.  We proceed by factoring.
    
   % \begin{align*}
    %  \lim_{\point{x,y}\to\point{9,3}} \frac{x^2y-xy^3}{x^2-y^4}
      %&=\lim_{\point{x,y}\to\point{9,3}} \frac{xy\left(\answer[given]{x-y^2}\right)}{(x+y^2)\left(\answer[given]{x-y^2}\right)}\\
     % &=\lim_{\point{x,y}\to\point{9,3}} \frac{xy}{(x+y^2)}\\
     % &=\answer[given]{\frac{3}{2}}
   % \end{align*}
    
   % What allows us to perform the cancellation of the common factors of $x-y^2$?  Note that when determining whether a limit exists or not, we must look near, but not at, $\point{x,y}$ near $\point{9,3}$.  No matter how close a $\point{x,y}$ in the domain is to $\point{9,3}$, if $\point{x,y} \neq \point{9,3}$, $x-y^2 \neq 0$, so this cancellation is valid.
 % \end{explanation}
%\end{example}
\begin{example}
  Compute $\lim_{\point{x,y}\to\point{0,0}} \frac{x^3+xy^2}{x^2+y^2}$. 
  \begin{explanation}
   To compute this limit, note that direct substitution leads us to the indeterminate form $\frac{0}{0}$.      
    \begin{align*}
      \lim_{\point{x,y}\to\point{0,0}}\frac{x^3+xy^2}{x^2+y^2}
      &=\lim_{\point{x,y}\to\point{0,0}} \frac{\answer[given]{x}\left(x^2+y^2\right)}{\answer[given]{x^2+y^2}}\\
      &=\lim_{\point{x,y}\to\point{0,0}} x\\
      &=\answer[given]{0}
    \end{align*}
    
    What allows us to perform the cancellation of the common factors of $x^2+y^2$?  Note that when determining whether a limit exists or not, we must look near the point $\point{0,0}$, but not at the point $\point{0,0}$.  No matter how close a point  $\point{x,y}$ is to the point $\point{0,0}$, as long as  $\point{x,y} \neq \point{0,0}$, then $x^2+y^2 \neq 0$. So, this cancellation is valid.
  \end{explanation}
\end{example}



Limits exist when functions locally look like a smooth sheet. On the
other hand, limits don't exist when the function make a large jump,
\begin{onlineOnly}
  as in the case with $\arctan(y/x)$,
  \begin{center}
    \geogebra{RDr98P8J}{800}{600}% https://ggbm.at/RDr98P8J
  \end{center}
\end{onlineOnly}
or when the surface is somehow pinched.
\begin{onlineOnly}
  Here we see the function:
  \[
  F(x,y) = \frac{6x^2y}{x^4+y^2}
  \]
  \begin{center}
    \geogebra{E3jUp27u}{800}{600}%https://ggbm.at/E3jUp27u
  \end{center}
  If one approaches the origin along any line, you see the limit of
  the (composite) function is zero, by following the path on the
  surface. However, if one approaches the origin along a parabola,
  then we see the limit does not exist, as approaching along the
  parabola $y=x^2$ gives a limit of $3$, and approaching along the
  parabola $y=-x^2$ gives a limit of $-3$. Thus this is a case where
  the limit does not exist.
\end{onlineOnly}



\begin{example}
  Compute:
  \[
  \lim_{\vector{x,y}\to\vector{9,3}} \frac{x^2y-xy^3}{x^2-y^4}
  \]
  \begin{explanation}
    To compute this limit, we proceed by factoring. Write with me,
    \begin{align*}
      \lim_{\vector{x,y}\to\vector{9,3}} \frac{x^2y-xy^3}{x^2-y^4}
      &=\lim_{\vector{x,y}\to\vector{9,3}} \frac{xy\left(\answer[given]{x-y^2}\right)}{(x+y^2)\left(\answer[given]{x-y^2}\right)}\\
      &=\lim_{\vector{x,y}\to\vector{9,3}} \frac{xy}{(x+y^2)}\\
      &=\answer[given]{\frac{3}{2}}
    \end{align*}
  \end{explanation}
\end{example}



\section{Continuity}

Now we will use the idea of a limit to define continuity.

\begin{definition}
  Let $F:\R^n\to \R$ be defined on an open ball $B$ centered at
  $\vec{a}$. $F$ is \dfn{continuous} at $\vec{a}$, if
  \begin{itemize}
  \item $F(\vec{a})$ exists.
  \item $\lim_{\vec{x}\to\vec{a}} F(\vec{x})$ exists.
  \item $\lim_{\vec{x}\to\vec{a}} F(\vec{x}) = F(\vec{a}).$
  \end{itemize}
  $F$ is \dfn{continuous on an open ball} $B$ if $F$ is continuous at
  all points in $B$.
\end{definition}

To really use this definition, we need \textit{limit laws} which in
some sense are really \textit{continuity laws}.

\begin{theorem}[Limit Laws]
  Let $F:\R^n\to \R$ and $G:\R^n\to \R$ be functions of several
  variables, $b$, $L$ and $M$ are real numbers,
  \begin{align*}
    \vec{x} &= \vector{x_1,x_2,\dots,x_n}\\ \vec{a} &=
    \vector{a_1,a_2,\dots,a_n},
  \end{align*}
  where
  \[
  \lim_{\vec{x}\to\vec{a}}F(\vec{x}) = L \quad \text{and}\quad \lim_{\vec{x}\to\vec{a}} G(\vec{x}) = M.
  \]
\begin{description}
\item[Constant Law] $\lim_{\vec{x}\to \vec{a}} b = b$.
\item[Identity Law] $\lim_{\vec{x}\to \vec{a}} x_i = a_i$.
\item[Sum/Difference Law] $\lim_{\vec{x}\to \vec{a}}\big(F(\vec{x})\pm G(\vec{x})\big) = L\pm M$.
\item[Scalar Multiple Law] $\lim_{\vec{x}\to \vec{a}} b\cdot F(\vec{x}) = bL$.
\item[Product Law] $\lim_{\vec{x}\to \vec{a}} \left(F(\vec{x})\cdot G(\vec{x})\right) = LM$.
\item[Quotient Law] $\lim_{\vec{x}\to \vec{a}} \frac{F(\vec{x})}{G(\vec{x})} = \frac{L}{M}$, if $M\neq 0$.
\end{description}
\end{theorem}

\begin{question}
  True or false: If $F:\R^2\to\R$ and $G:\R^2\to\R$ are continuous
  functions on an open disk $B$, then $F\pm G$ is continuous on $B$.
  \begin{prompt}
    \begin{multipleChoice}
      \choice[correct]{True}
      \choice{False}
  \end{multipleChoice}
  \end{prompt}
\end{question}

\begin{question}
  True or false: If $F:\R^2\to\R$ and $G:\R^2\to\R$ are continuous
  functions on an open disk $B$, then $F/G$ is continuous on $B$.
  \begin{prompt}
    \begin{multipleChoice}
      \choice{True}
      \choice[correct]{False}
    \end{multipleChoice}
    \begin{feedback}
      The function $F/G$ may or may not be continuous, it depends on
      whether $G(x,y)=0$. If $G(x,y)=0$, then $F/G$ not continuous at that point.
    \end{feedback}
  \end{prompt}
\end{question}


\begin{theorem}[Composition Limit Law]
  Let $f:\R\to\R$ be a continuous function on an interval $I$. Let
  $G:\R^n\to \R$ be a function whose range is contained in (or equal
  to) $I$, Then
  \[
  \lim_{\vec{x}\to\vec{a}} f( G(\vec{x})) = f(\lim_{\vec{x}\to\vec{a}}G(\vec{x}))
  \]
\end{theorem}


\begin{corollary}[Composition of Composite Functions]
  Let $G:\R^n\to \R$ be continuous on an open disk $B$, where the
  range of $G$ on $B$ is $I$, and let $f$ be a single variable
  function that is continuous on $I$. Then
  \[
  f\circ G(\vec{x}) =f(G(\vec{x})),
  \]
  is continuous on $B$.
\end{corollary}



\begin{example}
  Show that the function
  \[
  F(x,y) = \sin(x^2\cos(y))
  \]
  is continuous for all points in $\R^2$.
  \begin{explanation}
    Let
    \[
    F_1(x,y) = x^2.
    \]
    Since $y$ is not actually used in the function, and polynomials
    \wordChoice{\choice[correct]{are continuous}\choice{are not
        continuous}}, we conclude $F_1$ is continuous everywhere. A
    similar statement can be made about
    \[
    F_2(x,y) = \cos(y).
    \]
    Setting
    \[
    F_3=F_1\cdot F_2
    \]
    we obtain a continuous function from $\R^2\to \R$. Since sine \wordChoice{\choice[correct]{is
    continuous}\choice{is not continuous}} for all real values, the composition of sine with $F_3$
    is continuous. Hence, $\sin (F_3(x,y)) = \sin(x^2\cos y)$ is
    continuous everywhere.
    \begin{onlineOnly}
      We finish by presenting you with a plot of $F$:
      \begin{center}
        \geogebra{TNETssA9}{800}{600} %https://ggbm.at/TNETssA9
      \end{center}
    \end{onlineOnly}
  \end{explanation}
\end{example}


\begin{example}
  Let
  \[
  F(x,y) = \begin{cases}
    \frac{\cos(y)\sin(x)}{x} & x\neq 0 \\
    \cos(y) & x=0
  \end{cases}
  \]
  Is $F$ continuous at $(0,0)$? Is $F$ continuous everywhere?
  \begin{explanation}
    To determine if $F$ is continuous at $(0,0)$, we need to compare
    \[
    \lim_{\vector{x,y}\to\vector{0,0}} F(x,y)\quad\text{to}\quad F(0,0).
    \]
    Applying the definition of $F$, we see that:
    \[
    F(0,0) = \answer[given]{1}
    \]
    We now consider the limit
    \[
    \lim_{\vector{x,y}\to\vector{0,0}}F(x,y).
    \]
    Substituting $0$ for $x$ and $y$ in $(\cos(y)\sin(x))/x$ returns the
    indeterminate form \zeroOverZero, so we need to do more work to
    evaluate this limit.
    
    Consider two related limits:
    \begin{align*}
      \lim_{\vector{x,y}\to\vector{0,0}} \cos(y)\\
      \lim_{\vector{x,y}\to\vector{0,0}} \frac{\sin(x)}{x}.
    \end{align*}
    The first limit does not contain $x$, and since $\cos(y)$ is
    continuous,
    \begin{align*}
    \lim_{\vector{x,y}\to\vector{0,0}} \cos(y) &=\lim_{y\to 0} \cos(y) \\
    &=\answer[given]{1}
    \end{align*}
    The second limit does not contain $y$. But we know
    \begin{align*}
      \lim_{\vector{x,y}\to\vector{0,0}} \frac{\sin(x)}{x} &= \lim_{x\to 0} \frac{\sin(x)}{x} \\
      &= \answer[given]{1}.
    \end{align*}
    Finally, we know that we can combine these two limits so that 
    $\lim_{\vector{x,y}\to\vector{0,0}} \frac{\cos(y)\sin(x)}{x}$
    \begin{align*}
      &= \lim_{\vector{x,y}\to\vector{0,0}} (\cos(y))\left(\frac{\sin(x)}{x}\right) \\ 
      &=\left(\lim_{\vector{x,y}\to\vector{0,0}} \cos(y)\right)\left(\lim_{\vector{x,y}\to\vector{0,0}} \frac{\sin(x)}{x}\right) \\
            &=\answer[given]{1}\cdot \answer[given]{1}.
    \end{align*}
    We have found that $\lim_{\vector{x,y}\to\vector{0,0}} \frac{\cos(y)\sin(x)}{x} =
    F(0,0)$, so $F$ is continuous at $(0,0)$.

    A similar analysis shows that $F$ is continuous at all points in
    $\mathbb{R}^2$. As long as $x\neq0$, we can evaluate the limit
    directly; when $x=0$, a similar analysis shows that the limit is $\cos
    y$. Thus we can say that $F$ is continuous everywhere.
    \begin{onlineOnly}
      We finish by presenting you with a plot of $F$:
      \begin{center}
        \geogebra{VK6thpMa}{800}{600} %https://ggbm.at/VK6thpMa
      \end{center}
    \end{onlineOnly}
  \end{explanation}
\end{example}


\subsection{When limits don't exist}

When dealing with functions of a single variable we often considered
one-sided limits and stated
\[
\lim_{x\to a}f(x) = L
\]
if and only if
\[
\lim_{x\to a^+}f(x) =L \quad\textbf{and}\quad \lim_{x\to a^-}f(x) =L.
\]
That is, the limit is $L$ if and only if $F$ approaches $L$ when
$x$ approaches $a$ from \textbf{either} direction.

In $\R^n$ when $n>2$ there are \textbf{infinite paths} from which
$\vec{x}$ might approach $\vec{a}$. Now we have a fact, 
\[
\lim_{\vec{x}\to \vec{a}}F(\vec{x}) = L
\]
if and only if
\[
F(\vec{x})\to  L \quad{as}\quad \vec{x}\to \vec{a}
\]
along every path.
\begin{quote}
  \textbf{If it is possible to arrive at different limiting values by
    approaching along different paths, the limit does not exist.}
\end{quote}
This is analogous to the left and right hand limits of single variable
functions not being equal, implying that the limit does not exist.
Our theorems tell us that we can evaluate most limits quite simply,
without worrying about paths. When indeterminate forms arise, the
limit may or may not exist. If it does exist, it can be difficult to
prove this as we need to show the same limiting value is obtained
regardless of the path chosen.  The case where the limit does not
exist is often easier to deal with, for we can often pick two paths
along which the limit is different.


\begin{example}
  Show
  \[
  \lim_{\vector{x,y}\to\vector{0,0}} \frac{3xy}{x^2+y^2}
  \]
  does not exist by finding the limits along the lines $y=mx$.
  \begin{explanation}
    Evaluating $\lim_{\vector{x,y}\to\vector{0,0}} \frac{3xy}{x^2+y^2}$ along
    the lines $y=mx$ means replace all $y$'s with $mx$ and evaluating
    the resulting limit:
    \begin{align*}
      \lim_{\vector{x,mx}\to\vector{0,0}} \frac{3x(\answer[given]{m x})}{x^2+(\answer[given]{mx})^2} &=\lim_{x\to0} \frac{3mx^2}{x^2(m^2+1)}\\
      &= \lim_{x\to 0}\frac{3m}{m^2+1}\\
      &= \answer[given]{\frac{3m}{m^2+1}}.
    \end{align*}
    While the limit exists for each choice of $m$, we get a
    \textit{different} limit for each choice of $m$. Suppose $m=0$,
    then:
    \[
    \lim_{x\to 0}\frac{3m}{m^2+1} = \answer[given]{0}
    \]
    Now suppose that $m=1$, then:
    \[
    \lim_{x\to 0}\frac{3m}{m^2+1} = \answer[given]{3/2}
    \]
    Since we find differing limiting values when computing the limit
    along different paths, we must conclude that the limit does not
    exist.
    \begin{onlineOnly}
      We finish by presenting you with a plot of $F$:
      \begin{center}
        \geogebra{e2pAxbeP}{800}{600} %https://ggbm.at/e2pAxbeP
      \end{center}
    \end{onlineOnly}
  \end{explanation}
\end{example}

\begin{example}
  Show
  \[
  \lim_{\vector{x,y}\to\vector{0,0}} \frac{\sin(xy)}{x+y}
  \]
  does not exist by finding the limit along the path $y=-\sin(x)$.
  \begin{explanation}
    Let
    \[
    F(x,y) = \frac{\sin(xy)}{x+y}.
    \]
    We will show that
    \[
    \lim_{\vector{x,y}\to\vector{0,0}} F(x,y)
    \]
    does not exist by finding the limit along the path
    $y=-\sin(x)$. First, however, let's try the same trick that worked
    before and see what happens. Consider the limits found along the
    lines $y=mx$ as done above.
    \begin{align*}
      \lim_{\vector{x,mx}\to\vector{0,0}} \frac{\sin\big(x(\answer[given]{mx})\big)}{x+\answer[given]{mx}} &= \lim_{x\to 0} \frac{\sin (mx^2)}{x(m+1)} \\
      &= \lim_{x\to 0} \frac{\sin(mx^2)}{x}\cdot\answer[given]{\frac{1}{m+1}}.
    \end{align*}
    By applying L'H\^opital's Rule, we can show this limit is $0$
    \emph{except} when $m=-1$, that is, along the line $y=-x$. This
    line is not in the domain of $F$, so we have found the following
    fact: along every line $y=mx$ in the domain of $F$,
    \[
    \lim_{\vector{x,y}\to\vector{0,0}} F(x,y)=0.
    \]
    Now consider the limit along the path $y=-\sin(x)$:
    \begin{align*}
      \lim_{\vector{x,-\sin(x)}\to\vector{0,0}} \frac{\sin\big(-x\sin(x)\big)}{x-\sin(x)} &= \lim_{x\to0} \frac{\sin\big(-x\sin(x)\big)}{x-\sin(x)}
    \end{align*}
    Now apply L'H\^opital's Rule twice to find a limit is of the form
    \numOverZero.  Hence the limit does not exist.  Step back and
    consider what we have just discovered.
    \begin{itemize}
    \item Along any line $y=mx$ in the domain of the $F(x,y)$, the
      limit is $0$.
    \item However, along the path $y=-\sin(x)$, which lies in the
      domain of the $F(x,y)$ for all $x\neq 0$, the limit does not
      exist.
    \end{itemize}
    Since the limit is not the same along every path to $(0,0)$, we say
    $\lim_{\vector{x,y}\to\vector{0,0}}\frac{\sin(xy)}{x+y}$ does not exist.
    \begin{onlineOnly}
      We finish by presenting you with a plot of $F$:
      \begin{center}
        \geogebra{vBrjrPkN}{800}{600} %https://ggbm.at/vBrjrPkN
      \end{center}
    \end{onlineOnly}
  \end{explanation}
\end{example}



\end{document}
