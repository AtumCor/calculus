\documentclass{ximera}

%\usepackage{todonotes}
%\usepackage{mathtools} %% Required for wide table Curl and Greens
%\usepackage{cuted} %% Required for wide table Curl and Greens
\newcommand{\todo}{}

\usepackage{esint} % for \oiint
\ifxake%%https://math.meta.stackexchange.com/questions/9973/how-do-you-render-a-closed-surface-double-integral
\renewcommand{\oiint}{{\large\bigcirc}\kern-1.56em\iint}
\fi


\graphicspath{
  {./}
  {ximeraTutorial/}
  {basicPhilosophy/}
  {functionsOfSeveralVariables/}
  {normalVectors/}
  {lagrangeMultipliers/}
  {vectorFields/}
  {greensTheorem/}
  {shapeOfThingsToCome/}
  {dotProducts/}
  {partialDerivativesAndTheGradientVector/}
  {../productAndQuotientRules/exercises/}
  {../motionAndPathsInSpace/exercises/}
  {../normalVectors/exercisesParametricPlots/}
  {../continuityOfFunctionsOfSeveralVariables/exercises/}
  {../partialDerivativesAndTheGradientVector/exercises/}
  {../directionalDerivativeAndChainRule/exercises/}
  {../commonCoordinates/exercisesCylindricalCoordinates/}
  {../commonCoordinates/exercisesSphericalCoordinates/}
  {../greensTheorem/exercisesCurlAndLineIntegrals/}
  {../greensTheorem/exercisesDivergenceAndLineIntegrals/}
  {../shapeOfThingsToCome/exercisesDivergenceTheorem/}
  {../greensTheorem/}
  {../shapeOfThingsToCome/}
  {../separableDifferentialEquations/exercises/}
  {vectorFields/}
}

\newcommand{\mooculus}{\textsf{\textbf{MOOC}\textnormal{\textsf{ULUS}}}}

\usepackage{tkz-euclide}\usepackage{tikz}
\usepackage{tikz-cd}
\usetikzlibrary{arrows}
\tikzset{>=stealth,commutative diagrams/.cd,
  arrow style=tikz,diagrams={>=stealth}} %% cool arrow head
\tikzset{shorten <>/.style={ shorten >=#1, shorten <=#1 } } %% allows shorter vectors

\usetikzlibrary{backgrounds} %% for boxes around graphs
\usetikzlibrary{shapes,positioning}  %% Clouds and stars
\usetikzlibrary{matrix} %% for matrix
\usepgfplotslibrary{polar} %% for polar plots
\usepgfplotslibrary{fillbetween} %% to shade area between curves in TikZ
%\usetkzobj{all} obsolete

\usepackage[makeroom]{cancel} %% for strike outs
%\usepackage{mathtools} %% for pretty underbrace % Breaks Ximera
%\usepackage{multicol}
\usepackage{pgffor} %% required for integral for loops



%% http://tex.stackexchange.com/questions/66490/drawing-a-tikz-arc-specifying-the-center
%% Draws beach ball
\tikzset{pics/carc/.style args={#1:#2:#3}{code={\draw[pic actions] (#1:#3) arc(#1:#2:#3);}}}



\usepackage{array}
\setlength{\extrarowheight}{+.1cm}
\newdimen\digitwidth
\settowidth\digitwidth{9}
\def\divrule#1#2{
\noalign{\moveright#1\digitwidth
\vbox{\hrule width#2\digitwidth}}}





\newcommand{\RR}{\mathbb R}
\newcommand{\R}{\mathbb R}
\newcommand{\N}{\mathbb N}
\newcommand{\Z}{\mathbb Z}

\newcommand{\sagemath}{\textsf{SageMath}}


%\renewcommand{\d}{\,d\!}
\renewcommand{\d}{\mathop{}\!d}
\newcommand{\dd}[2][]{\frac{\d #1}{\d #2}}
\newcommand{\pp}[2][]{\frac{\partial #1}{\partial #2}}
\renewcommand{\l}{\ell}
\newcommand{\ddx}{\frac{d}{\d x}}

\newcommand{\zeroOverZero}{\ensuremath{\boldsymbol{\tfrac{0}{0}}}}
\newcommand{\inftyOverInfty}{\ensuremath{\boldsymbol{\tfrac{\infty}{\infty}}}}
\newcommand{\zeroOverInfty}{\ensuremath{\boldsymbol{\tfrac{0}{\infty}}}}
\newcommand{\zeroTimesInfty}{\ensuremath{\small\boldsymbol{0\cdot \infty}}}
\newcommand{\inftyMinusInfty}{\ensuremath{\small\boldsymbol{\infty - \infty}}}
\newcommand{\oneToInfty}{\ensuremath{\boldsymbol{1^\infty}}}
\newcommand{\zeroToZero}{\ensuremath{\boldsymbol{0^0}}}
\newcommand{\inftyToZero}{\ensuremath{\boldsymbol{\infty^0}}}



\newcommand{\numOverZero}{\ensuremath{\boldsymbol{\tfrac{\#}{0}}}}
\newcommand{\dfn}{\textbf}
%\newcommand{\unit}{\,\mathrm}
\newcommand{\unit}{\mathop{}\!\mathrm}
\newcommand{\eval}[1]{\bigg[ #1 \bigg]}
\newcommand{\seq}[1]{\left( #1 \right)}
\renewcommand{\epsilon}{\varepsilon}
\renewcommand{\phi}{\varphi}


\renewcommand{\iff}{\Leftrightarrow}

\DeclareMathOperator{\arccot}{arccot}
\DeclareMathOperator{\arcsec}{arcsec}
\DeclareMathOperator{\arccsc}{arccsc}
\DeclareMathOperator{\si}{Si}
\DeclareMathOperator{\scal}{scal}
\DeclareMathOperator{\sign}{sign}


%% \newcommand{\tightoverset}[2]{% for arrow vec
%%   \mathop{#2}\limits^{\vbox to -.5ex{\kern-0.75ex\hbox{$#1$}\vss}}}
\newcommand{\arrowvec}[1]{{\overset{\rightharpoonup}{#1}}}
%\renewcommand{\vec}[1]{\arrowvec{\mathbf{#1}}}
\renewcommand{\vec}[1]{{\overset{\boldsymbol{\rightharpoonup}}{\mathbf{#1}}}\hspace{0in}}

\newcommand{\point}[1]{\left(#1\right)} %this allows \vector{ to be changed to \vector{ with a quick find and replace
\newcommand{\pt}[1]{\mathbf{#1}} %this allows \vec{ to be changed to \vec{ with a quick find and replace
\newcommand{\Lim}[2]{\lim_{\point{#1} \to \point{#2}}} %Bart, I changed this to point since I want to use it.  It runs through both of the exercise and exerciseE files in limits section, which is why it was in each document to start with.

\DeclareMathOperator{\proj}{\mathbf{proj}}
\newcommand{\veci}{{\boldsymbol{\hat{\imath}}}}
\newcommand{\vecj}{{\boldsymbol{\hat{\jmath}}}}
\newcommand{\veck}{{\boldsymbol{\hat{k}}}}
\newcommand{\vecl}{\vec{\boldsymbol{\l}}}
\newcommand{\uvec}[1]{\mathbf{\hat{#1}}}
\newcommand{\utan}{\mathbf{\hat{t}}}
\newcommand{\unormal}{\mathbf{\hat{n}}}
\newcommand{\ubinormal}{\mathbf{\hat{b}}}

\newcommand{\dotp}{\bullet}
\newcommand{\cross}{\boldsymbol\times}
\newcommand{\grad}{\boldsymbol\nabla}
\newcommand{\divergence}{\grad\dotp}
\newcommand{\curl}{\grad\cross}
%\DeclareMathOperator{\divergence}{divergence}
%\DeclareMathOperator{\curl}[1]{\grad\cross #1}
\newcommand{\lto}{\mathop{\longrightarrow\,}\limits}

\renewcommand{\bar}{\overline}

\colorlet{textColor}{black}
\colorlet{background}{white}
\colorlet{penColor}{blue!50!black} % Color of a curve in a plot
\colorlet{penColor2}{red!50!black}% Color of a curve in a plot
\colorlet{penColor3}{red!50!blue} % Color of a curve in a plot
\colorlet{penColor4}{green!50!black} % Color of a curve in a plot
\colorlet{penColor5}{orange!80!black} % Color of a curve in a plot
\colorlet{penColor6}{yellow!70!black} % Color of a curve in a plot
\colorlet{fill1}{penColor!20} % Color of fill in a plot
\colorlet{fill2}{penColor2!20} % Color of fill in a plot
\colorlet{fillp}{fill1} % Color of positive area
\colorlet{filln}{penColor2!20} % Color of negative area
\colorlet{fill3}{penColor3!20} % Fill
\colorlet{fill4}{penColor4!20} % Fill
\colorlet{fill5}{penColor5!20} % Fill
\colorlet{gridColor}{gray!50} % Color of grid in a plot

\newcommand{\surfaceColor}{violet}
\newcommand{\surfaceColorTwo}{redyellow}
\newcommand{\sliceColor}{greenyellow}




\pgfmathdeclarefunction{gauss}{2}{% gives gaussian
  \pgfmathparse{1/(#2*sqrt(2*pi))*exp(-((x-#1)^2)/(2*#2^2))}%
}


%%%%%%%%%%%%%
%% Vectors
%%%%%%%%%%%%%

%% Simple horiz vectors
\renewcommand{\vector}[1]{\left\langle #1\right\rangle}


%% %% Complex Horiz Vectors with angle brackets
%% \makeatletter
%% \renewcommand{\vector}[2][ , ]{\left\langle%
%%   \def\nextitem{\def\nextitem{#1}}%
%%   \@for \el:=#2\do{\nextitem\el}\right\rangle%
%% }
%% \makeatother

%% %% Vertical Vectors
%% \def\vector#1{\begin{bmatrix}\vecListA#1,,\end{bmatrix}}
%% \def\vecListA#1,{\if,#1,\else #1\cr \expandafter \vecListA \fi}

%%%%%%%%%%%%%
%% End of vectors
%%%%%%%%%%%%%

%\newcommand{\fullwidth}{}
%\newcommand{\normalwidth}{}



%% makes a snazzy t-chart for evaluating functions
%\newenvironment{tchart}{\rowcolors{2}{}{background!90!textColor}\array}{\endarray}

%%This is to help with formatting on future title pages.
\newenvironment{sectionOutcomes}{}{}



%% Flowchart stuff
%\tikzstyle{startstop} = [rectangle, rounded corners, minimum width=3cm, minimum height=1cm,text centered, draw=black]
%\tikzstyle{question} = [rectangle, minimum width=3cm, minimum height=1cm, text centered, draw=black]
%\tikzstyle{decision} = [trapezium, trapezium left angle=70, trapezium right angle=110, minimum width=3cm, minimum height=1cm, text centered, draw=black]
%\tikzstyle{question} = [rectangle, rounded corners, minimum width=3cm, minimum height=1cm,text centered, draw=black]
%\tikzstyle{process} = [rectangle, minimum width=3cm, minimum height=1cm, text centered, draw=black]
%\tikzstyle{decision} = [trapezium, trapezium left angle=70, trapezium right angle=110, minimum width=3cm, minimum height=1cm, text centered, draw=black]

\author{Bart Snapp \and Jim Talamo}

\outcome{Define a function of several variables.}
\outcome{Find the domain and range of a function of several variables.}


\title[Dig-In:]{Functions of several variables}


\begin{document}
\begin{abstract}
  We introduce functions that take vectors or points as inputs and
  output a number.
\end{abstract}
\maketitle

The world is constantly changing. Sometimes this change is very slow,
other times it is shockingly fast. Consider \textit{Meteor Crater} in
northern Arizona.
%% http://commons.wikimedia.org/wiki/File:Meteorcrater.jpg
%% CC-BY-3.0 Shane Torgerson http://commons.wikimedia.org/wiki/User:Shane.torgerson
\begin{image}[3in]
  \includegraphics{meteorcrater.jpg}
\end{image}
This area was once grasslands and woodlands inhabited by bison,
camels, wooly mammoths, and giant ground sloths. During the
Pleistocene epoch, a meteor only $40$ meters in diameter collided
with the Earth and this changed very quickly. The collision released
around $4\times10^{16}$ joules of energy, comparable to the energy
released by a large nuclear weapon. A fireball extended out $10$
kilometers from the center of the impact, destroying all life in its
wake. It is estimated it took one hundred years for the local plant
and animal life to repopulate the area.  Fifty thousand years later,
the remains of the impact crater are still intact on our ever-changing
Earth.

To help us understand events like these, we need to precisely describe
what we are observing (in this case, the crater).  To do this we use a
\textit{contour map}, often called a \textit{topographical map}:
%% Based on Meteor Crater USGS Topographic Map
\begin{image}[2in]
\begin{tikzpicture}[y=0.80pt, x=0.80pt, yscale=-1.000000, xscale=1.000000, inner sep=0pt, outer sep=0pt]
\path[draw=black,even odd rule] (308.0482,210.3114) -- (266.7601,199.6668) --
  (248.6965,201.6022) -- (236.1165,199.9894) -- (221.9237,199.6668) --
  (212.2468,201.6022) -- (190.3125,203.8602) -- (176.4423,213.2145) --
  (164.8300,225.7945) -- (158.0561,237.7293) -- (152.2500,249.9868) --
  (149.6695,263.2119) -- (151.2823,270.6308) -- (149.3469,281.9206) --
  (150.3146,293.2103) -- (149.9921,301.9195) -- (154.1854,305.7903) --
  (153.8628,313.8543) -- (156.4433,321.2733) -- (161.2818,327.5633) --
  (170.9587,339.0143) -- (170.3136,344.1753) -- (175.4746,349.0138) --
  (180.3130,348.0461) -- (183.5387,351.2717) -- (183.5387,355.7876) --
  (188.3771,355.7876) -- (192.5704,359.6584) -- (192.2479,363.5291) --
  (198.3766,363.2066) -- (200.9571,367.0773) -- (205.7955,369.3353) --
  (207.0858,374.4963) -- (212.5694,376.4317) -- (213.5371,381.5927) --
  (223.2140,385.9473) -- (233.6973,385.7860) -- (246.7611,395.6242) --
  (256.1155,396.1081) -- (258.0508,400.9465) -- (272.8888,400.3014) --
  (282.5657,400.7852) -- (295.1457,398.0434) -- (304.6613,397.5596) --
  (319.0154,391.2696) -- (328.3697,381.5927) -- (333.2082,379.9799) --
  (339.3369,370.6255) -- (339.6594,363.8517) -- (351.9168,343.2076) --
  (352.8845,326.7569) -- (355.7876,320.6282) -- (360.6261,303.5323) --
  (363.2066,289.6621) -- (360.6261,286.4364) -- (359.9809,277.4047) --
  (362.8840,273.5339) -- (358.0455,263.5344) .. controls (358.0455,263.5344) and
  (349.6589,254.5027) .. (349.3363,251.2770) .. controls (349.0138,248.0514) and
  (346.4333,241.9227) .. (346.4333,241.9227) -- (341.5948,236.4391) --
  (339.3369,230.6329) -- (332.8856,223.2140) -- (330.9502,221.6012) --
  (328.3697,220.7948) -- (317.8864,211.7630) -- cycle;
\path[draw=black,even odd rule] (124.8321,258.6960) -- (123.5418,281.2754) --
  (126.1223,302.2421) -- (129.9931,315.1446) -- (134.1864,324.8215) --
  (133.8639,330.6277) -- (135.7993,336.4338) -- (137.7346,341.9174) --
  (144.5085,351.2717) -- (152.8951,361.9163) -- (153.2177,369.0127) --
  (160.3141,375.7866) -- (159.6690,381.5927) -- (165.7977,383.5281) --
  (173.8618,395.7855) -- (182.2484,400.3014) -- (190.9576,410.9460) --
  (199.3443,413.2039) -- (203.5376,418.0424) -- (217.0853,420.9454) --
  (227.7299,425.4613) -- (229.0201,427.3967) -- (235.1488,425.4613) --
  (302.8872,420.3003) -- (324.1764,414.4942) -- (337.7241,408.6880) --
  (343.5302,406.7527) -- (347.7235,399.0111) -- (358.6907,387.3988) --
  (362.8840,385.4635) -- (364.8194,378.6896) -- (369.9804,373.2060) --
  (374.4963,373.2060) -- (379.9799,363.8517) -- (376.7542,359.0132) --
  (381.2701,350.9492) -- (384.4958,341.5948) -- (390.9470,337.7241) --
  (389.9793,328.0471) -- (387.3988,321.2733) -- (389.0117,304.5000) --
  (386.7537,300.3067) -- (388.0440,296.1134) -- (390.9470,291.9200) --
  (389.9793,288.6944) -- (385.1409,281.9206) -- (385.1409,276.1144) --
  (379.0122,271.2760) -- (380.9476,254.1801) -- (376.1091,248.0514) --
  (376.7542,240.3099) -- (372.8835,236.1165) -- (373.5286,229.0201) --
  (369.9804,223.8591) -- (365.7871,224.1817) -- (360.3035,216.1176) --
  (359.9809,214.1822) -- (354.8199,209.9889) -- (352.2394,206.1181) --
  (346.4333,204.8279) -- (340.6271,194.5058) -- (332.2405,192.2479) --
  (325.4666,188.6997) -- (322.5636,188.3771) -- (320.6282,183.5387) --
  (309.3385,182.8935) -- (307.4031,180.9582) -- (299.6616,181.2807) --
  (295.3069,176.9261) -- (286.7590,177.4100) -- (278.0498,177.0874) --
  (271.9211,179.0228) -- (265.1473,177.0874) -- (258.3734,177.7325) --
  (249.0191,178.0551) -- (242.5678,176.7648) -- (239.0196,179.6679) --
  (223.8591,179.0228) -- (218.0530,180.9582) -- (201.6022,184.1838) --
  (182.2484,188.0546) -- (164.5074,194.8284) -- (156.1208,196.4412) --
  (153.2177,200.6345) -- (145.7987,207.7309) -- (131.6059,226.4396) --
  (130.3157,241.4388) -- (127.7352,245.9547) -- cycle;
\path[draw=black,even odd rule] (135.3154,191.4415) -- (144.3915,186.6855) --
  (147.4730,180.1045) -- (170.9587,176.7648) -- (190.3125,173.2166) --
  (203.8602,167.7331) -- (224.1817,165.1525) -- (245.7934,159.3464) .. controls
  (245.7934,159.3464) and (254.1801,161.9269) .. (257.0832,160.9592) .. controls
  (259.9862,159.9915) and (266.7601,158.7013) .. (266.7601,158.7013) .. controls
  (266.7601,158.7013) and (276.4370,160.9592) .. (278.3724,160.9592) --
  (292.2426,160.9592) .. controls (294.1780,160.9592) and (299.3390,158.7013) ..
  (301.5969,161.2818) .. controls (303.8549,163.8623) and (306.4354,166.7654) ..
  (306.4354,166.7654) -- (317.4025,165.4751) .. controls (317.4025,165.4751) and
  (324.1764,165.4751) .. (325.4666,166.4428) .. controls (326.7569,167.4105) and
  (339.9820,176.4423) .. (339.9820,176.4423) -- (345.4656,179.0228) --
  (350.6266,178.0551) -- (357.7230,188.3771) -- (367.0773,194.5058) --
  (368.0450,199.0217) -- (374.4963,203.2150) -- (376.1091,209.3438) --
  (385.4635,214.1822) -- (386.1086,218.0530) -- (392.8824,227.0848) --
  (392.8824,238.6970) -- (401.9142,260.9539) -- (405.7850,269.3406) --
  (405.1398,282.5657) -- (408.3655,295.7908) -- (408.3655,299.9841) --
  (406.7527,344.1753) -- (407.7203,350.6266) -- (398.0434,356.7553) --
  (395.7855,370.3030) -- (388.0440,375.1414) -- (387.7214,384.1732) --
  (380.6250,387.0763) -- (359.6584,406.4301) -- (357.7230,417.0747) --
  (350.9492,421.9131) -- (326.1118,423.5260) -- (314.1769,431.2675) --
  (306.1128,431.5900) -- (297.0810,435.4608) -- (232.8909,441.2670) --
  (227.4073,442.2346) -- (221.9237,447.3957) -- (215.1499,441.5895) --
  (208.3761,439.9767) -- (188.6997,431.9126) -- (180.9582,433.2029) --
  (174.5069,429.6547) -- (173.5392,416.4296) -- (169.6684,411.9137) --
  (160.6366,411.9137) -- (160.6366,406.4301) -- (150.9597,397.3983) --
  (141.2828,396.7532) -- (142.2505,388.6891) -- (139.3475,386.4311) --
  (139.6700,381.9153) -- (137.4121,378.0445) -- (134.5090,374.4963) --
  (125.4772,357.4004) -- (124.1870,351.9168) -- (116.4455,344.4979) --
  (117.0906,338.0466) -- (114.9433,328.8259) -- (112.0794,315.6008) --
  (108.7039,305.7903) -- (109.9942,300.3067) -- (107.0911,289.9846) --
  (107.0911,279.0175) -- (106.7685,267.4052) -- (104.8332,264.5021) --
  (106.1234,245.4709) -- (109.6716,241.2775) -- (109.3491,229.9878) --
  (115.4778,225.7945) -- (115.1552,219.0207) -- (120.6388,212.2468) --
  (120.9613,206.7632) -- (135.7993,190.7963);
\path[draw=black,even odd rule] (92.8983,240.3099) -- (90.9629,250.3093) -- (89.6727,271.9211)
  -- (91.9306,285.7913) -- (94.5111,295.1457) -- (96.4465,296.4359) --
  (94.8337,309.9836) -- (100.3173,320.3056) -- (100.6398,328.6923) --
  (104.8332,334.4984) -- (102.5752,340.3046) -- (106.7685,351.2717) --
  (111.6070,356.4327) -- (111.2844,363.2066) -- (119.3485,373.8512) --
  (119.6711,381.2701) -- (130.3157,388.3665) -- (131.6059,392.5599) --
  (132.5736,394.8178) -- (132.8962,401.2691) -- (139.3475,409.6557) --
  (147.7341,411.5911) -- (147.7341,418.6875) -- (158.7013,424.1711) --
  (162.2495,425.7839) -- (160.3141,431.9126) -- (177.7325,449.3310) --
  (187.0869,445.7828) -- (197.7315,447.0731) -- (212.5694,458.0403) --
  (225.7945,461.9110) -- (232.5683,456.1049) -- (243.5355,452.5567) --
  (256.1155,453.2018) -- (259.6637,450.2987) -- (272.2436,450.9439) --
  (285.1462,450.9439) -- (297.0810,447.3957) -- (310.6287,437.7188) --
  (321.2733,432.8803) -- (347.0784,432.8803) -- (362.5614,428.0418) --
  (371.2707,427.0742) -- (372.8835,416.1070) -- (382.8829,404.1721) --
  (393.2050,392.8824) -- (400.3014,380.6250) -- (405.4624,377.3994) --
  (408.0429,370.6255) -- (417.7198,358.0456) -- (418.3649,333.8533) --
  (417.0747,306.4354) -- (418.3649,296.4359) -- (415.4619,288.0493) --
  (415.4619,270.9534) -- (405.7850,241.2775) -- (404.1721,235.1488) --
  (403.2044,225.1494) -- (392.2373,209.0212) -- (380.6250,197.7315) --
  (372.8835,183.5387) -- (355.7876,167.7331) -- (345.4656,165.4751) --
  (325.7892,154.5079) -- (304.8226,153.5403) -- (297.4036,145.7987) --
  (252.5673,151.6049) -- (232.2458,149.0244) -- (206.4407,154.8305) --
  (189.9899,166.1202) -- (165.4751,172.5715) -- (143.8633,167.7331) --
  (137.0895,177.7325) -- (118.7034,191.6028) -- (113.5424,199.9894) --
  (108.0588,207.4084) -- (103.5429,211.6017) -- cycle;
\path[draw=black,even odd rule] (94.7701,206.1986) -- (94.4280,217.1467) -- (87.1292,224.9017)
  -- (84.0500,237.9026) -- (78.0057,249.8772) -- (76.0670,259.7990) --
  (79.7164,268.3522) -- (80.0585,297.3193) -- (83.9360,304.9602) --
  (83.5938,313.6275) -- (94.6561,345.5597) -- (94.1999,355.2534) --
  (99.5599,367.5700) -- (108.5694,377.1497) -- (108.2272,385.9310) --
  (119.5175,397.5635) -- (122.0265,402.5814) -- (122.5967,409.7661) --
  (127.9567,413.5295) -- (158.4063,442.2685) -- (162.8540,448.4268) --
  (162.7400,451.0498) -- (171.0652,458.4627) -- (178.0218,460.6295) --
  (186.0048,457.5503) -- (193.8738,457.2082) -- (200.6024,459.3750) --
  (205.8484,465.1912) -- (210.7523,471.9198) -- (228.6571,470.4372) --
  (243.4827,462.2261) -- (269.8268,461.0857) -- (274.0464,459.4891) --
  (282.7137,459.9452) -- (287.5035,455.3835) -- (290.5827,455.4975) --
  (291.9512,457.7784) -- (296.6270,457.7784) -- (305.4083,447.5145) --
  (311.6807,445.9179) -- (319.3216,441.4702) -- (344.4112,442.1544) --
  (352.3942,445.5757) -- (360.9475,444.8915) -- (367.4480,440.6719) --
  (371.6676,440.6719) -- (377.4838,437.1365) -- (382.1596,429.0394) --
  (382.0455,416.2666) -- (387.0634,414.8981) -- (392.8796,406.9150) --
  (396.7571,406.3448) -- (400.9767,402.0112) .. controls (400.9767,402.0112) and
  (400.5206,400.5286) .. (401.0908,400.1865) .. controls (401.6610,399.8443) and
  (403.4857,399.1601) .. (403.4857,399.1601) .. controls (403.4857,399.1601) and
  (405.5385,394.2562) .. (405.7666,393.8000) .. controls (405.9946,393.3438) and
  (406.1087,389.3523) .. (406.1087,389.3523) -- (420.7062,372.3599) --
  (421.2765,369.0526) -- (425.9522,365.7453) -- (429.3735,359.4729) --
  (429.4876,347.9546) -- (427.2067,342.5945) -- (428.6893,338.7170) --
  (430.6280,334.4974) -- (428.9174,321.1544) -- (424.6978,310.0921) --
  (425.8382,298.8018) -- (426.4084,291.5031) -- (424.2416,287.2834) --
  (423.3292,275.9931) -- (416.7147,258.4305) -- (417.9692,246.1138) --
  (413.2934,242.2363) -- (411.1266,223.9894) -- (401.5469,204.6020) --
  (393.6780,199.8121) -- (383.6421,188.0657) -- (381.3613,179.3984) --
  (371.5535,171.0732) -- (369.5007,164.2306) -- (363.0003,158.0722) --
  (347.2623,155.9054) -- (336.8843,146.4398) -- (314.8739,143.4747) --
  (306.7768,139.3691) -- (293.6619,140.0534) -- (288.9861,135.9478) --
  (259.1067,136.2899) -- (256.0275,134.1231) -- (250.3253,135.3776) --
  (244.8513,137.3163) -- (228.6571,139.4832) -- (217.1387,143.9308) --
  (210.6220,146.5882) -- (201.4952,148.2206) -- (188.4551,159.3058) --
  (172.2886,158.5189) -- (160.4591,159.1573) -- (138.9049,151.8585) --
  (132.0623,155.1071) -- (130.0095,163.4323) -- (133.3168,174.9507) .. controls
  (133.3168,174.9507) and (129.3253,178.1439) .. (128.7550,178.1439) .. controls
  (128.1848,178.1439) and (121.2282,184.8724) .. (121.2282,184.8724) --
  (117.0086,186.1269) -- (105.7183,194.1100) -- (103.3234,199.5840) -- cycle;
\path[draw=black,even odd rule] (179.8292,113.3811) .. controls (179.8292,113.3811) and
  (182.6516,105.5589) .. (182.6516,105.0751) .. controls (182.6516,104.5912) and
  (181.1194,89.9146) .. (181.1194,89.9146) -- (170.0716,83.3827) --
  (166.9266,82.2537) -- (164.3461,86.8502) -- (156.1208,94.4305) --
  (120.1549,114.8326) -- (118.0583,117.8970) -- (118.0583,124.6708) --
  (111.1231,125.9611) -- (106.9298,127.8965) .. controls (106.9298,127.8965) and
  (104.1880,125.9611) .. (103.3816,126.1223) .. controls (102.5752,126.2836) and
  (101.9301,128.7029) .. (101.9301,128.7029) -- (104.3493,132.5736) --
  (102.4139,135.6380) -- (88.3824,136.2831) .. controls (88.3824,136.2831) and
  (82.2537,129.3480) .. (81.6086,129.3480) .. controls (80.9635,129.3480) and
  (72.8994,128.8641) .. (72.8994,128.8641) -- (70.3189,135.7993) --
  (72.5768,146.2826) -- (75.3186,151.7661) .. controls (75.3186,151.7661) and
  (73.8671,154.3467) .. (73.7058,153.7015) .. controls (73.5445,153.0564) and
  (57.9002,140.9603) .. (57.9002,140.9603) .. controls (57.9002,140.9603) and
  (54.8358,146.4438) .. (54.8358,147.2503) .. controls (54.8358,148.0567) and
  (53.3843,151.6049) .. (52.7391,151.6049) .. controls (52.0940,151.6049) and
  (46.2879,151.1210) .. (44.9976,151.1210) .. controls (43.7074,151.1210) and
  (38.2238,153.0564) .. (38.2238,153.0564) -- (35.8046,167.4105) --
  (39.3528,172.0877) -- (37.4174,178.2164) -- (29.0307,182.2484) --
  (26.1276,187.2482) -- (18.7087,197.8927) -- (24.6761,204.3440) --
  (17.7410,210.3114) -- (19.0312,215.9563) -- (20.6441,220.4722) --
  (28.5469,224.5042) -- (32.7402,232.2458) -- (36.6110,239.9873) --
  (38.5463,245.7934) -- (35.4820,251.1157) -- (29.6758,252.4060) --
  (29.6758,256.7606) -- (22.4182,267.8890) -- (24.3535,275.9531) --
  (23.3859,291.9200) -- (22.9020,295.1457) -- (18.0636,310.3061) --
  (21.2892,318.6928) -- (28.2243,321.7572) -- (32.0951,321.7572) --
  (29.6758,333.8533) -- (29.8371,338.2079) -- (22.2569,338.2079) --
  (19.8377,344.4979) -- (29.3533,364.9807) -- (28.2243,376.4317) --
  (34.9981,385.4635) -- (42.2558,392.2373) -- (41.9333,397.8822) --
  (29.8371,400.3014) -- (29.0307,416.4296) -- (30.1597,429.8159) --
  (39.9979,442.3959) -- (42.4171,449.9762) -- (56.2873,468.2010) --
  (55.8035,470.6202) -- (49.3522,476.2651) -- (38.2238,476.4264) --
  (29.3533,466.5882) -- (23.5471,466.7495) -- (26.4502,472.0718) --
  (24.1923,482.2325) -- (24.5148,488.2000) -- (35.3207,496.7479) --
  (49.8361,497.2317) -- (52.7391,501.1025) -- (51.6102,508.5215) --
  (47.2556,511.7471) .. controls (47.2556,511.7471) and (47.4168,514.1663) ..
  (48.0620,514.0050) .. controls (48.7071,513.8438) and (54.8358,513.1986) ..
  (54.8358,513.1986) .. controls (54.8358,513.1986) and (63.8676,506.4248) ..
  (64.5127,506.4248) .. controls (65.1578,506.4248) and (67.4158,509.0053) ..
  (67.4158,509.0053) -- (65.6417,529.9719) -- (68.2222,533.1976) --
  (89.6727,527.5527) -- (94.5111,532.3912) -- (95.8014,535.9394) --
  (90.9629,543.0358) -- (75.6412,542.8745) -- (68.2222,552.0675) --
  (66.7707,574.4857) -- (69.6737,578.8403) -- (90.8016,580.1306) --
  (94.1886,582.7111) -- (97.2529,583.1949) -- (100.8011,590.4526) --
  (106.7685,590.2913) -- (119.9936,583.1949) -- (121.6065,583.5175) --
  (128.8641,574.9695) -- (133.8639,575.1308) -- (135.7993,577.8726) --
  (141.1216,575.2921) -- (139.0249,568.3570) -- (129.8318,557.0673) --
  (131.4446,550.7773) -- (134.6703,549.0032) -- (155.4756,548.3581) .. controls
  (155.4756,548.3581) and (162.8946,546.2614) .. (163.8623,546.1001) .. controls
  (164.8300,545.9388) and (175.6359,544.1647) .. (177.0874,544.0034) .. controls
  (178.5389,543.8422) and (190.9576,543.5196) .. (190.9576,543.5196) --
  (197.5702,542.2293) -- (203.2150,541.9068) -- (207.0858,548.6806) --
  (212.8919,551.2611) -- (219.3432,553.1965) -- (233.0522,561.7444) --
  (242.5678,563.5185) -- (248.0514,560.7768) -- (251.5996,560.4542) --
  (245.1483,570.9375) -- (245.3096,573.5180) -- (250.1480,574.9695) --
  (264.9860,574.3244) -- (267.2439,571.7439) -- (272.5662,574.1631) --
  (274.6629,574.4857) -- (274.1790,567.5506) -- (273.6952,563.3573) --
  (278.6949,553.5191) -- (277.2434,546.7452) -- (284.3398,542.7132) --
  (282.8882,537.8747) -- (293.0490,536.4232) -- (299.8228,525.9399) --
  (302.0808,521.4240) -- (339.9820,508.6827) -- (346.1107,508.8440) --
  (352.5620,502.7153) -- (364.9807,503.8443) -- (380.7863,490.1353) --
  (382.3991,488.3612) -- (410.6234,480.7810) -- (427.2354,476.9102) --
  (436.2672,468.6848) -- (436.4285,456.2661) -- (445.7828,448.2021) .. controls
  (445.7828,448.2021) and (447.8795,440.1380) .. (448.3633,439.4928) .. controls
  (448.8472,438.8477) and (462.7174,423.0421) .. (462.7174,423.0421) --
  (464.8141,410.3008) -- (467.2333,395.3016) -- (468.5236,384.0119) --
  (475.2974,379.0122) -- (486.4259,378.0445) -- (491.9094,371.5932) --
  (499.0058,370.4642) -- (495.2964,359.4971) .. controls (495.2964,359.4971) and
  (491.7482,359.4971) .. (490.7805,359.6584) .. controls (489.8128,359.8196) and
  (481.9100,361.4325) .. (481.1036,361.2712) .. controls (480.2971,361.1099) and
  (477.5554,354.1748) .. (477.5554,354.1748) .. controls (477.5554,354.1748) and
  (479.6520,350.1427) .. (479.8133,348.8525) .. controls (479.9746,347.5622) and
  (480.9423,343.0463) .. (480.9423,343.0463) -- (485.9420,338.5304) --
  (482.2325,331.9179) -- (474.3297,316.7574) -- (473.2007,302.7259) --
  (471.4266,301.1131) -- (466.5882,301.5969) -- (458.0403,290.1459) --
  (461.2659,283.2108) -- (461.5885,273.8565) -- (451.1051,256.9219) --
  (454.9759,245.7934) -- (450.9439,234.0199) -- (451.4277,220.9560) --
  (442.5572,212.2468) -- (436.7511,210.4727) -- (431.7513,200.1507) --
  (431.1062,189.1835) -- (421.1067,180.7969) -- (417.8811,167.0879) --
  (412.8814,155.6369) -- (406.2688,149.3469) -- (396.7532,138.8636) --
  (394.4952,130.4770) -- (383.3668,119.1872) -- (381.4314,113.8649) --
  (379.1735,99.9947) -- (372.0771,97.0916) -- (368.2063,100.9624) --
  (356.4327,100.8011) -- (355.6263,94.9950) -- (360.9486,88.5437) --
  (360.1422,84.3504) -- (350.7879,83.0601) -- (350.9492,77.7378) --
  (344.1753,69.1899) -- (343.2076,67.2545) -- (335.4661,71.7704) --
  (332.0792,76.7701) -- (326.5956,76.7701) -- (318.5315,85.1568) --
  (315.9510,83.2214) -- (315.1446,75.8024) -- (307.2418,70.8027) --
  (304.1774,79.3506) -- (300.3067,82.7376) -- (289.1782,83.3827) --
  (283.6947,79.6732) -- (281.9206,83.3827) -- (284.0172,87.8986) --
  (279.6626,89.1888) -- (268.8567,87.5760) -- (261.7603,96.1239) --
  (257.8896,95.6401) -- (257.4057,84.5117) -- (251.1157,76.9314) --
  (248.8578,71.4478) -- (245.9547,72.7381) -- (245.1483,80.6409) --
  (239.9873,86.6083) -- (239.0196,95.8014) -- (242.4065,101.9301) --
  (243.8580,106.2847) -- (239.3422,110.8006) -- (220.3109,109.6716) --
  (214.5048,115.1552) -- (207.5696,115.1552) -- (192.7317,123.2193) --
  (186.6030,121.1226) -- cycle;
\end{tikzpicture}
\end{image}
In essence, we are looking at the crater from directly above, and each
curve in the map above represents a fixed, constant height.
Mathematically, a contour map illustrates a \textit{function} of two
variables. We will now define a more general case of a function of $n$
variables. These are often called \textit{functions of several
  variables}.

\begin{definition}
  Let $D$ be a subset of $\R^n$.  A \dfn{function $F$ of $n$
    variables}, also called a \dfn{function $F$ of several variables}, with \dfn{domain} $D$ is a relation that assigns to
  every ordered $n$-tuple in $D$ a unique real number in $\R$.  We
  denote this by each of the following types of notation.

  \begin{align*}
    F: D &\to \R\\
    \pt{x} &\mapsto y\\
    \point{x_1,x_2,\dots,x_n} &\mapsto y
  \end{align*}
  
  The \dfn{range} of $F$ is the set of all outputs of $F$.  It is a
  subset of $\R$, not $\R^n$.
\end{definition}

\begin{example}
  Consider
  \begin{align*}
    F: \R^2 &\to \R\\
    (x,y) &\mapsto x^2+y^2.
  \end{align*}
  Find the domain and range of $F$.
  \begin{explanation}
    Here, the domain is \wordChoice{\choice{$(-\infty,\infty)$}\choice[correct]{$\R^2$}\choice{$\R^n$} \choice{All points $(x,y)$ in $\R^2$ with $x \geq 0$ and $y \geq 0$}}
    and the range is \wordChoice{\choice{$(-\infty,\infty)$}\choice[correct]{$[0,\infty)$}\choice{$\R^2$}\choice{$\R^n$}}.
  \end{explanation}
\end{example}
The relationship from the previous example can be described more
succinctly by the equation
\[
F(x,y)=x^2+y^2,
\]
which is the notation that we will use most frequently 
when describing functions.

\begin{remark}
  In this text, we will use an upper-case letter to denote a function
  of several variables.
\end{remark}

  Often, we will not specify the domain of a function in order to
  shorten its description.  Unless otherwise specified, we will take the
  domain of a given function on $\R^n$ to be the set of all ordered
  $n$-tuples in $\R^n$ for which the given expression is defined. We are familiar with this 
  concept from one-variable calculus, where we would see a function defined by a formula such as 
  $f(x) = \sqrt{x}$ and take its domain to be $[0, \infty)$.  In our example
 $F(x,y) = x^2+y^2$, we take its domain to be $\R^2$.


Let's investigate a few functions of two variables, $F:\R^2\to\R$.

\begin{question}
  Consider
  \[
  F(x,y) = \ln(9-x^2-y^2).
  \]
  What is $F(2,1)$?
  \begin{prompt}
    \[
    F(2,1) = \answer{\ln(4)}
    \]
  \end{prompt}
  \begin{question}
    What is the domain of $F$?
    
    \begin{prompt}
      Since we have not specified the domain, we take it to be the set of all vectors $\point{x,y}$ allowable as
      \wordChoice{\choice[correct]{inputs}\choice{outputs}} for $F$.
      Because of the logarithm, we need $\point{x,y}$ such that \wordChoice{\choice[correct]{$0 < 9-x^2-y^2$}\choice{ $ 0 \leq 9-x^2-y^2$} \choice{$  0 > 9-x^2-y^2$}\choice{ $ 0 \geq 9-x^2-y^2$}}
      
      The observant reader may note that this inequality
      describes the interior of a circle of radius $\answer[given]{3}$
      centered at $(0,0)$ in the $(x,y)$-plane, since we can write
      \begin{align*}
        0 & < 9-x^2-y^2 \\
        \answer{x^2+y^2} &< 9.
      \end{align*}
      While the domain may not always be easy to visualize, it is excellent practice and often insightful to try such a visualization.
      
    \end{prompt}
    \begin{question}
      What is the range of $F$?
      
      \begin{prompt}
        The range is the set of all possible
        \wordChoice{\choice{input}\choice[correct]{output}} values.  If we visualize 
        the graph of $y = \ln(x)$, we can see that the logarithm function outputs all 
        values in $(-\infty, \infty)$.  However, the input for our logarithm function is 
        not any value of $x$, but any value of $9 - x^2 - y^2$. Since 
        the $x$ and $y$ terms are squared and then subtracted from $9$, the 
        largest possible value of $9-x^2-y^2$ occurs where $x=\answer{0}$ and
        $y=\answer{0}$, in which case $F(0,0) = \answer{\ln(9)}$.  Notice that 
        we must also have $9-x^2-y^2 > \answer[given]{0}$ in order to calculate the 
        logarithm.
        
        What do these calculations mean for the range of $F$? 
        
       
        In general, the logarithm is an
        \wordChoice{\choice[correct]{increasing}\choice{decreasing}}
        function of its input, meaning that as the input gets larger,
        the output gets
        \wordChoice{\choice[correct]{larger}\choice{smaller}}.  In
        other words, the largest value of $9-x^2-y^2$ gives us the
        largest possible value of $F$.  We similarly find smaller
        values of $F$ by plugging in smaller values of $9-x^2-y^2$.
        We have determined that the values of $9-x^2-y^2$ which make
        sense for this problem are those in the interval $(0, 9]$, and
  so evaluating the logarithm on this interval gives us that the range
  $R$ is the interval $\left(\answer{-\infty},\answer{\ln(9)}\right]$.
      \end{prompt}
    \end{question}
  \end{question}
\end{question}

Consider this geometric example.
\begin{example}
  The volume of a cylinder with base radius $R$ and height $h$ is
  given by
  \[
  V=\pi R^2h.
  \]
  We can now think of the volume of a cylinder as a function of two
  variables, $R$ and $h$
  \[
  V(R,h) = \pi R^2h.
  \]
  Find the domain and the range of $V$.
  \begin{explanation}
    By requiring that the radius and height be nonnegative, we find that the domain is
    \wordChoice{
      \choice{$\R$}
      \choice{$[0,\infty)$}
      \choice[correct]{Points $(R,h)$ in $\R^2$ where $R \geq 0$ and $h \geq 0$, or in set notation $\{ (R,h) \in \R^2 : R \geq 0, h \geq 0\}$}
      }. 
The range is:
\wordChoice{
  \choice{$\R$}
  \choice[correct]{$[0,\infty)$}
  \choice{$\{ (R,h) \in \R^2 : R \geq 0, h \geq 0\}$}
  }.
The domain represents the set of all possible nonnegative radii and heights of the cylinder, and the range represents the set of all possible volumes that a cylinder could have.
  \end{explanation}
\end{example}



\section{Visualizing functions of several variables}

There are many ways to interpret a function of several variables.  Two
very common ways to do this are to consider the surface obtained by graphing 
the function or to look at what we will call the level sets of our function.

Recall that given a function $f(x)$ of a single variable, we can
consider the equation $y=f(x)$, which allows us to visualize the
function as the set of all points $(x,y)$ in the $(x,y)$-plane.  To do
this, we pick an $x$-value in the domain, and then the corresponding
$y$-coordinate is given by $f(x)$.

%Include graph?%

Given a function $F(x,y)$ of two variables, we can take the same
approach.  We'll consider the set of all points in $(x,y,z)$-space
where $z=F(x,y)$.  By choosing a point $(x,y)$ in the domain of the
function, the corresponding $z$-coordinate will be given by $F(x,y)$.
\begin{image}
  \begin{tikzpicture}
    \begin{axis}[tick label style={font=\scriptsize},axis on top,
	axis lines=center,
	view={110}{25},
	name=myplot,
	xtick=\empty,
        ytick=\empty,
        ztick=\empty,
	ymin=-.1,ymax=1.2,
	xmin=-.1,xmax=1.2,
	zmin=-.2, zmax=2.1,
	every axis x label/.style={at={(axis cs:\pgfkeysvalueof{/pgfplots/xmax},0,0)},xshift=-1pt,yshift=-4pt},
	xlabel={\scriptsize $x$},
	every axis y label/.style={at={(axis cs:0,\pgfkeysvalueof{/pgfplots/ymax},0)},xshift=5pt,yshift=-3pt},
	ylabel={\scriptsize $y$},
	every axis z label/.style={at={(axis cs:0,0,\pgfkeysvalueof{/pgfplots/zmax})},xshift=0pt,yshift=4pt},
	zlabel={\scriptsize $z$},
        colormap/cool,
      ]
      \addplot3[gray,domain=0:2,samples y=0,dashed] ({1*cos(45)},{1*sin(45)},x); %% line for z
      \addplot3[gray,domain=0:cos(45),samples y=0,dashed] ({x},{1*sin(45)},0); %% line for x
      \addplot3[gray,domain=0:cos(45),samples y=0,dashed] ({sin(45)},{x},0); %% line for y
      \filldraw [black] (axis cs:{1*cos(45)},{1*sin(45)},1.95) circle (2.5pt);        
      \node[right] at (axis cs:{1*cos(45)},{1*sin(45)},2) {$(x,y,F(x,y))$};

      \filldraw [black!50!white] (axis cs:{1*cos(45)},{1*sin(45)},0) circle (2.5pt);        
      \node[below right,black!50!white] at (axis cs:{1*cos(45)},{1*sin(45)},0) {$(x,y)$};
    \end{axis}
  \end{tikzpicture}
\end{image}

Thus, one way of visualizing the function $F(x,y)$ is to consider the
equation $z=F(x,y)$ and consider the set of all of the points in the $(x,y,z)$-space
that satisfy this criteria.  We can then interpret that the function assigns a
height to each point $(x,y)$ in its domain.  Be very careful with this way 
of visualizing the function, however!  The ``height'' can sometimes be 
negative, while we tend to almost always visualize a positive height.  

\begin{remark}
We do not always interpret the output $F(x,y)$ as a height.  For instance, we might 
want to talk about a density function for a region in $\R^2$, and define a function $\rho(x,y)$
by the density at each point $(x,y)$ in the region.  We can still graph $z=\rho(x,y)$, but the 
$z$-values now should be interpreted as densities. Be careful to keep the meaning of the 
function in mind.
\end{remark}

To make a sketch of a surface, we can specify many locations in the
$(x,y)$-plane (by picking many different values for $x$ and $y$), and
plot the corresponding $z$-values.  While this is tedious to do by hand,
computers can do it very easily.  For example, if we consider the
function $F(x,y) = 2-4x^3+y^2$, we can evaluate the function at many
different points $(x,y)$ and plot the results.  For instance, at the point
$(x,y)=(1,2)$, we have $F(1,2) = \answer[given]{2}$.  Using software
to graph both the surface and this point gives the following.

\begin{image}
  \begin{tikzpicture}
    \begin{axis}[tick label style={font=\scriptsize},axis on top,
	axis lines=center,
	view={110}{25},
	name=myplot,
	xtick=\empty,
        ytick=\empty,
        ztick=\empty,
	ymin=-.25,ymax=2.2,
	xmin=-.25,xmax=2.2,
	zmin=-.5, zmax=5.1,
	every axis x label/.style={at={(axis cs:\pgfkeysvalueof{/pgfplots/xmax},0,0)},xshift=-1pt,yshift=-4pt},
	xlabel={\scriptsize $x$},
	every axis y label/.style={at={(axis cs:0,\pgfkeysvalueof{/pgfplots/ymax},0)},xshift=5pt,yshift=-3pt},
	ylabel={\scriptsize $y$},
	every axis z label/.style={at={(axis cs:0,0,\pgfkeysvalueof{/pgfplots/zmax})},xshift=0pt,yshift=4pt},
	zlabel={\scriptsize $z$},
        colormap/cool,
      ]
      \addplot3[gray,domain=0:2,samples y=0,dashed] ({1*cos(45)},{1*sin(45)},x); %% line for z
      \addplot3[gray,domain=0:cos(45),samples y=0,dashed] ({x},{1*sin(45)},0); %% line for x
      \addplot3[gray,domain=0:cos(45),samples y=0,dashed] ({sin(45)},{x},0); %% line for y
\addplot3[domain=-.3:2.2,y domain=-.3:2.5,mesh,samples y=25,very thin,z buffer=sort,  samples=25,] (x,y,{2-4*x^3+y^2});            
      \filldraw [black] (axis cs:{1*cos(45)},{1*sin(45)},1.95) circle (2.5pt);        
      \node[right] at (axis cs:{1*cos(45)},{1*sin(45)},1.9) {$(1,2,2)$};

      \filldraw [black!50!white] (axis cs:{1*cos(45)},{1*sin(45)},0) circle (2.5pt);        
      \node[below right,black!50!white] at (axis cs:{1*cos(45)},{1*sin(45)},0) {$(1,2)$};
    \end{axis}
  \end{tikzpicture}
\end{image}


\subsection{Generating curves on surfaces}
Recall that we described curves in $\R^n$ by giving vector-valued 
functions $\pt{p}(t)$, where the coordinates of any point on 
the curve can be determined from a single parameter.  We would 
now like to consider vector-valued functions alongside functions 
of several variables.  As usual, we will work with two variables so 
that we can better visualize our examples, but our results will also 
extend to the case of $n$ variables.

If we have a function $F(x,y) : D \to \R$ and a vector-valued 
function $\vec{p}(t) = \vector{x(t), y(t)}$ so that for any value of $t$, 
$\point{x(t), y(t)}$ is in the domain $D$, we can evaluate $F$ at 
each point along $\pt{p}$ and produce another curve $F\left (\pt{p} \right )$
on the surface.

\begin{example}
Consider again $F(x,y) = 2-x^3+y^2$.  Also consider the curve defined by $y=2x$ in the $(x,y)$-plane.  Note that the domain of $F(x,y) = 2-x^3+y^2$ is all of $\R^2$, so each point on $\vec{p}(t) = \vector{t, 2t}$ is in the domain of $F$.

Fill out the table below:
\[
\begin{array}{c|c||c|c||c}
x & y =2x & (x,y) & z=2-x^3+y^2 & (x,y,z) \\
\hline
-0.5 & -1 & \left(-0.5,-1\right) &3.125 & \left(-0.5, -1,3.125\right)\\
0 & \answer[given]{0} & \left(\answer[given]{0},\answer[given]{0}\right) & \answer[given]{2}          & \left(\answer[given]{0},\answer[given]{0},    \answer[given]{2}\right) \\
0.5 & \answer[given]{1} & \left(\answer[given]{0.5}, \answer[given]{1}\right) &  \answer[given]{2.875}  & \left(\answer[given]{0.5}, \answer[given]{1}, \answer[given]{2.875}\right)\\
1 & \answer[given]{2} & \left(\answer[given]{1},\answer[given]{2}\right) & \answer[given]{5}          & \left(\answer[given]{1}, \answer[given]{2},   \answer[given]{5}\right)
\end{array}
\]
The points $(x,y)$ in our table lie on the curve $\vec{p}(t) = \vector{t, 2t}$ in the $(x,y)$-plane.  The points $(x,y,z) = (x,y,F(x,y))$ in our table lie on the surface $F(x,y)$.  If we would evaluate $F\left (\pt{p}(t) \right )$ for every point on $\pt{p}(t)$ we would see the curve on the surface corresponding to the curve in the plane, as pictured below.

\begin{image}
  \begin{tikzpicture}
    \begin{axis}[tick label style={font=\scriptsize},axis on top,
	axis lines=center,
	view={110}{25},
	name=myplot,
	xtick=\empty,
        ytick=\empty,
        ztick=\empty,
	ymin=-.8,ymax=2.2,
	xmin=-.8,xmax=2.2,
	zmin=-.5, zmax=5.1,
	every axis x label/.style={at={(axis cs:\pgfkeysvalueof{/pgfplots/xmax},0,0)},xshift=-1pt,yshift=-4pt},
	xlabel={\scriptsize $x$},
	every axis y label/.style={at={(axis cs:0,\pgfkeysvalueof{/pgfplots/ymax},0)},xshift=5pt,yshift=-3pt},
	ylabel={\scriptsize $y$},
	every axis z label/.style={at={(axis cs:0,0,\pgfkeysvalueof{/pgfplots/zmax})},xshift=0pt,yshift=4pt},
	zlabel={\scriptsize $z$},
        colormap/cool,
      ]
%      \addplot3[gray,domain=0:2,samples y=0,dashed] ({1*cos(45)},{1*sin(45)},x); %% line for z
%      \addplot3[gray,domain=0:cos(45),samples y=0,dashed] ({x},{1*sin(45)},0); %% line for x
%      \addplot3[gray,domain=0:cos(45),samples y=0,dashed] ({sin(45)},{x},0); %% line for y
\addplot3[domain=-.8:2.2,y domain=-.8:2.5,mesh,samples y=25,very thin,z buffer=sort,  samples=25,] (x,y,{2-4*x^3+y^2});   
\addplot3 [very thick,penColor, smooth,domain=-.39:3,samples=20,samples y=0] ({x},{2*x},{2-4*x^3+4*x^2});
\addplot3 [very thick,penColor, dashed,domain=-.5:3,samples=20,samples y=0] ({x},{2*x},{0});
     

%points they plotted in the xy plane and point on curve%
	%\node[below,black!50!white] at (axis cs:{-.5*cos(45)},{-1*sin(45)},0) {$(-.5,-1)$};
		      \filldraw [black] (axis cs:{-.5*cos(45)},{-1*sin(45)},0) circle (2.5pt); 
		      \filldraw [black] (axis cs:{-.5*cos(45)},{-1*sin(45)},2.6768) circle (2.5pt);  
	       	      \addplot3[black,thick, domain=0:2.6768,samples y=0,dashed] ({-.5*cos(45)},{-1*sin(45)},x); %% line for z 
      	%\node[below,black!50!white] at (axis cs:{0*cos(45)},{0*sin(45)},0) {$(0,0)$};
		  \filldraw [black] (axis cs:{0*cos(45)},{0*sin(45)},0) circle (2.5pt); 
		  \filldraw [black] (axis cs:{0*cos(45)},{0*sin(45)},2) circle (2.5pt);  
		  \addplot3[black,thick, domain=0:2,samples y=0,dashed] ({0*cos(45)},{0*sin(45)},x); %% line for z 
      	%\node[below,black!50!white] at (axis cs:{.5*cos(45)},{1*sin(45)},0) {$(.5,1)$};
		  \filldraw [black] (axis cs:{.5*cos(45)},{1*sin(45)},0) circle (2.5pt); 
		  \filldraw [black] (axis cs:{.5*cos(45)},{1*sin(45)},2.3232) circle (2.5pt); 
		  \addplot3[black,thick, domain=0:2.3232,samples y=0,dashed] ({.5*cos(45)},{1*sin(45)},x); %% line for z  
      	%\node[below,black!50!white] at (axis cs:{1*cos(45)},{2*sin(45)},0) {$(1,2)$};
	  	\filldraw [black] (axis cs:{1*cos(45)},{2*sin(45)},0) circle (2.5pt); 
		\filldraw [black] (axis cs:{1*cos(45)},{2*sin(45)},2.5858) circle (2.5pt);  
		\addplot3[black,thick, domain=0:2.5858,samples y=0,dashed] ({1*cos(45)},{2*sin(45)},x); %% line for z 
      	%\node[below,black!50!white] at (axis cs:{1.5*cos(45)},{3*sin(45)},0) {$(1.5,3)$};
	  	\filldraw [black] (axis cs:{1.5*cos(45)},{3*sin(45)},0) circle (2.5pt); 
		\filldraw [black] (axis cs:{1.5*cos(45)},{3*sin(45)},1.727) circle (2.5pt);
		\addplot3[black,thick, domain=0:1.727,samples y=0,dashed] ({1.5*cos(45)},{3*sin(45)},x); %% line for z   		

    \end{axis}
  \end{tikzpicture}
  
  %%%%%SHOULD WE LABEL THESE POINTS?%%%%%%%%%
\end{image}

\end{example} 

We can think of the function as ``lifting'' a curve onto the surface 
$z=F(x,y)$ in $(x,y,z)$-space.  Of course, the curve must be in the 
domain of the function $F$, and we should always be cautious 
when using the notion of ``height'' for our $z$-coordinate.

So far, we have focused mainly on the curve $\pt{p}(t)$ lying in the 
domain of a function $F$.  Let's now focus on the curve $z=F \left ( \pt{p} (t) \right )$ 
on the surface.  Since the curve is in $\R^n$, we must use a 
parametric equation to describe it.  Fortunately, with our background 
on vector-valued functions, finding such a description should be straightforward.

\begin{example}
  Let $F(x,y) = 2-x^3+y^2$ as before.  Give a parametric
  description of the the curve $\pt{r}(t)$ that lies on this surface above the line
  $y=2x$ in the $(x,y)$-plane.
  \begin{explanation}
   A parameterization $\vec{p}(t)$ of $y = 2x$ in the $(x,y)$-plane is 
   
   \[
   \vec{p}(t) = \vector{t, \answer[given]{2t}}.
   \]
    
     We can now use the equation of the surface $z=F(x,y)$ to give $z$ in terms of $t$.  
     Since $z=2-x^3+y^2$, setting $x=t$ and expressing $y$ in terms of $t$ gives 
     $z(t) = \answer[given]{2+4t^2-t^3}$. Thus, a parameterization of the curve is
    \[
    \pt{r}(t) =\point{\answer[given]{t} , \answer[given]{2t}, \answer[given]{2+4t^2-t^3}}.
    \]
  \end{explanation}
\end{example}

The idea of looking at a curve in the domain of $F$ and the corresponding 
curve on the surface can be helpful when thinking about many topics 
that follow.  We may see these ideas again when we discuss limits of 
functions of several variables, derivatives and differentiability, the chain 
rule, tangent planes, as well as constrained optimization.  Any time we work 
with these curves on surfaces, remember to think carefully about whether 
we are working in the domain of $F$ or on the surface itself.

\end{document}
