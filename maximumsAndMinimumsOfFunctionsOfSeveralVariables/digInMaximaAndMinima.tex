\documentclass{ximera}

%\usepackage{todonotes}
%\usepackage{mathtools} %% Required for wide table Curl and Greens
%\usepackage{cuted} %% Required for wide table Curl and Greens
\newcommand{\todo}{}

\usepackage{esint} % for \oiint
\ifxake%%https://math.meta.stackexchange.com/questions/9973/how-do-you-render-a-closed-surface-double-integral
\renewcommand{\oiint}{{\large\bigcirc}\kern-1.56em\iint}
\fi


\graphicspath{
  {./}
  {ximeraTutorial/}
  {basicPhilosophy/}
  {functionsOfSeveralVariables/}
  {normalVectors/}
  {lagrangeMultipliers/}
  {vectorFields/}
  {greensTheorem/}
  {shapeOfThingsToCome/}
  {dotProducts/}
  {partialDerivativesAndTheGradientVector/}
  {../productAndQuotientRules/exercises/}
  {../motionAndPathsInSpace/exercises/}
  {../normalVectors/exercisesParametricPlots/}
  {../continuityOfFunctionsOfSeveralVariables/exercises/}
  {../partialDerivativesAndTheGradientVector/exercises/}
  {../directionalDerivativeAndChainRule/exercises/}
  {../commonCoordinates/exercisesCylindricalCoordinates/}
  {../commonCoordinates/exercisesSphericalCoordinates/}
  {../greensTheorem/exercisesCurlAndLineIntegrals/}
  {../greensTheorem/exercisesDivergenceAndLineIntegrals/}
  {../shapeOfThingsToCome/exercisesDivergenceTheorem/}
  {../greensTheorem/}
  {../shapeOfThingsToCome/}
  {../separableDifferentialEquations/exercises/}
  {vectorFields/}
}

\newcommand{\mooculus}{\textsf{\textbf{MOOC}\textnormal{\textsf{ULUS}}}}

\usepackage{tkz-euclide}\usepackage{tikz}
\usepackage{tikz-cd}
\usetikzlibrary{arrows}
\tikzset{>=stealth,commutative diagrams/.cd,
  arrow style=tikz,diagrams={>=stealth}} %% cool arrow head
\tikzset{shorten <>/.style={ shorten >=#1, shorten <=#1 } } %% allows shorter vectors

\usetikzlibrary{backgrounds} %% for boxes around graphs
\usetikzlibrary{shapes,positioning}  %% Clouds and stars
\usetikzlibrary{matrix} %% for matrix
\usepgfplotslibrary{polar} %% for polar plots
\usepgfplotslibrary{fillbetween} %% to shade area between curves in TikZ
%\usetkzobj{all} obsolete

\usepackage[makeroom]{cancel} %% for strike outs
%\usepackage{mathtools} %% for pretty underbrace % Breaks Ximera
%\usepackage{multicol}
\usepackage{pgffor} %% required for integral for loops



%% http://tex.stackexchange.com/questions/66490/drawing-a-tikz-arc-specifying-the-center
%% Draws beach ball
\tikzset{pics/carc/.style args={#1:#2:#3}{code={\draw[pic actions] (#1:#3) arc(#1:#2:#3);}}}



\usepackage{array}
\setlength{\extrarowheight}{+.1cm}
\newdimen\digitwidth
\settowidth\digitwidth{9}
\def\divrule#1#2{
\noalign{\moveright#1\digitwidth
\vbox{\hrule width#2\digitwidth}}}





\newcommand{\RR}{\mathbb R}
\newcommand{\R}{\mathbb R}
\newcommand{\N}{\mathbb N}
\newcommand{\Z}{\mathbb Z}

\newcommand{\sagemath}{\textsf{SageMath}}


%\renewcommand{\d}{\,d\!}
\renewcommand{\d}{\mathop{}\!d}
\newcommand{\dd}[2][]{\frac{\d #1}{\d #2}}
\newcommand{\pp}[2][]{\frac{\partial #1}{\partial #2}}
\renewcommand{\l}{\ell}
\newcommand{\ddx}{\frac{d}{\d x}}

\newcommand{\zeroOverZero}{\ensuremath{\boldsymbol{\tfrac{0}{0}}}}
\newcommand{\inftyOverInfty}{\ensuremath{\boldsymbol{\tfrac{\infty}{\infty}}}}
\newcommand{\zeroOverInfty}{\ensuremath{\boldsymbol{\tfrac{0}{\infty}}}}
\newcommand{\zeroTimesInfty}{\ensuremath{\small\boldsymbol{0\cdot \infty}}}
\newcommand{\inftyMinusInfty}{\ensuremath{\small\boldsymbol{\infty - \infty}}}
\newcommand{\oneToInfty}{\ensuremath{\boldsymbol{1^\infty}}}
\newcommand{\zeroToZero}{\ensuremath{\boldsymbol{0^0}}}
\newcommand{\inftyToZero}{\ensuremath{\boldsymbol{\infty^0}}}



\newcommand{\numOverZero}{\ensuremath{\boldsymbol{\tfrac{\#}{0}}}}
\newcommand{\dfn}{\textbf}
%\newcommand{\unit}{\,\mathrm}
\newcommand{\unit}{\mathop{}\!\mathrm}
\newcommand{\eval}[1]{\bigg[ #1 \bigg]}
\newcommand{\seq}[1]{\left( #1 \right)}
\renewcommand{\epsilon}{\varepsilon}
\renewcommand{\phi}{\varphi}


\renewcommand{\iff}{\Leftrightarrow}

\DeclareMathOperator{\arccot}{arccot}
\DeclareMathOperator{\arcsec}{arcsec}
\DeclareMathOperator{\arccsc}{arccsc}
\DeclareMathOperator{\si}{Si}
\DeclareMathOperator{\scal}{scal}
\DeclareMathOperator{\sign}{sign}


%% \newcommand{\tightoverset}[2]{% for arrow vec
%%   \mathop{#2}\limits^{\vbox to -.5ex{\kern-0.75ex\hbox{$#1$}\vss}}}
\newcommand{\arrowvec}[1]{{\overset{\rightharpoonup}{#1}}}
%\renewcommand{\vec}[1]{\arrowvec{\mathbf{#1}}}
\renewcommand{\vec}[1]{{\overset{\boldsymbol{\rightharpoonup}}{\mathbf{#1}}}\hspace{0in}}

\newcommand{\point}[1]{\left(#1\right)} %this allows \vector{ to be changed to \vector{ with a quick find and replace
\newcommand{\pt}[1]{\mathbf{#1}} %this allows \vec{ to be changed to \vec{ with a quick find and replace
\newcommand{\Lim}[2]{\lim_{\point{#1} \to \point{#2}}} %Bart, I changed this to point since I want to use it.  It runs through both of the exercise and exerciseE files in limits section, which is why it was in each document to start with.

\DeclareMathOperator{\proj}{\mathbf{proj}}
\newcommand{\veci}{{\boldsymbol{\hat{\imath}}}}
\newcommand{\vecj}{{\boldsymbol{\hat{\jmath}}}}
\newcommand{\veck}{{\boldsymbol{\hat{k}}}}
\newcommand{\vecl}{\vec{\boldsymbol{\l}}}
\newcommand{\uvec}[1]{\mathbf{\hat{#1}}}
\newcommand{\utan}{\mathbf{\hat{t}}}
\newcommand{\unormal}{\mathbf{\hat{n}}}
\newcommand{\ubinormal}{\mathbf{\hat{b}}}

\newcommand{\dotp}{\bullet}
\newcommand{\cross}{\boldsymbol\times}
\newcommand{\grad}{\boldsymbol\nabla}
\newcommand{\divergence}{\grad\dotp}
\newcommand{\curl}{\grad\cross}
%\DeclareMathOperator{\divergence}{divergence}
%\DeclareMathOperator{\curl}[1]{\grad\cross #1}
\newcommand{\lto}{\mathop{\longrightarrow\,}\limits}

\renewcommand{\bar}{\overline}

\colorlet{textColor}{black}
\colorlet{background}{white}
\colorlet{penColor}{blue!50!black} % Color of a curve in a plot
\colorlet{penColor2}{red!50!black}% Color of a curve in a plot
\colorlet{penColor3}{red!50!blue} % Color of a curve in a plot
\colorlet{penColor4}{green!50!black} % Color of a curve in a plot
\colorlet{penColor5}{orange!80!black} % Color of a curve in a plot
\colorlet{penColor6}{yellow!70!black} % Color of a curve in a plot
\colorlet{fill1}{penColor!20} % Color of fill in a plot
\colorlet{fill2}{penColor2!20} % Color of fill in a plot
\colorlet{fillp}{fill1} % Color of positive area
\colorlet{filln}{penColor2!20} % Color of negative area
\colorlet{fill3}{penColor3!20} % Fill
\colorlet{fill4}{penColor4!20} % Fill
\colorlet{fill5}{penColor5!20} % Fill
\colorlet{gridColor}{gray!50} % Color of grid in a plot

\newcommand{\surfaceColor}{violet}
\newcommand{\surfaceColorTwo}{redyellow}
\newcommand{\sliceColor}{greenyellow}




\pgfmathdeclarefunction{gauss}{2}{% gives gaussian
  \pgfmathparse{1/(#2*sqrt(2*pi))*exp(-((x-#1)^2)/(2*#2^2))}%
}


%%%%%%%%%%%%%
%% Vectors
%%%%%%%%%%%%%

%% Simple horiz vectors
\renewcommand{\vector}[1]{\left\langle #1\right\rangle}


%% %% Complex Horiz Vectors with angle brackets
%% \makeatletter
%% \renewcommand{\vector}[2][ , ]{\left\langle%
%%   \def\nextitem{\def\nextitem{#1}}%
%%   \@for \el:=#2\do{\nextitem\el}\right\rangle%
%% }
%% \makeatother

%% %% Vertical Vectors
%% \def\vector#1{\begin{bmatrix}\vecListA#1,,\end{bmatrix}}
%% \def\vecListA#1,{\if,#1,\else #1\cr \expandafter \vecListA \fi}

%%%%%%%%%%%%%
%% End of vectors
%%%%%%%%%%%%%

%\newcommand{\fullwidth}{}
%\newcommand{\normalwidth}{}



%% makes a snazzy t-chart for evaluating functions
%\newenvironment{tchart}{\rowcolors{2}{}{background!90!textColor}\array}{\endarray}

%%This is to help with formatting on future title pages.
\newenvironment{sectionOutcomes}{}{}



%% Flowchart stuff
%\tikzstyle{startstop} = [rectangle, rounded corners, minimum width=3cm, minimum height=1cm,text centered, draw=black]
%\tikzstyle{question} = [rectangle, minimum width=3cm, minimum height=1cm, text centered, draw=black]
%\tikzstyle{decision} = [trapezium, trapezium left angle=70, trapezium right angle=110, minimum width=3cm, minimum height=1cm, text centered, draw=black]
%\tikzstyle{question} = [rectangle, rounded corners, minimum width=3cm, minimum height=1cm,text centered, draw=black]
%\tikzstyle{process} = [rectangle, minimum width=3cm, minimum height=1cm, text centered, draw=black]
%\tikzstyle{decision} = [trapezium, trapezium left angle=70, trapezium right angle=110, minimum width=3cm, minimum height=1cm, text centered, draw=black]


\outcome{Find the critical points of a function of two variables.}
\outcome{Use the second derivative test to classify local extrema.}
\outcome{Find local extrema of functions of two variables.}



\title[Dig-In:]{Maxima and minima}

\begin{document}
\begin{abstract}
  We see how to find extrema of functions of several variables.
\end{abstract}
\maketitle


Given a function $z=F(x,y)$, we are often interested in points where
$z$ takes on the largest or smallest values. For instance, if $z$
represents a cost function, we would likely want to know what $(x,y)$
values minimize the cost. If $z$ represents the ratio of a volume to
surface area, we would likely want to know where $z$ is greatest. This
leads to the following definition that we will state rather generally,
but use mostly in the case of a function $F:\R^2\to\R$.

\begin{definition}
Let $F:\R^n\to\R$ be defined on a set $S\subset\R^n$ containing the
point $\vec{c}$ in $\R^n$.
\index{maximum!local}\index{minimum!local}\index{extrema!absolute}\index{maximum!absolute}\index{minimum!absolute}
\begin{itemize}
\item If there is an open ball $B$ containing $\vec{c}$ such that
  $F(\vec{c}) \geq F(\vec{x})$ for all $\vec{x}$ in $B$, then $F$ has a
  \dfn{local maximum} at $\vec{c}$; if $F(\vec{c}) \leq F(\vec{x})$ for all
  $\vec{x}$ in $B$, then $F$ has a \dfn{local minimum} at $\vec{c}$.
\item If $F(\vec{c})\geq F(\vec{x})$ for all $\vec{x}$ in $S$, then $F$ has
  an \dfn{absolute maximum} at $\vec{c}$; if $F(\vec{c})\leq F(\vec{x})$ for
  all $\vec{x}$ in $S$, then $F$ has an \dfn{absolute minimum} at
  $\vec{c}$.
\item If $F$ has a local maximum or minimum at $\vec{c}$, then $F$ has a
  \dfn{local extremum} at $\vec{c}$; if $F$ has an absolute maximum or
  minimum at $\vec{c}$, then $F$ has a \dfn{absolute extremum} at $\vec{c}$.
\end{itemize}
\end{definition}


\section{Critical points}



If $F$ has a local maximum at $\vec{c}$, it means the
gradient will point ``nowhere'' since the gradient points in the
initial direction of greatest increase. This means it is pointing in a
``direction'' whose components are zero or "direction" is undefined. In an
entirely similar way, the gradient will be a vector whose components
are either zero or undefined at local minimums as well. 

\begin{definition}
  Let $z = F(x,y)$ be continuous on an open set $S$. A
  \dfn{critical point} $\vec{c}=\vector{c_1,c_2}$ of $F$ is a point in $S$ such
  that
  \begin{itemize}
  \item $\grad F(\vec{c}) = \vec{0}$ or
  \item $\grad F(\vec{c})$ is undefined.
  \end{itemize}
\end{definition}

\begin{theorem}
Let $F:\R^n\to\R$ be defined on an open set $S$ containing
$\vec{c}$. If $F$ has a local extremum at $\vec{c}$, then $\vec{c}$ is a
critical point of $F$.
\end{theorem}

Therefore, to find local extrema, we find the critical points of $F$
and determine which correspond to local maxima, local minima, or
neither. We'll use examples to demonstrate this process.

\begin{example}
  Let $F(x,y) = x^2+y^2-xy-x-2$. Find the local extrema of $F$.
  \begin{explanation}
    We start by computing $\grad F$:
    \[
    \grad F(x,y) = \vector{\answer[given]{2x-y-1},\answer[given]{2y-x}}
    \]
    Each component of $\grad F$ is never undefined. A critical point
    occurs when $\grad F(x,y) = \vec{0}$, leading us to solve the
    following system of linear equations:
    \[
    2x-y-1 =\answer[given]{0}\quad \text{and}\quad -x+2y = \answer[given]{0}.
    \]
    This solution to this system is $x=\answer[given]{2/3}$,
    $y=\answer[given]{1/3}$. So the critical point is
    $\vec{c}=\vector{\answer[given]{2/3},\answer[given]{1/3}}$. When
    possible, it is good to confirm your answer with a graph:
    \begin{image}
      \begin{tikzpicture}
        \begin{axis}%
          [tick label style={font=\scriptsize},axis on top,
	    axis lines=center,
	    view={135}{25},
	    name=myplot,
	    %xtick=\empty,
	    %ytick={5},
	    %ztick={.7,-.7},
	    minor xtick=1,
	    minor ytick=1,
	    ymin=-1.5,ymax=2.5,
	    xmin=-1.5,xmax=2.5,
	    zmin=-2.5, zmax=5.5,
	    every axis x label/.style={at={(axis cs:\pgfkeysvalueof{/pgfplots/xmax},0,0)},xshift=-5pt,yshift=-1pt},
	    xlabel={\scriptsize $x$},
	    every axis y label/.style={at={(axis cs:0,\pgfkeysvalueof{/pgfplots/ymax},0)},xshift=4pt,yshift=-4pt},
	    ylabel={\scriptsize $y$},
	    every axis z label/.style={at={(axis cs:0,0,\pgfkeysvalueof{/pgfplots/zmax})},xshift=0pt,yshift=4pt},
	    zlabel={\scriptsize $z$},colormap/cool
	  ]
          
          \addplot3[domain=-1.:3,,y domain=-1.:3,mesh,samples y=10,very thin,z buffer=sort] {x^2+y^2-x*y-x-2};
          
          \addplot3 [ultra thick,penColor, smooth,domain=-1.:3,samples=20,samples y=0,opacity=.3] ({x},{3},{x^2+3^2-x*3-x-2});

          \addplot3 [ultra thick,penColor, smooth,domain=-1.:3,samples=20,samples y=0,opacity=.3] ({x},{-1},{x^2+(-1)^2-x*(-1)-x-2});

          \addplot3 [ultra thick,penColor, smooth,domain=-1.:3,samples=20,samples y=0,opacity=.3] ({-1},{x},{x^2+(-1)^2-x*(-1)-(-1)-2});

          \addplot3 [ultra thick,penColor, smooth,domain=-1.:3,samples=20,samples y=0,opacity=.3] ({3},{x},{x^2+3^2-x*3-3-2});

          \filldraw [black,opacity=.7] (axis cs:.666,.333,-2.333) circle (3pt);
          \node[right] at (axis cs:2/3,1/3,-7/3) {$(2/3,1/3,-7/3)$};
        \end{axis}
      \end{tikzpicture}
    \end{image}
    The graph above shows $F$ along with this critical point. It is
    clear from the graph that this is a local minimum.
  \end{explanation}
\end{example}

\begin{example}
  Let $F(x,y) = -\sqrt{x^2+y^2}+2$. Find the local extrema of $F$.
  \begin{explanation}
    We start by computing $\grad F$:
    \[
    \grad F(x,y) = \vector{\frac{-x}{\sqrt{x^2+y^2}},\frac{-y}{\sqrt{x^2+y^2}}}
    \]
    Here the only critical point is at $\vector{0,0}$ because  $\grad F$ is undefined at $\vector{0,0}$, and because  $\grad
    F\ne  \vec{0} $ at all other points. 
    \begin{image}
      \begin{tikzpicture}
        \begin{axis}%
          [tick label style={font=\scriptsize},axis on top,
	    axis lines=center,
	    view={135}{25},
	    name=myplot,
	    %xtick=\empty,
	    %ytick={5},
	    %ztick={.7,-.7},
	    minor xtick=1,
	    minor ytick=1,
	    ymin=-2.5,ymax=2.5,
	    xmin=-2.5,xmax=2.5,
	    zmin=-0.5, zmax=2.5,
	    every axis x label/.style={at={(axis cs:\pgfkeysvalueof{/pgfplots/xmax},0,0)},xshift=-5pt,yshift=-1pt},
	    xlabel={\scriptsize $x$},
	    every axis y label/.style={at={(axis cs:0,\pgfkeysvalueof{/pgfplots/ymax},0)},xshift=4pt,yshift=-4pt},
	    ylabel={\scriptsize $y$},
	    every axis z label/.style={at={(axis cs:0,0,\pgfkeysvalueof{/pgfplots/zmax})},xshift=0pt,yshift=4pt},
	    zlabel={\scriptsize $z$},colormap/cool
	  ]
          
          \addplot3[domain=-2:2,,y domain=-2:2,mesh,samples=25,samples y=25,very thin,z buffer=sort] {-sqrt(x^2+y^2)+2};
          \filldraw [black,] (axis cs:0,0,2) circle (3pt);
        \end{axis}
      \end{tikzpicture}
    \end{image}
    The graph of $F$ is shown above along with the point
    $(0,0,2)$. It is clear from the graph that this point is the absolute maximum
    of $F$.
  \end{explanation}
\end{example}

In each of the previous two examples, we found a critical point of $F$
and then determined whether or not it was a local (or absolute)
maximum or minimum by graphing. It would be nice to be able to
determine whether a critical point corresponded to a max or a min
without a graph. Before we develop such a test, we do one more example
that sheds more light on the issues our test needs to consider.


\begin{example}
  Let $F(x,y) = x^3-3x-y^2+4y$. Find the local extrema of $F$.
  \begin{explanation}
    Once again we start by computing $\grad F$:
    \[
    \grad F(x,y) = \vector{\answer[given]{3x^2-3},\answer[given]{-2y+4}}
    \]
    Each component is always defined. Setting $\grad F(x,y) = \vec{0}$
    and solving for $x$ and $y$, we find
    \begin{align*}
      x &=\pm \answer[given]{1}\\
      y &= \answer[given]{2}.
    \end{align*}
    We have two critical points: $\vector{-1,2}$ and $\vector{1,2}$. In order to determine
    if they correspond to a local maximum or minimum or neither, we consider the
    graph of $F$ below:
    \begin{image}
      \begin{tikzpicture}
        \begin{axis}%
          [tick label style={font=\scriptsize},axis on top,
	    axis lines=center,
	    view={165}{35},
	    name=myplot,
	    %xtick=\empty,
	    %ytick={5},
	    %ztick={.7,-.7},
	    minor xtick=1,
	    minor ytick=1,
	    ymin=-.5,ymax=3.5,
	    xmin=-1.5,xmax=2,
	    zmin=-2, zmax=6,
	    every axis x label/.style={at={(axis cs:\pgfkeysvalueof{/pgfplots/xmax},0,0)},xshift=-5pt,yshift=-1pt},
	    xlabel={\scriptsize $x$},
	    every axis y label/.style={at={(axis cs:0,\pgfkeysvalueof{/pgfplots/ymax},0)},xshift=4pt,yshift=-4pt},
	    ylabel={\scriptsize $y$},
	    every axis z label/.style={at={(axis cs:0,0,\pgfkeysvalueof{/pgfplots/zmax})},xshift=0pt,yshift=4pt},
	    zlabel={\scriptsize $z$},colormap/cool
	  ]
          \addplot3[domain=-1.5:2,,y domain=-0:3.5,mesh,samples=25,samples y=25,very thin,z buffer=sort] {x^3-3*x-y^2+4*y};

          \filldraw [black,] (axis cs:1,2,2) circle (3pt);
          \filldraw [black,] (axis cs:-1,2,6) circle (3pt);
        \end{axis}
      \end{tikzpicture}
    \end{image}

    %% in Figure \ref{fig:multi_extreme3}.
    %% \mfigure[scale=1.1]{.7}{The surface in Example
    %%   \ref{ex_multi_extreme3} with both critical points
    %%   marked.}{fig:multi_extreme3}{figures/figmulti_extreme3}

    The critical point $\vector{-1,2}$ clearly corresponds to a local
    maximum. However, the critical point at $\vector{1,2}$ is neither
    a maximum nor a minimum, displaying a different, interesting
    characteristic.

    If one walks parallel to the $y$-axis towards this critical point,
    then this point becomes a local maximum along this path. But if
    one walks towards this point parallel to the $x$-axis, this point
    becomes a local minimum along this path. A point that seems to act
    in some dircetion as a max and in another as a min is a \textit{saddle point}. A formal
    definition follows.
  \end{explanation}
\end{example}


\begin{definition}
  Let $F:\R^n\to\R$ and $\vec{c}$ be in the domain of $F$ where $\grad
  F(\vec{c})=\vec{0}$. The function $F$ has a  \dfn{saddle point} at $\vec{c}$ 
   if, for every open ball $B$ containing
  $\vec{c}$, there are points $\vec{a}$ and $\vec{b}$ in $B$ such
  that $F(\vec{c})>F(\vec{a})$ and $F(\vec{c})<F(\vec{b})$.
\end{definition}

The most obvious example of a saddle point is a the point determined
by $\vector{0,0}$ on a hyperbolic paraboloid of the form $z = \pm x^2 \mp y^2$. 
\begin{image}
  \begin{tikzpicture}
    \begin{axis}%
      [tick label style={font=\scriptsize},axis on top,
	axis lines=center,
	view={165}{35},
	name=myplot,
	%xtick=\empty,
	%ytick={5},
	%ztick={.7,-.7},
	minor xtick=1,
	minor ytick=1,
	ymin=-2.1,ymax=2.1,
	xmin=-2.1,xmax=2.1,
	zmin=-4, zmax=4,
	every axis x label/.style={at={(axis cs:\pgfkeysvalueof{/pgfplots/xmax},0,0)},xshift=-5pt,yshift=-1pt},
	xlabel={\scriptsize $x$},
	every axis y label/.style={at={(axis cs:0,\pgfkeysvalueof{/pgfplots/ymax},0)},xshift=4pt,yshift=-4pt},
	ylabel={\scriptsize $y$},
	every axis z label/.style={at={(axis cs:0,0,\pgfkeysvalueof{/pgfplots/zmax})},xshift=0pt,yshift=4pt},
	zlabel={\scriptsize $z$},colormap/cool,clip=false
      ]
      \addplot3[mesh,samples=25,samples y=25,very thin,z buffer=sort] {x^2-y^2};
    \end{axis}
      \end{tikzpicture}
\end{image}

When thinking about a graph of $z= F(x,y)$ at a saddle point, the
instantaneous rate of change in all directions is $0$ and there are
points nearby with $z$-values both less than and greater than the
$z$-value of the saddle point.



\section{The second derivative test}

In theory to identify local extrema verses saddle points, we could
compute the Taylor polynomial of degree $2$ at the critical point in
question, and then identify the Taylor polynomial as either:
\begin{description}
\item[Elliptic paraboloid] Indicating we have found local extrema.
\item[Hyperbolic paraboloid] Indicating that we are at a saddle point.
\end{description}
Fortunately, as we have seen, there is a second derivative test that
does exactly this for us. We will now restate this test in the context
of identifying local extrema.

\begin{theorem}[Second derivative test]
  Given a function $F:\R^2\to \R$, and a critical point $\vec{c}$ where
  \[
  D(\vec{c}) = F^{(2,0)}(\vec{c})F^{(0,2)}(\vec{c})-\left[F^{(1,1)}(\vec{c})\right]^2.
  \]
  \begin{itemize}
  \item If $D(\vec{c})>0$ and $F^{(2,0)}(\vec{c})<0$ then $F$ has  a local maximum at $\vec{c}$.
  \item If $D(\vec{c})>0$ and $F^{(2,0)}(\vec{c})>0$ then $F$ has a local minimum at $\vec{c}$.
  \item	If $D(\vec{c})<0$, then $F$ has a saddle point at $\vec{c}$.
  \item If $D(\vec{c})=0$, the test is inconclusive.
  \end{itemize}
\end{theorem}

We first practice using this test with the function in the previous
example, where we visually determined we had a local maximum and a
saddle point.

\begin{example}
  Let $F(x,y) = x^3-3x-y^2+4y$. Determine whether the function has a
  local minimum, maximum, or saddle point at each critical point.
  \begin{explanation}
    We determined previously that the critical points of $F$ are
    $\vector{-1,2}$ and $\vector{1,2}$. To use the second derivative
    test, we must find the second partial derivatives of $F$:
    \begin{align*}
      F^{(2,0)}(x,y) &= \answer[given]{6x}\\
      F^{(0,2)}(x,y) &= \answer[given]{-2}\\
      F^{(1,1)}(x,y) &= \answer[given]{0}
    \end{align*}
    Thus:
    \[
    D(x,y) = \answer[given]{-12x}
    \]
    Since $D(-1,2) = \answer[given]{12}>0$, and $F^{(2,0)}(-1,2) =
    \answer[given]{-6}$, by the second derivative test, locally looks
    like \wordChoice{\choice[correct]{an elliptic paraboloid opening downward}\choice{an elliptic paraboloid opening upward}\choice{a hyperbolic paraboloid}} at $\vector{-1,2}$, meaning $F$ has a
    local maximum at $\vector{-1,2}$.
    
    Since $D(1,2) = \answer[given]{-12} <0$, by the second derivative
    test, $F$ locally looks like \wordChoice{\choice{an elliptic paraboloid opening downward}\choice{an elliptic paraboloid opening upward}\choice[correct]{a hyperbolic paraboloid}} at
    $\vector{1,2}$ meaning $F$ has a saddle point at $\vector{1,2}$.
    
    The second derivative test has confirmed the visual evidence we
    found before.
  \end{explanation}
\end{example}

\begin{example}
  Find the local extrema of $F(x,y) = x^2y+y^2+xy$.
  \begin{explanation}
    We start by finding the first and second partial derivatives of $F$:
    \begin{align*}
      F^{(1,0)}(x,y) &= 2xy+y\\
      F^{(0,1)}(x,y) &= x^2+2y+x \\
      F^{(2,0)}(x,y) &= 2y\\
      F^{(0,2)}(x,y) &= 2\\
      F^{(1,1)}(x,y) &= 2x+1 
    \end{align*}
    We find the critical points by finding where $\grad F(x,y) =
    \vec{0}$, since the partial derivatives are defined everywhere on
    $\R^2$. With a bit of algebra we can show that there are three
    critical points
    \begin{itemize}
    \item $\vector{0,0}$,
    \item $\vector{\answer[given]{-1},0}$, and
    \item $\vector{\answer[given]{-1/2},\answer[given]{1/8}}$. 
    \end{itemize}
    Now for each critical point $\vec{c}$, we compute compute
    $D(\vec{c}) =
    F^{(2,0)}(\vec{c})F^{(0,2)}(\vec{c})-F^{(1,1)}(\vec{c})^2$.  We
    find
    \[
    D(x,y) = \answer[given]{4y-(2x+1)^2}
    \]
    \begin{itemize}
    \item Since $D(0,0)=\answer[given]{-1}<0$, we see that $F$ has a saddle point at $\vector{0,0}$.
    
    \item Since $D(-1,0)=\answer[given]{-1} <0$, we see that $F$ has a  saddle point at $\vector{-1,0}$.
   
    \item Since $D(-1/2,1/8)=\answer[given]{1/2}>0$ and
      $F^{(2,0)}(-1/2,1/8) = \answer[given]{1/4} > 0$, we see that $F$ has
     a local \wordChoice{\choice{maximum}\choice[correct]{minimum}} at  $\vector{-1/2,1/8}$ .
    \end{itemize}
    Below we see a graph of $F$ and the three critical points.
    \begin{image}
      \begin{tikzpicture}
        \begin{axis}%
          [tick label style={font=\scriptsize},axis on top,
	    axis lines=center,
	    view={100}{45},
	    name=myplot,
	    %xtick=\empty,
	    %ytick={5},
	    %ztick={.7,-.7},
	    minor xtick=1,
	    minor ytick=1,
	    ymin=-.25,ymax=.5,
	    xmin=-2.1,xmax=1.1,
	    zmin=-.25, zmax=.75,
	    every axis x label/.style={at={(axis cs:\pgfkeysvalueof{/pgfplots/xmax},0,0)},xshift=-5pt,yshift=-1pt},
	    xlabel={\scriptsize $x$},
	    every axis y label/.style={at={(axis cs:0,\pgfkeysvalueof{/pgfplots/ymax},0)},xshift=4pt,yshift=-4pt},
	    ylabel={\scriptsize $y$},
	    every axis z label/.style={at={(axis cs:0,0,\pgfkeysvalueof{/pgfplots/zmax})},xshift=0pt,yshift=4pt},
	    zlabel={\scriptsize $z$},colormap/cool
	  ]
          
          \addplot3[domain=-1.75:1,,y domain=-.5:.4,mesh,samples=25,samples y=25,very thin,z buffer=sort] {x^2*y+x*y+y^2};
          
          
          \filldraw [black,] (axis cs:-1,0,0) circle (3pt);
          \filldraw [black,] (axis cs:0,0,0) circle (3pt);
          \filldraw [black,] (axis cs:-.5,.125,-.016) circle (3pt);
          
        \end{axis}
      \end{tikzpicture}
    \end{image}
    Note how this function does not vary much near the critical
    points. Visually it is difficult to determine whether a point is a
    saddle point or local minimum (or even a critical point at
    all!). This is one reason why the second derivative test is so
    important to have.
  \end{explanation}
\end{example}

For some interesting extra reading check out:
\begin{itemize}
\item \link[\textit{Three Observations on a Theme: Editorial Note}, D.A.\ Smith, Mathematics Magazine, May 1985]{http://www.jstor.org/stable/2689908}.
\item \link[\textit{A Surface with One Local Minimum}, J.M.\ Ash and H.\ Sexton, Mathematics Magazine, May 1985]{http://www.jstor.org/stable/2689909}.
\item \link[\textit{``The Only Critical Point in Town'' Test}, I.\ Rosenholtz and  L.\ Smylie, Mathematics Magazine, May 1985]{http://www.jstor.org/stable/2689910}.
\item \link[\textit{Two Mountains Without a Valley}, I.\ Rosenholtz, Mathematics Magazine, February 1987]{http://www.jstor.org/stable/2690137?seq=10}.
\end{itemize}
  
\end{document}
