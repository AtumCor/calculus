\documentclass{ximera}

%\usepackage{todonotes}
%\usepackage{mathtools} %% Required for wide table Curl and Greens
%\usepackage{cuted} %% Required for wide table Curl and Greens
\newcommand{\todo}{}

\usepackage{esint} % for \oiint
\ifxake%%https://math.meta.stackexchange.com/questions/9973/how-do-you-render-a-closed-surface-double-integral
\renewcommand{\oiint}{{\large\bigcirc}\kern-1.56em\iint}
\fi


\graphicspath{
  {./}
  {ximeraTutorial/}
  {basicPhilosophy/}
  {functionsOfSeveralVariables/}
  {normalVectors/}
  {lagrangeMultipliers/}
  {vectorFields/}
  {greensTheorem/}
  {shapeOfThingsToCome/}
  {dotProducts/}
  {partialDerivativesAndTheGradientVector/}
  {../productAndQuotientRules/exercises/}
  {../motionAndPathsInSpace/exercises/}
  {../normalVectors/exercisesParametricPlots/}
  {../continuityOfFunctionsOfSeveralVariables/exercises/}
  {../partialDerivativesAndTheGradientVector/exercises/}
  {../directionalDerivativeAndChainRule/exercises/}
  {../commonCoordinates/exercisesCylindricalCoordinates/}
  {../commonCoordinates/exercisesSphericalCoordinates/}
  {../greensTheorem/exercisesCurlAndLineIntegrals/}
  {../greensTheorem/exercisesDivergenceAndLineIntegrals/}
  {../shapeOfThingsToCome/exercisesDivergenceTheorem/}
  {../greensTheorem/}
  {../shapeOfThingsToCome/}
  {../separableDifferentialEquations/exercises/}
  {vectorFields/}
}

\newcommand{\mooculus}{\textsf{\textbf{MOOC}\textnormal{\textsf{ULUS}}}}

\usepackage{tkz-euclide}\usepackage{tikz}
\usepackage{tikz-cd}
\usetikzlibrary{arrows}
\tikzset{>=stealth,commutative diagrams/.cd,
  arrow style=tikz,diagrams={>=stealth}} %% cool arrow head
\tikzset{shorten <>/.style={ shorten >=#1, shorten <=#1 } } %% allows shorter vectors

\usetikzlibrary{backgrounds} %% for boxes around graphs
\usetikzlibrary{shapes,positioning}  %% Clouds and stars
\usetikzlibrary{matrix} %% for matrix
\usepgfplotslibrary{polar} %% for polar plots
\usepgfplotslibrary{fillbetween} %% to shade area between curves in TikZ
%\usetkzobj{all} obsolete

\usepackage[makeroom]{cancel} %% for strike outs
%\usepackage{mathtools} %% for pretty underbrace % Breaks Ximera
%\usepackage{multicol}
\usepackage{pgffor} %% required for integral for loops



%% http://tex.stackexchange.com/questions/66490/drawing-a-tikz-arc-specifying-the-center
%% Draws beach ball
\tikzset{pics/carc/.style args={#1:#2:#3}{code={\draw[pic actions] (#1:#3) arc(#1:#2:#3);}}}



\usepackage{array}
\setlength{\extrarowheight}{+.1cm}
\newdimen\digitwidth
\settowidth\digitwidth{9}
\def\divrule#1#2{
\noalign{\moveright#1\digitwidth
\vbox{\hrule width#2\digitwidth}}}





\newcommand{\RR}{\mathbb R}
\newcommand{\R}{\mathbb R}
\newcommand{\N}{\mathbb N}
\newcommand{\Z}{\mathbb Z}

\newcommand{\sagemath}{\textsf{SageMath}}


%\renewcommand{\d}{\,d\!}
\renewcommand{\d}{\mathop{}\!d}
\newcommand{\dd}[2][]{\frac{\d #1}{\d #2}}
\newcommand{\pp}[2][]{\frac{\partial #1}{\partial #2}}
\renewcommand{\l}{\ell}
\newcommand{\ddx}{\frac{d}{\d x}}

\newcommand{\zeroOverZero}{\ensuremath{\boldsymbol{\tfrac{0}{0}}}}
\newcommand{\inftyOverInfty}{\ensuremath{\boldsymbol{\tfrac{\infty}{\infty}}}}
\newcommand{\zeroOverInfty}{\ensuremath{\boldsymbol{\tfrac{0}{\infty}}}}
\newcommand{\zeroTimesInfty}{\ensuremath{\small\boldsymbol{0\cdot \infty}}}
\newcommand{\inftyMinusInfty}{\ensuremath{\small\boldsymbol{\infty - \infty}}}
\newcommand{\oneToInfty}{\ensuremath{\boldsymbol{1^\infty}}}
\newcommand{\zeroToZero}{\ensuremath{\boldsymbol{0^0}}}
\newcommand{\inftyToZero}{\ensuremath{\boldsymbol{\infty^0}}}



\newcommand{\numOverZero}{\ensuremath{\boldsymbol{\tfrac{\#}{0}}}}
\newcommand{\dfn}{\textbf}
%\newcommand{\unit}{\,\mathrm}
\newcommand{\unit}{\mathop{}\!\mathrm}
\newcommand{\eval}[1]{\bigg[ #1 \bigg]}
\newcommand{\seq}[1]{\left( #1 \right)}
\renewcommand{\epsilon}{\varepsilon}
\renewcommand{\phi}{\varphi}


\renewcommand{\iff}{\Leftrightarrow}

\DeclareMathOperator{\arccot}{arccot}
\DeclareMathOperator{\arcsec}{arcsec}
\DeclareMathOperator{\arccsc}{arccsc}
\DeclareMathOperator{\si}{Si}
\DeclareMathOperator{\scal}{scal}
\DeclareMathOperator{\sign}{sign}


%% \newcommand{\tightoverset}[2]{% for arrow vec
%%   \mathop{#2}\limits^{\vbox to -.5ex{\kern-0.75ex\hbox{$#1$}\vss}}}
\newcommand{\arrowvec}[1]{{\overset{\rightharpoonup}{#1}}}
%\renewcommand{\vec}[1]{\arrowvec{\mathbf{#1}}}
\renewcommand{\vec}[1]{{\overset{\boldsymbol{\rightharpoonup}}{\mathbf{#1}}}\hspace{0in}}

\newcommand{\point}[1]{\left(#1\right)} %this allows \vector{ to be changed to \vector{ with a quick find and replace
\newcommand{\pt}[1]{\mathbf{#1}} %this allows \vec{ to be changed to \vec{ with a quick find and replace
\newcommand{\Lim}[2]{\lim_{\point{#1} \to \point{#2}}} %Bart, I changed this to point since I want to use it.  It runs through both of the exercise and exerciseE files in limits section, which is why it was in each document to start with.

\DeclareMathOperator{\proj}{\mathbf{proj}}
\newcommand{\veci}{{\boldsymbol{\hat{\imath}}}}
\newcommand{\vecj}{{\boldsymbol{\hat{\jmath}}}}
\newcommand{\veck}{{\boldsymbol{\hat{k}}}}
\newcommand{\vecl}{\vec{\boldsymbol{\l}}}
\newcommand{\uvec}[1]{\mathbf{\hat{#1}}}
\newcommand{\utan}{\mathbf{\hat{t}}}
\newcommand{\unormal}{\mathbf{\hat{n}}}
\newcommand{\ubinormal}{\mathbf{\hat{b}}}

\newcommand{\dotp}{\bullet}
\newcommand{\cross}{\boldsymbol\times}
\newcommand{\grad}{\boldsymbol\nabla}
\newcommand{\divergence}{\grad\dotp}
\newcommand{\curl}{\grad\cross}
%\DeclareMathOperator{\divergence}{divergence}
%\DeclareMathOperator{\curl}[1]{\grad\cross #1}
\newcommand{\lto}{\mathop{\longrightarrow\,}\limits}

\renewcommand{\bar}{\overline}

\colorlet{textColor}{black}
\colorlet{background}{white}
\colorlet{penColor}{blue!50!black} % Color of a curve in a plot
\colorlet{penColor2}{red!50!black}% Color of a curve in a plot
\colorlet{penColor3}{red!50!blue} % Color of a curve in a plot
\colorlet{penColor4}{green!50!black} % Color of a curve in a plot
\colorlet{penColor5}{orange!80!black} % Color of a curve in a plot
\colorlet{penColor6}{yellow!70!black} % Color of a curve in a plot
\colorlet{fill1}{penColor!20} % Color of fill in a plot
\colorlet{fill2}{penColor2!20} % Color of fill in a plot
\colorlet{fillp}{fill1} % Color of positive area
\colorlet{filln}{penColor2!20} % Color of negative area
\colorlet{fill3}{penColor3!20} % Fill
\colorlet{fill4}{penColor4!20} % Fill
\colorlet{fill5}{penColor5!20} % Fill
\colorlet{gridColor}{gray!50} % Color of grid in a plot

\newcommand{\surfaceColor}{violet}
\newcommand{\surfaceColorTwo}{redyellow}
\newcommand{\sliceColor}{greenyellow}




\pgfmathdeclarefunction{gauss}{2}{% gives gaussian
  \pgfmathparse{1/(#2*sqrt(2*pi))*exp(-((x-#1)^2)/(2*#2^2))}%
}


%%%%%%%%%%%%%
%% Vectors
%%%%%%%%%%%%%

%% Simple horiz vectors
\renewcommand{\vector}[1]{\left\langle #1\right\rangle}


%% %% Complex Horiz Vectors with angle brackets
%% \makeatletter
%% \renewcommand{\vector}[2][ , ]{\left\langle%
%%   \def\nextitem{\def\nextitem{#1}}%
%%   \@for \el:=#2\do{\nextitem\el}\right\rangle%
%% }
%% \makeatother

%% %% Vertical Vectors
%% \def\vector#1{\begin{bmatrix}\vecListA#1,,\end{bmatrix}}
%% \def\vecListA#1,{\if,#1,\else #1\cr \expandafter \vecListA \fi}

%%%%%%%%%%%%%
%% End of vectors
%%%%%%%%%%%%%

%\newcommand{\fullwidth}{}
%\newcommand{\normalwidth}{}



%% makes a snazzy t-chart for evaluating functions
%\newenvironment{tchart}{\rowcolors{2}{}{background!90!textColor}\array}{\endarray}

%%This is to help with formatting on future title pages.
\newenvironment{sectionOutcomes}{}{}



%% Flowchart stuff
%\tikzstyle{startstop} = [rectangle, rounded corners, minimum width=3cm, minimum height=1cm,text centered, draw=black]
%\tikzstyle{question} = [rectangle, minimum width=3cm, minimum height=1cm, text centered, draw=black]
%\tikzstyle{decision} = [trapezium, trapezium left angle=70, trapezium right angle=110, minimum width=3cm, minimum height=1cm, text centered, draw=black]
%\tikzstyle{question} = [rectangle, rounded corners, minimum width=3cm, minimum height=1cm,text centered, draw=black]
%\tikzstyle{process} = [rectangle, minimum width=3cm, minimum height=1cm, text centered, draw=black]
%\tikzstyle{decision} = [trapezium, trapezium left angle=70, trapezium right angle=110, minimum width=3cm, minimum height=1cm, text centered, draw=black]

\author{Jim Talamo}
\outcome{Use higher order polynomials to approximate a sufficiently differentiable function.}

\title[Dig-In:]{Higher Order Polynomial Approximations}

\begin{document}
\begin{abstract}
We can approximate sufficiently differentiable functions by polynomials.
\end{abstract}
\maketitle

Previously, we have seen that if a function is differentiable on an
open interval containing a point $x=c$, we can approximate the
function near $x=c$ by the tangent line at $x=c$.  Visually, we recall
that as we ``zoom in" on the graph around $x=c$, it becomes less
distinguishable from the graph of a line (which is the tangent line at
$x=c$).

\begin{image}
\begin{tikzpicture}
  \begin{axis}[
            domain=0:6, range=0:7,
            ymin=-.2,ymax=7,
            width=6in,
            height=2.5in, %% Hard coded height! Moreover this effects the aspect ratio of the zoom--sort of BAD
            axis lines=none,
          ]   
          \addplot [draw=none, fill=textColor!10!background] plot coordinates {(.8,1.6) (2.834,5)} \closedcycle; %% zoom fill
          \addplot [draw=none, fill=textColor!10!background] plot coordinates {(2.834,5) (4.166,5)} \closedcycle; %% zoom fill
          \addplot [draw=none, fill=background] plot coordinates {(1.2,1.6) (4.166,5)} \closedcycle; %% zoom fill
          \addplot [draw=none, fill=background] plot coordinates {(.8,1.6) (1.2,1.6)} \closedcycle; %% zoom fill

          \addplot [draw=none, fill=textColor!10!background] plot coordinates {(3.3,3.6) (5.334,5)} \closedcycle; %% zoom fill
          \addplot [draw=none, fill=textColor!10!background] plot coordinates {(5.334,5) (6.666,5)} \closedcycle; %% zoom fill
          \addplot [draw=none, fill=background] plot coordinates {(3.7,3.6) (6.666,5)} \closedcycle; %% zoom fill
          \addplot [draw=none, fill=background] plot coordinates {(3.3,3.6) (3.7,3.6)} \closedcycle; %% zoom fill
          
          \addplot [draw=none, fill=textColor!10!background] plot coordinates {(3.7,2.4) (6.666,1)} \closedcycle; %% zoom fill
          \addplot [draw=none, fill=textColor!10!background] plot coordinates {(3.3,2.4) (3.7,2.4)} \closedcycle; %% zoom fill
          \addplot [draw=none, fill=background] plot coordinates {(3.3,2.4) (5.334,1)} \closedcycle; %% zoom fill          
          \addplot [draw=none, fill=background] plot coordinates {(5.334,1) (6.666,1)} \closedcycle; %% zoom fill
          

          \addplot [draw=none, fill=textColor!10!background] plot coordinates {(.8,.4) (2.834,1)} \closedcycle; %% zoom fill
          \addplot [draw=none, fill=textColor!10!background] plot coordinates {(2.834,1) (4.166,1)} \closedcycle; %% zoom fill
          \addplot [draw=none, fill=background] plot coordinates {(1.2,.4) (4.166,1)} \closedcycle; %% zoom fill
          \addplot [draw=none, fill=background] plot coordinates {(.8,.4) (1.2,.4)} \closedcycle; %% zoom fill

          \addplot[very thick,penColor, smooth,domain=(0:1.833)] {-1/(x-2)};
          \addplot[very thick,penColor, smooth,domain=(2.834:4.166)] {3.333/(2.050-.3*x)-0.333}; %% 2.5 to 4.333
          %\addplot[very thick,penColor, smooth,domain=(5.334:6.666)] {11.11/(1.540-.09*x)-8.109}; %% 5 to 6.833
          \addplot[very thick,penColor, smooth,domain=(5.334:6.666)] {x-3}; %% 5 to 6.833
          
          \addplot[color=penColor,fill=penColor,only marks,mark=*] coordinates{(1,1)};  %% point to be zoomed
          \addplot[color=penColor,fill=penColor,only marks,mark=*] coordinates{(3.5,3)};  %% zoomed pt 1
          \addplot[color=penColor,fill=penColor,only marks,mark=*] coordinates{(6,3)};  %% zoomed pt 2

          \addplot [->,textColor] plot coordinates {(0,0) (0,6)}; %% axis
          \addplot [->,textColor] plot coordinates {(0,0) (2,0)}; %% axis
          
          \addplot [textColor!50!background] plot coordinates {(.8,.4) (.8,1.6)}; %% box around pt
          \addplot [textColor!50!background] plot coordinates {(1.2,.4) (1.2,1.6)}; %% box around pt
          \addplot [textColor!50!background] plot coordinates {(.8,1.6) (1.2,1.6)}; %% box around pt
          \addplot [textColor!50!background] plot coordinates {(.8,.4) (1.2,.4)}; %% box around pt
          
          \addplot [textColor!50!background] plot coordinates {(2.834,1) (2.834,5)}; %% zoomed box 1
          \addplot [textColor!50!background] plot coordinates {(4.166,1) (4.166,5)}; %% zoomed box 1
          \addplot [textColor!50!background] plot coordinates {(2.834,1) (4.166,1)}; %% zoomed box 1
          \addplot [textColor!50!background] plot coordinates {(2.834,5) (4.166,5)}; %% zoomed box 1

          \addplot [textColor] plot coordinates {(3.3,2.4) (3.3,3.6)}; %% box around zoomed pt
          \addplot [textColor] plot coordinates {(3.7,2.4) (3.7,3.6)}; %% box around zoomed pt
          \addplot [textColor] plot coordinates {(3.3,3.6) (3.7,3.6)}; %% box around zoomed pt
          \addplot [textColor] plot coordinates {(3.3,2.4) (3.7,2.4)}; %% box around zoomed pt

          \addplot [textColor] plot coordinates {(5.334,1) (5.334,5)}; %% zoomed box 2
          \addplot [textColor] plot coordinates {(6.666,1) (6.666,5)}; %% zoomed box 2
          \addplot [textColor] plot coordinates {(5.334,1) (6.666,1)}; %% zoomed box 2
          \addplot [textColor] plot coordinates {(5.334,5) (6.666,5)}; %% zoomed box 2

          \node at (axis cs:2.2,0) [anchor=east] {$x$};
          \node at (axis cs:0,6.6) [anchor=north] {$y$};
        \end{axis}
\end{tikzpicture}

\end{image}

When $y$ is given by an explicit formula in terms of $x$, the point
$(x_0,y_0)$ is found by evaluating the $y=f(x)$ at $x_0$, and the
slope is found by evaluating the derivative $f'(x)$ at $x=x_0$.  By
taking advantage of the point-slope form of a line

\[
y-y_0=m\left(x-x_0\right),
\]
an equation for the tangent line is found.  Let's explore this in the
context of an example.

\begin{question}
Suppose that we want to approximate $e^{2x}$ near $x=0$.  We pick
$c=0$ as the point off of which our approximation will be based, and
find the equation of the tangent line to $y=e^{2x}$ at $x=0$.
\begin{itemize}
\item $y_0=f(0) = \answer[given]{1}$, so the point $(0,1)$ is on the line.
\item $f'(x) = \answer[given]{2e^{2x}}$, so the slope is found by noting $m_{tan} = f'(0) = \answer{2}$.
\end{itemize}
Thus, the equation of the tangent line is

\begin{align*}
y-y_0 = m_{tan} (x-x_0) \\
y- \answer{1} &=\answer{2}(x-0) \\
y&=2x+1 .
\end{align*}

If we ``zoom in'' on the graphs of the function $y=e^{2x}$ and its tangent line at $x=0$, denoted by $l_0(x)=2x+1$, we see the following picture.

\begin{image}
\begin{tikzpicture}

\begin{axis}
	[
	domain=-.5:.5, ymax=2.2,xmax=.25, ymin=-.25, xmin=-.25,
	axis lines=center, xlabel=$x$, ylabel=$y$,
	xtick={-.2,-.1,.1,.2}, ytick={1,2}, 
	every axis y label/.style={at=(current axis.above origin),anchor=south},
	every axis x label/.style={at=(current axis.right of origin),anchor=west},
	axis on top,
	typeset ticklabels with strut,
	]

	\addplot [draw=penColor,very thick, smooth] {exp(2*x)};
	\addplot [draw=penColor2,very thick, smooth] {2*x+1};	
	\node at (axis cs:.15,1.6) [penColor] {$y=e^{2x}$};
	\node at (axis cs:-.1,.5) [penColor2] {$l_0(x)=2x+1$};	
\end{axis}

\end{tikzpicture}
\end{image}
\end{question}


From tangent line approximation, we can approximate values of $e^{2x}$
near $x=0$.  Visually, we can see this since the graphs are quite
close.  Computationally, we obtain the approximations by plugging
$x$-values into the equation of the tangent line; for instance, we can
approximate $e^{.2}$ by noting

\[
e^{.2} = f(.1) \approx 2(.1)+1 =1.2.
\]

The actual value of $e^{.2}$ to three decimal places is $1.221$, so
the simple arithmetic needed to estimate using the tangent line
produces a reasonable approximation.  As we ``zoom out'' to a larger
viewing window, however, the graphs start to become quite different.

\begin{image}
\begin{tikzpicture}

\begin{axis}
	[
	domain=-2.25:2.25, ymax=9.5,xmax=2.25, ymin=-5.25, xmin=-2.25,
	axis lines=center, xlabel=$x$, ylabel=$y$,
	xtick={-2,-1,1,2}, ytick={-4,-2,2,4,6,8}, 
	every axis y label/.style={at=(current axis.above origin),anchor=south},
	every axis x label/.style={at=(current axis.right of origin),anchor=west},
	axis on top,
	typeset ticklabels with strut,
	]

	\addplot [draw=penColor,very thick, smooth] {exp(2*x)};
	\addplot [draw=penColor2,very thick, smooth] {2*x+1};	
	\node at (axis cs:1.5,7) [penColor] {$y=e^{2x}$};
	\node at (axis cs:1.5,1.5) [penColor2] {$l_0(x)=2x+1$};	
\end{axis}

\end{tikzpicture}
\end{image}

Since evaluating polynomials involves only arithmetic operations, we
would like to be able to use them to give better results than the
tangent line approximation. Also, polynomials are easy to integrate
and differentiate, so it would be nice to use polynomial
approximations in applications that involve these operations.  This
will require that we try to extract the idea from the tangent line
approximation in a way that allows us to generalize it.

\section{Revisiting the Tangent Line Approximation}
Let's look for a first degree polynomial of the form 

\[
p_1(x) = a_0+a_1x,
\]
where $a_0$ and $a_1$ are constants that must be determined.
\begin{question}
  What do we want out of our approximation?
\begin{explanation}
\begin{itemize}
\item We certainly want the function and the approximation to agree at
  the $x$-value off of which the approximation is based; that is, we
  want

\[
p_1(0) =f(0).
\]

In our example, we note

\begin{itemize}
\item $f(0) = e^{2(0)} =1$.
\item $p_1(0) = a_0+a_1(0) = a_0$.
\end{itemize}
Thus, the requirement $p_1(0) =f(0)$ gives us that $a_0=\answer{1}$.
 
\item We also want to make sure that the function and the approximation change at the same initial rate so that both graphs head in similar general directions.  Since the derivative measures the rate of change, we require

\[
p_1'(0) =f'(0).
\] 

In our example, we note

\begin{itemize}
\item $f'(x) = 2e^{2x}$, so $f'(0) = 2$.
\item $p_1'(x) = a_1$, so $p_1'(0) = a_1$.
\end{itemize}
Thus, the requirement $p_1'(0) =f'(0)$ gives us that $a_1=\answer{2}$.

\end{itemize}

Our approximation is thus $p_1(x) = a_0+a_1x = 1+2x$, which matches the equation of the tangent line at $x=0$.
\end{explanation}
\end{question}

\section{A Quadratic Approximation}

While this should not be too surprising, it does allow for us to think
of conditions that will allow for higher degree polynomial
approximations.  Suppose that we want to use a quadratic polynomial
$p_2(x)$ of the form

\[
p_2(x) = a_0+a_1x+a_2x^2
\]

for making estimates.  Note that $f(x)=e^{2x}$ is clearly not linear;
in fact, it is concave-up
on its domain.  Note that by drawing tangent lines at different points
near $x=0$, the slopes are different, which is roughly what concavity
quantifies.  Slopes of tangent lines are found from the first
derivative, so in order to measure how these slopes are changing, we
should look at the derivative of the first derivative.  This is really
nothing new; we know already that concavity is measured using the
second derivative.

We'll keep the previous two conditions - that $p_2(0) = f(0)$ and $p_2'(0) = f'(0)$ - and also require that $p_2''(0) = f''(0)$.  We thus look for look for a polynomial
\[p_2(x) = a_0+a_1x+a_2x^2
\]
whose coefficients are found by the requirements
\begin{align*}
1) ~ p_2(0) & = f(0) \\
2) ~ p_2'(0) & = f'(0) \\
3) ~ p_2''(0) & = f''(0)
\end{align*}

By following the previous example, the reader can (and should) verify that we still have $a_0=1$ and $a_1=2$.  To find $a_2$, note that

\begin{itemize}
\item $f''(x) = 4e^{2x}$, so $f''(0) = \answer{4}$.
\item $p_2'(x) = a_1+2a_2x$, so $p_2''(x) = 2a_2$ and thus $p_2''(0) = 2 a_2$.
\end{itemize}
Thus, the requirement $p_2''(0) =f''(0)$ gives us that $2a_2=4$, or $a_2=2$.

The quadratic approximation is thus 
\begin{align*}
p_2(x) &= a_0+a_1x+a_2x^2 \\
&= 1+2x+2x^2.
\end{align*}
Let's now explore our approximations.  Geometrically, we can interpret the effectiveness of the approximations by looking at their graphs.

\begin{image}
\begin{tikzpicture}

\begin{axis}
	[
	domain=-2.25:2.25, ymax=9.5,xmax=2.25, ymin=-5.25, xmin=-2.25,
	axis lines=center, xlabel=$x$, ylabel=$y$,
	xtick={-2,-1,1,2}, ytick={-4,-2,2,4,6,8}, 
	every axis y label/.style={at=(current axis.above origin),anchor=south},
	every axis x label/.style={at=(current axis.right of origin),anchor=west},
	axis on top,
	typeset ticklabels with strut,
	]

	\addplot [draw=penColor,very thick, smooth] {exp(2*x)};
	\addplot [draw=penColor2,very thick, smooth] {2*x+1};	
	\addplot [draw=penColor4,very thick, smooth] {2*x^2+2*x+1};	
	\node at (axis cs:.5,8) [penColor] {$y=e^{2x}$};
	\node at (axis cs:1.5,1.8) [penColor2] {$y=p_1(x)$};
	\node at (axis cs:-1.2,5) [penColor4] {$y=p_2(x)$};		
\end{axis}

\end{tikzpicture}
\end{image}

We can also explore the approximations quantitatively for a given $x$-value.  For instance, if we want to approximate $e^{.2}$, we note that $f(x)=e^{2x}$, so $e^{.2}=f(.1)$.  We thus approximate $e^{.2}$ by evaluating the polynomials at $x=.1$.

\begin{itemize}
\item $e^{.2} \approx p_1(.1) = 1+2(.1) = 1.2$
\item $e^{.2} \approx p_2(.1) = 1+2(.1)+2(.1)^2 = 1.22$
\end{itemize}

By noting that the actual value to three decimal place is $1.221$, we can see that the quadratic approximation is better! 

\section{Higher Order Approximations}

We can continue to look for higher degree polynomial approximations.  Note that our approximations above require that the function be sufficiently differentiable at the point at which we wish to base the approximation.

\begin{definition}
Let $f(x)$ be a function whose first $n$ derivatives exist at $x=c$.  The \underline{$n$-th order Taylor polynomial centered at $x=c$} is the polynomial

\[
p_n(x) = \sum_{k=0}^n a_k(x-c)^k = a_0+a_1(x-c)+a_2(x-c)^2+\ldots+ a_n(x-c)^n,
\]
whose coefficients $a_k$ are found by requiring $p_n^{(k)}(c) = f^{(k)}(c)$ for each $0 \leq k \leq n$.
\end{definition}

We will develop a more computationally efficient method for computing Taylor Polynomials in the next section, but we conclude this section with a question that explores the ideas put forth so far.

%\begin{example}
%Could $p_2(x) = 1+3(x+1)+5(x+1)^2$ be the second order Taylor Polynomial centered at $x=-1$ for $f(x) = \sqrt{3+2x}$?
%
%\begin{explanation}
%$p_2(x)$ will be the second order Taylor Polynomial centered at $x=-1$ for $f(x)=\sqrt{3+2x}$ if and only if
%
%\begin{align*}
%1) ~ p_2(-1) & = f(-1) \\
%2) ~ p_2'(-1) & = f'(-1) \\
%3) ~ p_2''(-1) & = f''(-1)
%\end{align*}
%
%We thus check whether the requirements hold.
%
%
%We first check if $p_2(-1) = f(-1)$.  
%\begin{itemize}
%\item $p_2(x) = 1+3(x+1)+5(x+1)^2$, so $p_2(-1) = \answer[given]{1}.$
%\item $f(x) = \sqrt{3+2x}$ so $f(-1) = \answer[given]{1}.$
%\end{itemize}
%
%Since $p_2(-1)$ \wordChoice{\choice[correct]{$=$}\choice{$\neq$}} $f(-1)$, we see that $p_2(x)$ \wordChoice{\choice[correct]{might}\choice{must not}} be the the second order Taylor Polynomial centered at $x=-1$ for $f(x)=\sqrt{3+2x}$.
%
%To continue, we must now check if $p_2'(-1) = f'(-1)$.  
%\begin{itemize}
%\item $p_2'(x) = \answer{3+10(x+1)^2}$, so $p_2(-1) = \answer[given]{3}.$
%\item $f'(x) = \frac{1}{\sqrt{3+2x}}$ so $f(-1) = \answer[given]{1}.$
%\end{itemize}
%
%Since $p_2'(-1)$ \wordChoice{\choice{$=$}\choice[correct]{$\neq$}} $f'(-1)$, we see that $p_2(x)$ \wordChoice{\choice{might}\choice[correct]{must not}} be the the second order Taylor Polynomial centered at $x=-1$ for $f(x)=\sqrt{3+2x}$.
%
%\end{explanation} 
%\end{example}

\begin{question}
Suppose that $f(x)$ is a function for which $f''(2) = 4$.  

\begin{question}
Which of the following that \emph{could} be the second degree Taylor polynomial centered at $x=2$ for $f(x)$?

\begin{multipleChoice}
\choice{$2+3(x-2)+4(x-2)^2$}
\choice{$2+3(x-2)-4(x-2)^2$}
\choice[correct]{$3+2(x-2)^2$}
\choice{$3-2(x-2)^2$}
\end{multipleChoice}

\begin{feedback}
We must find the polynomial $p_2(x)$ for which $p_2''(2) = 4$.  Of the polynomials listed, only $3+2(x+2)^2$ has this property.
\end{feedback}

\begin{question}
Let $p_2(x) =3+2(x-2)^2$.  

Is there enough information to determine what $f'(2)$ is?

\begin{multipleChoice}
\choice[correct]{Yes}
\choice{No}
\end{multipleChoice}

\begin{question}
We find that $f'(2) = \answer[given]{0}$.
\end{question}

Is there enough information to determine what $f(3)$ is?

\begin{multipleChoice}
\choice{Yes}
\choice[correct]{No}
\end{multipleChoice}


\end{question}
\begin{feedback}
Since $p_2(x) = 3+2(x+2)^2$ is the second degree Taylor polynomial centered at $x=2$ for $f(x)$, we only know for sure that $f(2)=p_2(2)$, $f'(2)=p_2'(2)$, and $f''(2)=p_2''(2)$.  We can use $p_2(x)$ to \underline{approximate} $f(x)$ at other $x$-values, but there is no guarantee that $p_2(x)$ and $f(x)$ will agree at any $x$-value other than $x=2$.  The curious reader may inquire whether this would provide a reasonable approximation, and this will be discussed in a subsequent section.
\end{feedback}

\end{question}


\end{question}
\end{document}
