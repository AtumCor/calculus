\documentclass{ximera}

%\usepackage{todonotes}
%\usepackage{mathtools} %% Required for wide table Curl and Greens
%\usepackage{cuted} %% Required for wide table Curl and Greens
\newcommand{\todo}{}

\usepackage{esint} % for \oiint
\ifxake%%https://math.meta.stackexchange.com/questions/9973/how-do-you-render-a-closed-surface-double-integral
\renewcommand{\oiint}{{\large\bigcirc}\kern-1.56em\iint}
\fi


\graphicspath{
  {./}
  {ximeraTutorial/}
  {basicPhilosophy/}
  {functionsOfSeveralVariables/}
  {normalVectors/}
  {lagrangeMultipliers/}
  {vectorFields/}
  {greensTheorem/}
  {shapeOfThingsToCome/}
  {dotProducts/}
  {partialDerivativesAndTheGradientVector/}
  {../productAndQuotientRules/exercises/}
  {../motionAndPathsInSpace/exercises/}
  {../normalVectors/exercisesParametricPlots/}
  {../continuityOfFunctionsOfSeveralVariables/exercises/}
  {../partialDerivativesAndTheGradientVector/exercises/}
  {../directionalDerivativeAndChainRule/exercises/}
  {../commonCoordinates/exercisesCylindricalCoordinates/}
  {../commonCoordinates/exercisesSphericalCoordinates/}
  {../greensTheorem/exercisesCurlAndLineIntegrals/}
  {../greensTheorem/exercisesDivergenceAndLineIntegrals/}
  {../shapeOfThingsToCome/exercisesDivergenceTheorem/}
  {../greensTheorem/}
  {../shapeOfThingsToCome/}
  {../separableDifferentialEquations/exercises/}
  {vectorFields/}
}

\newcommand{\mooculus}{\textsf{\textbf{MOOC}\textnormal{\textsf{ULUS}}}}

\usepackage{tkz-euclide}\usepackage{tikz}
\usepackage{tikz-cd}
\usetikzlibrary{arrows}
\tikzset{>=stealth,commutative diagrams/.cd,
  arrow style=tikz,diagrams={>=stealth}} %% cool arrow head
\tikzset{shorten <>/.style={ shorten >=#1, shorten <=#1 } } %% allows shorter vectors

\usetikzlibrary{backgrounds} %% for boxes around graphs
\usetikzlibrary{shapes,positioning}  %% Clouds and stars
\usetikzlibrary{matrix} %% for matrix
\usepgfplotslibrary{polar} %% for polar plots
\usepgfplotslibrary{fillbetween} %% to shade area between curves in TikZ
%\usetkzobj{all} obsolete

\usepackage[makeroom]{cancel} %% for strike outs
%\usepackage{mathtools} %% for pretty underbrace % Breaks Ximera
%\usepackage{multicol}
\usepackage{pgffor} %% required for integral for loops



%% http://tex.stackexchange.com/questions/66490/drawing-a-tikz-arc-specifying-the-center
%% Draws beach ball
\tikzset{pics/carc/.style args={#1:#2:#3}{code={\draw[pic actions] (#1:#3) arc(#1:#2:#3);}}}



\usepackage{array}
\setlength{\extrarowheight}{+.1cm}
\newdimen\digitwidth
\settowidth\digitwidth{9}
\def\divrule#1#2{
\noalign{\moveright#1\digitwidth
\vbox{\hrule width#2\digitwidth}}}





\newcommand{\RR}{\mathbb R}
\newcommand{\R}{\mathbb R}
\newcommand{\N}{\mathbb N}
\newcommand{\Z}{\mathbb Z}

\newcommand{\sagemath}{\textsf{SageMath}}


%\renewcommand{\d}{\,d\!}
\renewcommand{\d}{\mathop{}\!d}
\newcommand{\dd}[2][]{\frac{\d #1}{\d #2}}
\newcommand{\pp}[2][]{\frac{\partial #1}{\partial #2}}
\renewcommand{\l}{\ell}
\newcommand{\ddx}{\frac{d}{\d x}}

\newcommand{\zeroOverZero}{\ensuremath{\boldsymbol{\tfrac{0}{0}}}}
\newcommand{\inftyOverInfty}{\ensuremath{\boldsymbol{\tfrac{\infty}{\infty}}}}
\newcommand{\zeroOverInfty}{\ensuremath{\boldsymbol{\tfrac{0}{\infty}}}}
\newcommand{\zeroTimesInfty}{\ensuremath{\small\boldsymbol{0\cdot \infty}}}
\newcommand{\inftyMinusInfty}{\ensuremath{\small\boldsymbol{\infty - \infty}}}
\newcommand{\oneToInfty}{\ensuremath{\boldsymbol{1^\infty}}}
\newcommand{\zeroToZero}{\ensuremath{\boldsymbol{0^0}}}
\newcommand{\inftyToZero}{\ensuremath{\boldsymbol{\infty^0}}}



\newcommand{\numOverZero}{\ensuremath{\boldsymbol{\tfrac{\#}{0}}}}
\newcommand{\dfn}{\textbf}
%\newcommand{\unit}{\,\mathrm}
\newcommand{\unit}{\mathop{}\!\mathrm}
\newcommand{\eval}[1]{\bigg[ #1 \bigg]}
\newcommand{\seq}[1]{\left( #1 \right)}
\renewcommand{\epsilon}{\varepsilon}
\renewcommand{\phi}{\varphi}


\renewcommand{\iff}{\Leftrightarrow}

\DeclareMathOperator{\arccot}{arccot}
\DeclareMathOperator{\arcsec}{arcsec}
\DeclareMathOperator{\arccsc}{arccsc}
\DeclareMathOperator{\si}{Si}
\DeclareMathOperator{\scal}{scal}
\DeclareMathOperator{\sign}{sign}


%% \newcommand{\tightoverset}[2]{% for arrow vec
%%   \mathop{#2}\limits^{\vbox to -.5ex{\kern-0.75ex\hbox{$#1$}\vss}}}
\newcommand{\arrowvec}[1]{{\overset{\rightharpoonup}{#1}}}
%\renewcommand{\vec}[1]{\arrowvec{\mathbf{#1}}}
\renewcommand{\vec}[1]{{\overset{\boldsymbol{\rightharpoonup}}{\mathbf{#1}}}\hspace{0in}}

\newcommand{\point}[1]{\left(#1\right)} %this allows \vector{ to be changed to \vector{ with a quick find and replace
\newcommand{\pt}[1]{\mathbf{#1}} %this allows \vec{ to be changed to \vec{ with a quick find and replace
\newcommand{\Lim}[2]{\lim_{\point{#1} \to \point{#2}}} %Bart, I changed this to point since I want to use it.  It runs through both of the exercise and exerciseE files in limits section, which is why it was in each document to start with.

\DeclareMathOperator{\proj}{\mathbf{proj}}
\newcommand{\veci}{{\boldsymbol{\hat{\imath}}}}
\newcommand{\vecj}{{\boldsymbol{\hat{\jmath}}}}
\newcommand{\veck}{{\boldsymbol{\hat{k}}}}
\newcommand{\vecl}{\vec{\boldsymbol{\l}}}
\newcommand{\uvec}[1]{\mathbf{\hat{#1}}}
\newcommand{\utan}{\mathbf{\hat{t}}}
\newcommand{\unormal}{\mathbf{\hat{n}}}
\newcommand{\ubinormal}{\mathbf{\hat{b}}}

\newcommand{\dotp}{\bullet}
\newcommand{\cross}{\boldsymbol\times}
\newcommand{\grad}{\boldsymbol\nabla}
\newcommand{\divergence}{\grad\dotp}
\newcommand{\curl}{\grad\cross}
%\DeclareMathOperator{\divergence}{divergence}
%\DeclareMathOperator{\curl}[1]{\grad\cross #1}
\newcommand{\lto}{\mathop{\longrightarrow\,}\limits}

\renewcommand{\bar}{\overline}

\colorlet{textColor}{black}
\colorlet{background}{white}
\colorlet{penColor}{blue!50!black} % Color of a curve in a plot
\colorlet{penColor2}{red!50!black}% Color of a curve in a plot
\colorlet{penColor3}{red!50!blue} % Color of a curve in a plot
\colorlet{penColor4}{green!50!black} % Color of a curve in a plot
\colorlet{penColor5}{orange!80!black} % Color of a curve in a plot
\colorlet{penColor6}{yellow!70!black} % Color of a curve in a plot
\colorlet{fill1}{penColor!20} % Color of fill in a plot
\colorlet{fill2}{penColor2!20} % Color of fill in a plot
\colorlet{fillp}{fill1} % Color of positive area
\colorlet{filln}{penColor2!20} % Color of negative area
\colorlet{fill3}{penColor3!20} % Fill
\colorlet{fill4}{penColor4!20} % Fill
\colorlet{fill5}{penColor5!20} % Fill
\colorlet{gridColor}{gray!50} % Color of grid in a plot

\newcommand{\surfaceColor}{violet}
\newcommand{\surfaceColorTwo}{redyellow}
\newcommand{\sliceColor}{greenyellow}




\pgfmathdeclarefunction{gauss}{2}{% gives gaussian
  \pgfmathparse{1/(#2*sqrt(2*pi))*exp(-((x-#1)^2)/(2*#2^2))}%
}


%%%%%%%%%%%%%
%% Vectors
%%%%%%%%%%%%%

%% Simple horiz vectors
\renewcommand{\vector}[1]{\left\langle #1\right\rangle}


%% %% Complex Horiz Vectors with angle brackets
%% \makeatletter
%% \renewcommand{\vector}[2][ , ]{\left\langle%
%%   \def\nextitem{\def\nextitem{#1}}%
%%   \@for \el:=#2\do{\nextitem\el}\right\rangle%
%% }
%% \makeatother

%% %% Vertical Vectors
%% \def\vector#1{\begin{bmatrix}\vecListA#1,,\end{bmatrix}}
%% \def\vecListA#1,{\if,#1,\else #1\cr \expandafter \vecListA \fi}

%%%%%%%%%%%%%
%% End of vectors
%%%%%%%%%%%%%

%\newcommand{\fullwidth}{}
%\newcommand{\normalwidth}{}



%% makes a snazzy t-chart for evaluating functions
%\newenvironment{tchart}{\rowcolors{2}{}{background!90!textColor}\array}{\endarray}

%%This is to help with formatting on future title pages.
\newenvironment{sectionOutcomes}{}{}



%% Flowchart stuff
%\tikzstyle{startstop} = [rectangle, rounded corners, minimum width=3cm, minimum height=1cm,text centered, draw=black]
%\tikzstyle{question} = [rectangle, minimum width=3cm, minimum height=1cm, text centered, draw=black]
%\tikzstyle{decision} = [trapezium, trapezium left angle=70, trapezium right angle=110, minimum width=3cm, minimum height=1cm, text centered, draw=black]
%\tikzstyle{question} = [rectangle, rounded corners, minimum width=3cm, minimum height=1cm,text centered, draw=black]
%\tikzstyle{process} = [rectangle, minimum width=3cm, minimum height=1cm, text centered, draw=black]
%\tikzstyle{decision} = [trapezium, trapezium left angle=70, trapezium right angle=110, minimum width=3cm, minimum height=1cm, text centered, draw=black]


\author{Jim Talamo}
\license{Creative Commons 3.0 By-bC}

\outcome{}

\begin{document}
\begin{exercise}
Suppose that $f(x) = \sum_{k=0}^{\infty} \frac{2^k}{2k-3}(x-2)^{3k+1}$.

\begin{exercise}
Find $f^{(16)}(2)$.

\[
f^{(16)}(2) = \answer{\frac{32}{7}} \cdot \left(\answer{16}\right)! 
\]
\end{exercise}

\begin{hint}
A good way to proceed is to use the relationship between the coefficients of the power series and the derivatives of the function it represents.

\[
\textrm{If } f(x) = \sum_{k=0}^{\infty} a_k(x-c)^k, \textrm{ then: } a_n = \frac{f^{(n)}(c)}{n!}
\]

Here, $c=\answer{2}$.  In order to find $f^{(16)}(2)$, we should use $n=\answer{16}$.

The coefficient in question is thus $a_{\answer{16}}$.  

\begin{question}
By definition $a_{16}$ is:

\begin{multipleChoice}
\choice{Always the coefficient obtained by plugging in $k=16$.}
\choice[correct]{The coefficient in front of $(x-c)^{16}$.}
\end{multipleChoice}

In this case, we find $(x-2)^{3k+1}=(x-2)^{16}$ when $k=\answer{5}$, so:

\[
a_{16} =  \frac{2^k}{2k-3} \bigg|_{k=\answer{5}} = \frac{\answer{32}}{\answer{7}}
\]

\begin{question}
We now use the formula $a_n = \frac{f^{(n)}(c)}{n!}$, with $n=16$ (as found earlier) to find:

\begin{align*}
a_{16} &= \frac{f^{(16)}(2)}{16!} \\
\answer{\frac{32}{7}} &= \frac{f^{(16)}(2)}{\left(\answer{16}\right)!}
\end{align*}

\begin{question}
Thus, $f^{(16)}(2) = \answer{\frac{32}{7}} \cdot \left(\answer{16}\right)! $

\end{question}
\end{question}
\end{question}

\end{hint}
%%%%%%%%%%%%%%
 
\begin{exercise}
Find $f^{(17)}(2)$.

\[
f^{(17)}(2) = \answer{0}
\]


\begin{hint}
A good way to proceed is to use the relationship between the coefficients of the power series and the derivatives of the function it represents.

\[
\textrm{If } f(x) = \sum_{k=0}^{\infty} a_k(x-c)^k, \textrm{ then: } a_n = \frac{f^{(n)}(c)}{n!}
\]

Here, $c=\answer{2}$.  In order to find $f^{(17)}(2)$, we should use $n=\answer{17}$.

The coefficient in question is thus $a_{\answer{17}}$.  

\begin{question}
By definition $a_{17}$ is:

\begin{multipleChoice}
\choice{Always the coefficient obtained by plugging in $k=17$.}
\choice[correct]{The coefficient in front of $(x-c)^{17}$.}
\end{multipleChoice}

In this case, we find $(x-2)^{3k+1}=(x-2)^{17}$ when $k=\answer{\frac{16}{3}}$.  

\begin{question}
Note that the index of summation only runs over integer values (i.e. $k=0,1,2,\ldots$). Since $k=\frac{16}{3}$ is not an integer:

\begin{multipleChoice}
\choice{$a_{17}$ is undefined.}
\choice[correct]{$a_{17}=0$.}
\end{multipleChoice}

\begin{question}
We now use the formula $a_n = \frac{f^{(n)}(c)}{n!}$, with $n=17$ (as found earlier) to find:

\begin{align*}
a_{17} &= \frac{f^{(17)}(2)}{17!} \\
\answer{0} &= \frac{f^{(17)}(2)}{\left(\answer{17}\right)!}
\end{align*}

Thus, $f^{(17)}(2) = \answer{0}$.
\end{question}
\end{question}
\end{question}
\end{hint}

\begin{exercise}
Of course, in this case, we could also write out the first several terms until we find the desired coefficients:

\begin{align*}
f(x) &=\sum_{k=0}^{\infty} \frac{2^k}{2k-3}(x-2)^{3k+1} \\
& = -\frac{1}{3} (x-2) +\answer{-2}(x-2)^4+\answer{4}(x-2)^7+\answer{\frac{8}{3}}(x-2)^{10}+ \\
& \qquad + \answer{\frac{16}{5}}(x-2)^{13}+\answer{\frac{32}{7}}(x-2)^{16}+\answer{\frac{64}{9}}(x-2)^{19}+\ldots \\
\end{align*}

Note that we obtain the coefficient $a_{16}$ when we substitute $k=5$ into the sum (as found earlier) and that the ``missing'' $(x-2)^{17}$ term should be thought of as $0 \cdot (x-2)^{17}$.  This is further justified via the observation:

\begin{image}
  \begin{tikzpicture}
        \node at (0,0) {
          $f(x)= \underbrace{-\frac{1}{3}(x-2)+\ldots+\frac{32}{7}(x-2)^{16}}+ \overbrace{\frac{64}{9}(x-2)^{19}+\ldots}$};
        \node at (-1.2,-.8) {\small{These terms will vanish}}; 
        \node at (3.1,1.3) {\small{These will have nonzero powers of $(x-2)$}};  
        \node at (3.1,1) {\small{and vanish when we evaluate at $x=2$}};  
             \end{tikzpicture}
  \end{image}

Thus, there will be nothing nonzero left after taking $16$ derivatives and evaluating the result at $x=2$, so $f^{(17)}(2)$ should be $0$.


\end{exercise}
\end{exercise}

%%%%%%%%%%%%%%%%%%%






\end{exercise}
\end{document}
