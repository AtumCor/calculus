\documentclass{ximera}

%\usepackage{todonotes}
%\usepackage{mathtools} %% Required for wide table Curl and Greens
%\usepackage{cuted} %% Required for wide table Curl and Greens
\newcommand{\todo}{}

\usepackage{esint} % for \oiint
\ifxake%%https://math.meta.stackexchange.com/questions/9973/how-do-you-render-a-closed-surface-double-integral
\renewcommand{\oiint}{{\large\bigcirc}\kern-1.56em\iint}
\fi


\graphicspath{
  {./}
  {ximeraTutorial/}
  {basicPhilosophy/}
  {functionsOfSeveralVariables/}
  {normalVectors/}
  {lagrangeMultipliers/}
  {vectorFields/}
  {greensTheorem/}
  {shapeOfThingsToCome/}
  {dotProducts/}
  {partialDerivativesAndTheGradientVector/}
  {../productAndQuotientRules/exercises/}
  {../motionAndPathsInSpace/exercises/}
  {../normalVectors/exercisesParametricPlots/}
  {../continuityOfFunctionsOfSeveralVariables/exercises/}
  {../partialDerivativesAndTheGradientVector/exercises/}
  {../directionalDerivativeAndChainRule/exercises/}
  {../commonCoordinates/exercisesCylindricalCoordinates/}
  {../commonCoordinates/exercisesSphericalCoordinates/}
  {../greensTheorem/exercisesCurlAndLineIntegrals/}
  {../greensTheorem/exercisesDivergenceAndLineIntegrals/}
  {../shapeOfThingsToCome/exercisesDivergenceTheorem/}
  {../greensTheorem/}
  {../shapeOfThingsToCome/}
  {../separableDifferentialEquations/exercises/}
  {vectorFields/}
}

\newcommand{\mooculus}{\textsf{\textbf{MOOC}\textnormal{\textsf{ULUS}}}}

\usepackage{tkz-euclide}\usepackage{tikz}
\usepackage{tikz-cd}
\usetikzlibrary{arrows}
\tikzset{>=stealth,commutative diagrams/.cd,
  arrow style=tikz,diagrams={>=stealth}} %% cool arrow head
\tikzset{shorten <>/.style={ shorten >=#1, shorten <=#1 } } %% allows shorter vectors

\usetikzlibrary{backgrounds} %% for boxes around graphs
\usetikzlibrary{shapes,positioning}  %% Clouds and stars
\usetikzlibrary{matrix} %% for matrix
\usepgfplotslibrary{polar} %% for polar plots
\usepgfplotslibrary{fillbetween} %% to shade area between curves in TikZ
%\usetkzobj{all} obsolete

\usepackage[makeroom]{cancel} %% for strike outs
%\usepackage{mathtools} %% for pretty underbrace % Breaks Ximera
%\usepackage{multicol}
\usepackage{pgffor} %% required for integral for loops



%% http://tex.stackexchange.com/questions/66490/drawing-a-tikz-arc-specifying-the-center
%% Draws beach ball
\tikzset{pics/carc/.style args={#1:#2:#3}{code={\draw[pic actions] (#1:#3) arc(#1:#2:#3);}}}



\usepackage{array}
\setlength{\extrarowheight}{+.1cm}
\newdimen\digitwidth
\settowidth\digitwidth{9}
\def\divrule#1#2{
\noalign{\moveright#1\digitwidth
\vbox{\hrule width#2\digitwidth}}}





\newcommand{\RR}{\mathbb R}
\newcommand{\R}{\mathbb R}
\newcommand{\N}{\mathbb N}
\newcommand{\Z}{\mathbb Z}

\newcommand{\sagemath}{\textsf{SageMath}}


%\renewcommand{\d}{\,d\!}
\renewcommand{\d}{\mathop{}\!d}
\newcommand{\dd}[2][]{\frac{\d #1}{\d #2}}
\newcommand{\pp}[2][]{\frac{\partial #1}{\partial #2}}
\renewcommand{\l}{\ell}
\newcommand{\ddx}{\frac{d}{\d x}}

\newcommand{\zeroOverZero}{\ensuremath{\boldsymbol{\tfrac{0}{0}}}}
\newcommand{\inftyOverInfty}{\ensuremath{\boldsymbol{\tfrac{\infty}{\infty}}}}
\newcommand{\zeroOverInfty}{\ensuremath{\boldsymbol{\tfrac{0}{\infty}}}}
\newcommand{\zeroTimesInfty}{\ensuremath{\small\boldsymbol{0\cdot \infty}}}
\newcommand{\inftyMinusInfty}{\ensuremath{\small\boldsymbol{\infty - \infty}}}
\newcommand{\oneToInfty}{\ensuremath{\boldsymbol{1^\infty}}}
\newcommand{\zeroToZero}{\ensuremath{\boldsymbol{0^0}}}
\newcommand{\inftyToZero}{\ensuremath{\boldsymbol{\infty^0}}}



\newcommand{\numOverZero}{\ensuremath{\boldsymbol{\tfrac{\#}{0}}}}
\newcommand{\dfn}{\textbf}
%\newcommand{\unit}{\,\mathrm}
\newcommand{\unit}{\mathop{}\!\mathrm}
\newcommand{\eval}[1]{\bigg[ #1 \bigg]}
\newcommand{\seq}[1]{\left( #1 \right)}
\renewcommand{\epsilon}{\varepsilon}
\renewcommand{\phi}{\varphi}


\renewcommand{\iff}{\Leftrightarrow}

\DeclareMathOperator{\arccot}{arccot}
\DeclareMathOperator{\arcsec}{arcsec}
\DeclareMathOperator{\arccsc}{arccsc}
\DeclareMathOperator{\si}{Si}
\DeclareMathOperator{\scal}{scal}
\DeclareMathOperator{\sign}{sign}


%% \newcommand{\tightoverset}[2]{% for arrow vec
%%   \mathop{#2}\limits^{\vbox to -.5ex{\kern-0.75ex\hbox{$#1$}\vss}}}
\newcommand{\arrowvec}[1]{{\overset{\rightharpoonup}{#1}}}
%\renewcommand{\vec}[1]{\arrowvec{\mathbf{#1}}}
\renewcommand{\vec}[1]{{\overset{\boldsymbol{\rightharpoonup}}{\mathbf{#1}}}\hspace{0in}}

\newcommand{\point}[1]{\left(#1\right)} %this allows \vector{ to be changed to \vector{ with a quick find and replace
\newcommand{\pt}[1]{\mathbf{#1}} %this allows \vec{ to be changed to \vec{ with a quick find and replace
\newcommand{\Lim}[2]{\lim_{\point{#1} \to \point{#2}}} %Bart, I changed this to point since I want to use it.  It runs through both of the exercise and exerciseE files in limits section, which is why it was in each document to start with.

\DeclareMathOperator{\proj}{\mathbf{proj}}
\newcommand{\veci}{{\boldsymbol{\hat{\imath}}}}
\newcommand{\vecj}{{\boldsymbol{\hat{\jmath}}}}
\newcommand{\veck}{{\boldsymbol{\hat{k}}}}
\newcommand{\vecl}{\vec{\boldsymbol{\l}}}
\newcommand{\uvec}[1]{\mathbf{\hat{#1}}}
\newcommand{\utan}{\mathbf{\hat{t}}}
\newcommand{\unormal}{\mathbf{\hat{n}}}
\newcommand{\ubinormal}{\mathbf{\hat{b}}}

\newcommand{\dotp}{\bullet}
\newcommand{\cross}{\boldsymbol\times}
\newcommand{\grad}{\boldsymbol\nabla}
\newcommand{\divergence}{\grad\dotp}
\newcommand{\curl}{\grad\cross}
%\DeclareMathOperator{\divergence}{divergence}
%\DeclareMathOperator{\curl}[1]{\grad\cross #1}
\newcommand{\lto}{\mathop{\longrightarrow\,}\limits}

\renewcommand{\bar}{\overline}

\colorlet{textColor}{black}
\colorlet{background}{white}
\colorlet{penColor}{blue!50!black} % Color of a curve in a plot
\colorlet{penColor2}{red!50!black}% Color of a curve in a plot
\colorlet{penColor3}{red!50!blue} % Color of a curve in a plot
\colorlet{penColor4}{green!50!black} % Color of a curve in a plot
\colorlet{penColor5}{orange!80!black} % Color of a curve in a plot
\colorlet{penColor6}{yellow!70!black} % Color of a curve in a plot
\colorlet{fill1}{penColor!20} % Color of fill in a plot
\colorlet{fill2}{penColor2!20} % Color of fill in a plot
\colorlet{fillp}{fill1} % Color of positive area
\colorlet{filln}{penColor2!20} % Color of negative area
\colorlet{fill3}{penColor3!20} % Fill
\colorlet{fill4}{penColor4!20} % Fill
\colorlet{fill5}{penColor5!20} % Fill
\colorlet{gridColor}{gray!50} % Color of grid in a plot

\newcommand{\surfaceColor}{violet}
\newcommand{\surfaceColorTwo}{redyellow}
\newcommand{\sliceColor}{greenyellow}




\pgfmathdeclarefunction{gauss}{2}{% gives gaussian
  \pgfmathparse{1/(#2*sqrt(2*pi))*exp(-((x-#1)^2)/(2*#2^2))}%
}


%%%%%%%%%%%%%
%% Vectors
%%%%%%%%%%%%%

%% Simple horiz vectors
\renewcommand{\vector}[1]{\left\langle #1\right\rangle}


%% %% Complex Horiz Vectors with angle brackets
%% \makeatletter
%% \renewcommand{\vector}[2][ , ]{\left\langle%
%%   \def\nextitem{\def\nextitem{#1}}%
%%   \@for \el:=#2\do{\nextitem\el}\right\rangle%
%% }
%% \makeatother

%% %% Vertical Vectors
%% \def\vector#1{\begin{bmatrix}\vecListA#1,,\end{bmatrix}}
%% \def\vecListA#1,{\if,#1,\else #1\cr \expandafter \vecListA \fi}

%%%%%%%%%%%%%
%% End of vectors
%%%%%%%%%%%%%

%\newcommand{\fullwidth}{}
%\newcommand{\normalwidth}{}



%% makes a snazzy t-chart for evaluating functions
%\newenvironment{tchart}{\rowcolors{2}{}{background!90!textColor}\array}{\endarray}

%%This is to help with formatting on future title pages.
\newenvironment{sectionOutcomes}{}{}



%% Flowchart stuff
%\tikzstyle{startstop} = [rectangle, rounded corners, minimum width=3cm, minimum height=1cm,text centered, draw=black]
%\tikzstyle{question} = [rectangle, minimum width=3cm, minimum height=1cm, text centered, draw=black]
%\tikzstyle{decision} = [trapezium, trapezium left angle=70, trapezium right angle=110, minimum width=3cm, minimum height=1cm, text centered, draw=black]
%\tikzstyle{question} = [rectangle, rounded corners, minimum width=3cm, minimum height=1cm,text centered, draw=black]
%\tikzstyle{process} = [rectangle, minimum width=3cm, minimum height=1cm, text centered, draw=black]
%\tikzstyle{decision} = [trapezium, trapezium left angle=70, trapezium right angle=110, minimum width=3cm, minimum height=1cm, text centered, draw=black]



\outcome{Identify separable differential equation.}
\outcome{Compute general solution of a separable differential equation.}
\outcome{Find implicit solutions of a separable differential equation.}

\title[Dig-In:]{Separable differential equations}

\begin{document}
\begin{abstract}
  Separable differential equations are those in which the dependent and independent variables can be separated on opposite sides of the equation.
\end{abstract}
\maketitle

In this section we will see that the phrase
\begin{quote}
  Divide and Conquer
\end{quote}
is literally true (with the mathematical definition of ``divide'') in
the context of differential equations. Rather than talk about math,
let's just show you what we're getting at.

\begin{example}
  Solve the differential equation
  \[
  y'(x) = x\cdot y(x)^2
  \]
  where $y(0)= 5$.
  \begin{explanation}
    We're going to divide and conquer, literally. Divide both sides of
    \[
    y'(x) = x\cdot y(x)^2
    \]
    by $y(x)^2$ to find
    \[
    \frac{y'(x)}{y(x)^2} = \answer[given]{x}.
    \]
    Now integrate both sides with respect to $x$
    \begin{align*}
      \int \frac{y'(x)}{y(x)^2} \d x &= \int x \d x,\\
      \frac{-1}{y(x)} &= \answer[given]{\frac{x^2}{2}}+C
    \end{align*}
    and solve for $y(x)$ to find
    \[
    y(x) = \answer[given]{\frac{-1}{\frac{x^2}{2}+C}}.
    \]
    Since $y(0) =5$ we write
    \begin{align*}
    5&= y(0)\\
    &=\frac{-1}{C}
    \end{align*}
    So $C= -1/5$. Our final answer is
    \[
    y(x) = \frac{-1}{\frac{x^2}{2} \answer[given]{-\frac{1}{5}}}.
    \]
  \end{explanation}
\end{example}

Now that we've seen an example, let's discuss this in general.

\begin{definition}
  A \dfn{separable differential equation} is a differential equation
  which can be written in the form
  \[
  G(y) y' = H(x)
  \]
  In other words, the independent variable $x$ and the function $y$
  can be placed on \textbf{separate} sides of the equals sign.
\end{definition}

\begin{warning}
  Calculus has lots of different notation. Each of these expressions
  means exactly the same thing:
  \begin{itemize}
  \item $y' = r\cdot y(1-y)$
  \item $y'(x) = r\cdot y(x)(1-y(x))$
  \item $\dd[y]{x} = r \cdot y(1-y)$
  \item $\dd{x} y(x) = r\cdot y(x) (1-y(x))$
  \end{itemize}
  We will work with all the notation above.
\end{warning}

\begin{question}
  Which of the following are separable differential equations?  Select all that apply.
  \begin{selectAll}
    \choice[correct]{$y' = y/x$}
    \choice{$y' = \sqrt{x^2+y^2}$}
    \choice[correct]{$\cos(y(x))y'(x) = \frac{x}{1+y(x)}$}
    \choice[correct]{$\dd[y]{x} = y+5$}
    \choice{$y'(x) = \sin(x+y(x))$}
    \choice{$\dd[y]{x} + x^2 y= x$}
    \choice{$\dd{x} y(x) = \sin(x) y(x) - \cos(x)$}
    \choice[correct]{$\dd{x} y(x) = -5 y(x)^{1/3}$}
  \end{selectAll}
  \begin{hint}
    To be a separable differential equation, we must collect $y$ and
    $y'$ on one side, and $x$ on the other. Typically, you will do
    this with multiplication or division.
  \end{hint}
\end{question}

The reason we care about separable differential equations is that:
\begin{itemize}
\item Separable differential equations help model many real-world contexts.
\item Separable differential equations are solvable by humans.
\end{itemize}
The basic ideas is if
\[
G(y)y' = H(x)
\]
then we can integrate both sides, writing
\begin{align*}
  G(y)y'  &= H(x)\\
  \int G(y) y' \d x &= \int H(x) \d x\\
  \int G(y) \d y &= \int H(x) \d x
\end{align*}
If we can symbolically compute these integrals, then we can solve for
$y$. It is now time to work some examples. 


\subsection{Proportional reasoning}

In this example, we will show the power of knowing that one quantity
is proportional to another.

\begin{example}
  A snowball has a radius of $2$ inches. After $1.5$ hours, it has a
  radius of $1$ inch. Assume
  \begin{itemize}
  \item the radius changes at rate proportional to the snowball's
    surface area, and
  \item the surface area is proportional to the square of the
    snowball's radius.
  \end{itemize}
  When is the radius of the snowball $0.1$ inches?
  \begin{explanation}
    We'll do this by setting up a differential equation. Let $r$ be
    the radius of the snowball. From the statement of the problem we
    have
    \[
    \dd[r]{t} = k_1 \cdot \text{(surface area)}
    \]
    and
    \[
    \text{(surface area)} = k_2 \cdot r^2,
    \]
    hence
    \[
    \dd[r]{t} = k_1 \cdot k_2 \cdot r^2.
    \]
    Letting $K = k_1\cdot k_2$ we now can write
    \[
    \dd[r]{t} = \answer[given]{K}\cdot r^2.
    \]
    We now have a differential equation. Let's divide and
    conquer. Write with me
    \[
    \answer[given]{\frac{1}{r^2}}\dd[r]{t} = K
    \]
    integrating both sides with respect to $t$ we find
    \begin{align*}
      \int \answer[given]{\frac{1}{r^2}}\dd[r]{t}\d t &= \int K \d t\\
      \int \answer[given]{\frac{1}{r^2}}\d r &= \int K \d t\\
      \answer[given]{\frac{-1}{r}} &= K\cdot t+C\\
      \frac{1}{r} &= -K\cdot t- C\\
      r &=\answer[given]{\frac{1}{-K\cdot t- C}}.
    \end{align*}
    We know that $r(0) = \answer[given]{2}$, so let's put that information to use
    \begin{align*}
      \answer[given]{2} &= r(0)\\
      \answer[given]{2} &= \frac{1}{-K\cdot 0- C}\\
      \answer[given]{2} &= \frac{1}{- C}\\
      C &= \answer[given]{-1/2}.
    \end{align*}
    We also know that $r(1.5) = \answer[given]{1}$. Again write with me, noting that
    $1.5 = 3/2$
    \begin{align*}
    \answer[given]{1} &= r(3/2)\\
    \answer[given]{1} &= \frac{1}{-K\cdot 3/2 +1/2}\\
    -K\cdot 3/2 +1/2 &=1\\
    -K\cdot 3 +1 &=2\\
    -K\cdot 3 &=1\\
    K &=\answer[given]{-1/3}.
    \end{align*}
    Hence
    \[
    r(t) = \answer[given]{\frac{1}{t/3+1/2}}.
    \]
    To find when the radius is $0.1$ inches, solve
    \[
    0.1 = \frac{1}{t/3+1/2}
    \]
    and we find $t$ is $\answer[given]{28.5}$ hours.
  \end{explanation}
\end{example}
Some things to note about our last example:
\begin{itemize}
\item It was solved \textit{without} using the formula for the surface area
of a sphere. For us, it was sufficient to know that the surface area
of a sphere is proportional to the square of the radius.
\item Our solution, $r(t) = \frac{1}{t/3+1/2}$ can be plotted in the
  slope field determined by $r'=(-1/3)r^2$:
  \begin{image}
    {\def\length{sqrt(1+((-1/3)*y^2)^2)}
      \begin{tikzpicture}
        \begin{axis}[
            xmin=0, xmax=6,ymin=0,ymax=6,domain=0:6,view={0}{90},
            axis lines =center, xlabel=$t$, ylabel=$r$,
            every axis y label/.style={at=(current axis.above origin),anchor=south},
            every axis x label/.style={at=(current axis.right of origin),anchor=west},
            axis on top,
          ] 
          \addplot3 [penColor, quiver={u={1/\length}, v={((-1/3)*y^2)/(\length)},scale arrows=.2},samples=20] {0};
          \addplot[penColor,very thick]{1/((x/3)+1/2)};
        ]  \end{axis}
        \end{tikzpicture}}
  \end{image}
\end{itemize}
\begin{question}
  Which of the following are equilibrium solutions to the differential equation $r'=(-1/3)r^2$?
  \begin{multipleChoice}
    \choice[correct]{$r = 0$}
    \choice{$r = 1$}
    \choice{$r = t$}
    \choice{$t = 0$}
    \choice{$t = 1$}
  \end{multipleChoice}
\end{question}





\subsection{Exponential and logistic growth}

In the science fiction television series \index{Star Trek}\textit{Star
  Trek}, a \dfn{tribble} is an alien species that is furry, spherical
(radius $\approx 5$ inches), that essentially does nothing but eat and
reproduce. \youtube{p4wM5KvUGEc}

\begin{example}
  Mr.\ Spock claims that there are $1771561$ tribbles aboard the space
  station K7, ``assuming one tribble, multiplying with an average
  litter of ten, producing a new generation every twelve hours over a
  period of three days.'' Explain Mr.\ Spock's computation.
  \begin{explanation}
    Perhaps Mr.\ Spock used an exponential differential equation to
    model this.  Letting $P$ be the population of tribbles at time $t$
    (in hours), it makes sense that the rate that the population is
    growing is proportional to the size of the population; this is the
    definition of an exponential model. Write with me
    \[
    P' = k P.
    \]
    Ah, this is a separable differential equation. Let's solve it:
    \begin{align*}
      P' &= k P\\
      \frac{P'}{P} &= k\\
      \int \frac{P'}{P} \d t &= \int k\d t\\
      \answer[given]{\ln(P)} &= k\cdot t + C.
    \end{align*}
    However we also know that $P(0) = \answer[given]{1}$. Write with me
    \begin{align*}
      \ln(P(0)) &= k\cdot 0 + C\\
      \ln(\answer[given]{1}) &= C\\
      \answer[given]{0} &=C.
    \end{align*}
    After $12$ hours, we should have $\answer[given]{11}$ tribbles. So 
    \[
    P(12) = \answer[given]{11},
    \]
    and we may write
    \begin{align*}
      \ln(P(12)) &= k\cdot \answer[given]{12}\\
    \ln(\answer[given]{11}) &= k\cdot \answer[given]{12}\\
    \answer[given]{\frac{\ln(11)}{12}} &= k.
    \end{align*}
    Hence we can model the population growth with
    \begin{align*}
      \ln(P(t)) &= \frac{\ln(11)}{12}\cdot t\\
      P(t) &= e^{\frac{\ln(11)}{12} \cdot t}\\
      P(t) &= \answer[given]{11}^{t/12}.
    \end{align*}
    Checking this with Mr.\ Spock's computation, we see that
    \begin{align*}
      P(72) &= 11^{72/12}\\
      &= 11^6\\
      &=1771561.
    \end{align*}
    just as Mr.\ Spock stated.
  \end{explanation}
\end{example}

If we graph the solution to the differential equation representing the
population of tribbles that we found above
\begin{image}
  \begin{tikzpicture}
      \begin{axis}[
          xmin=0, xmax=100,domain=0:100,clip=false,width=4in,height=2in,
          axis lines =center, xlabel=$t$, ylabel=$P$,
          every axis y label/.style={at=(current axis.above origin),anchor=south},
          every axis x label/.style={at=(current axis.right of origin),anchor=west},
          axis on top,
        ] 
        \addplot[penColor,very thick,smooth]{e^(.2*x)};
      ]  \end{axis}
      \end{tikzpicture}
\end{image}
we see that some time after $72$ hours, the growth rate of the
tribbles explodes. In reality this would be an \textbf{ecological
  disaster}. Exponential growth can be scary.

Now we will see a model of population growth with environmental
limitations. Suppose that the birthrate of the tribbles is limited by
how much food and space is available. This gets us to the idea of \textit{carrying capacity}.

\begin{definition}
  The \dfn{carrying capacity} of a biological system is the maximum
  population that can be sustained indefinitely with the given
  resources.
\end{definition}

Let's work an example involving this concept:

\begin{example}
Suppose that the carrying capacity of tribbles aboard a space station
is $1000$ tribbles, we start with $1$ tribble, and that growth rate
(per tribble) is $10$ tribbles every twelve hours. Model this
population with a differential equation.
\begin{explanation}
  First, note that the population of tribbles is proportional to the
  \textbf{product} $P$ and $1000-P$. This makes sense as when $P$ is
  small, the growth rate should be small; and when $1000-P$ is small,
  the growth rate should also be small. Write with me
  \[
  \dd[P]{t}  = k \cdot P (1000-P)
  \]
  Ah! This is a separable differential equation. Let's solve it:
  \begin{align*}
    \dd[P]{t}  &= k \cdot P (1000-P)\\
    \frac{1}{P (1000-P)}\dd[P]{t}  &= k \\
    \int \frac{1}{P (1000-P)}\dd[P]{t}\d t  &= \int k\d t \\
    \int \answer[given]{\frac{1}{P (1000-P)}} \d P  &= \int k\d t.
  \end{align*}
  Use partial fraction decomposition to compute the antiderivative:
  \[
    \frac{1}{1000}\ln\left(\answer[given]{\frac{P}{1000-P}}\right)   = k\cdot t + C.
  \]
  However we also know that $P(0) = \answer[given]{1}$. Write with me
  \begin{align*}
      \frac{1}{1000}\ln\left(\frac{P(0)}{1000-P(0)}\right)  &= k\cdot 0 + C\\
      \frac{1}{1000}\ln\left(\frac{1}{999}\right)  &= C\\
      \answer[given]{\frac{-\ln(999)}{1000}}  &= C.
  \end{align*}
  After $12$ hours, we should have $\answer[given]{11}$ tribbles. So 
  \[
  P(12) = \answer[given]{11},
  \]
  and we may write
  \begin{align*}
    \frac{1}{1000}\ln\left(\frac{P(12)}{1000-P(12)}\right) &= k\cdot 12-\frac{\ln(999)}{1000}\\
    \ln\left(\frac{11}{989}\right) &= k\cdot 12000-\ln(999) \\
    \frac{\ln(10989)-\ln(989)}{12000} &= k.
  \end{align*}
  Hence we can model the population growth with
  \[
  \frac{1}{1000}\ln\left(\frac{P(t)}{1000-P(t)}\right) = \frac{\ln(10989)-\ln(989)}{12000}\cdot t - \frac{\ln(999)}{1000}
  \]
  \begin{align*}
    \ln\left(\frac{P(t)}{1000-P(t)}\right)  &= \frac{\ln(10989)-\ln(989)}{12}\cdot t - \ln(999)\\
    \frac{P(t)}{1000-P(t)}  &= \left(\frac{10989}{989}\right)^{t/12}\cdot \frac{1}{999}
  \end{align*}
  Solving for $P(t)$ we find
  \[
  P(t) = \frac{\left(\frac{10989}{989}\right)^{t/12}\cdot \frac{1000}{999}}{1+\left(\frac{10989}{989}\right)^{t/12}\cdot \frac{1}{999}}.
  \]
  If we graph this solution, we see a nice ``S-curve''
  \begin{image}
  \begin{tikzpicture}
      \begin{axis}[
          xmin=0, xmax=60,domain=0:60,clip=false,width=4in,height=2in,
          axis lines =center, xlabel=$t$, ylabel=$P$,
          every axis y label/.style={at=(current axis.above origin),anchor=south},
          every axis x label/.style={at=(current axis.right of origin),anchor=west},
          axis on top,
        ] 
        \addplot[penColor,very thick,smooth]{1000*(1-999/(999+(407/989)^(x/12)*3^(x/4)))};
      ]  \end{axis}
      \end{tikzpicture}
\end{image}
  \end{explanation}
\end{example}

In our last example we used the differential equation
\[
P' = k\cdot P(A-P)
\]
This is called the \dfn{logistic differential equation} where $k$ and
$A$ are constants. Let's examine a slope field for this model with
some reasonable values of $k$ and $A$:
 \begin{image}
    {\def\length{sqrt(1+((.6*y*(2-y))^2)}
      \begin{tikzpicture}
        \begin{axis}[
            xmin=-3, xmax=3,ymin=-3,ymax=3,domain=-3:3,view={0}{90},
            axis lines =center, xlabel=$t$, ylabel=$r$,
            every axis y label/.style={at=(current axis.above origin),anchor=south},
            every axis x label/.style={at=(current axis.right of origin),anchor=west},
            axis on top,
          ] 
          \addplot3 [penColor, quiver={u={1/\length}, v={(.6*y*(2-y))/(\length)},scale arrows=.2},samples=20] {0};
        ]  \end{axis}
        \end{tikzpicture}}
  \end{image}
\begin{question}
  Which of the following are equilibrium solutions to the differential equation above?
  \begin{selectAll}
    \choice[correct]{$P = 0$}
    \choice{$P = 1$}
    \choice[correct]{$P = 2$}
    \choice{$P = t$}
  \end{selectAll}
\end{question}



\subsection{Predator-prey model}

In nature there are animals that are predators and animals that are
prey. In the early 20th Century, Lotka and Volterra suggested the
following model to help us understand populations of predators and
prey. Let
\begin{align*}
  x(t) &= \text{population of predators at time $t$}\\
  y(t) &= \text{population of prey at time $t$}.
\end{align*}
Lotka and Volterra made the following assumptions:
\begin{itemize}
  \item If there are no prey, then the predators starve at some rate
    proportional to the number of predators. If there are prey, then
    the predator's population grows at a rate proportional to the product of the
    population of the predators and the prey. In the language of calculus we write
    \[
    \dd[x]{t} = -p x + q x y.
    \]
  \item If there are no predators, then the prey's population grows
    at some rate proportional to the number of prey. If there are
    predators, then the prey's population is reduced at a rate
    proportional to the product of the population of the predators and
    the prey. In the language of calculus we write
    \[
    \dd[y]{t} = r y - s xy.
    \]
\end{itemize}
Taking these two equations, and dividing them we find
\begin{align*}
\dd[y]{x} &= \frac{\dd[y]{t}}{\dd[x]{t}}\\
&=\frac{r y - s xy}{-p x + q x y}\\
&=\frac{y(r-sx)}{x(qy-p)}.
\end{align*}
In the previous examples, it worked out that we could solve for $y$ as
a function of $x$.  As we will see with the predator-prey model, this
is not always the case.

\begin{example}
  Find the solution to the differential equation
  \[
  y' = \frac{y(4-2x)}{x(y-3)}
  \]
  that passes through the point $(1,e)$.
  \begin{explanation}
    To start, note that this is a separable differential equation. Write with me:
    \begin{align*}
      y' &= \frac{y(4-2x)}{x(y-3)}\\
      \frac{(y-3)y'}{3}&=\frac{(4-2x)}{x}\\
      \int \frac{(y-3)y'}{y}\d x&=\int \frac{(4-2x)}{x}\d x\\
      \int y'-\frac{3y'}{y}\d x&=\int \frac{4}{x}-2\d x\\
      \answer[given]{y-3\ln(y)} &=4\ln(x)-2x+C.
    \end{align*}
    Remembering that the solution must pass through point $(1,e)$, we write
    \begin{align*}
      e-3\ln(e) &=4\ln(1)-2+C\\
      \answer[given]{e-3} &=-2+C\\
      \answer[given]{e-1} &=C.
    \end{align*}
    Hence we have the following implicit equation relating $x$ and $y$:
    \[
    y-3\ln(y) =4\ln(x)-2x+e-1.
    \]
    Let's look at this solution along with the slope field:
    \begin{image}
      {\def\length{sqrt((-3*x+1*x*y)^2+(4*y-2*x*y)^2)}
        \begin{tikzpicture}
          \begin{axis}[
              xmin=0, xmax=6,ymin=0,ymax=6,domain=0:6,view={0}{90},
              axis lines =center, xlabel=$x$, ylabel=$y$,
              every axis y label/.style={at=(current axis.above origin),anchor=south},
              every axis x label/.style={at=(current axis.right of origin),anchor=west},
              axis on top,
            ] 
            \addplot3 [penColor, quiver={u={(-3*x+1*x*y)/\length}, v={(4*y-2*x*y)/(\length)},scale arrows=.2},samples=20] {0};
            \addplot [penColor,ultra thick] coordinates {
              (1.01786, 2.4898) (1.02071, 2.46429) (1.02639, 2.41924) (1.02904,2.39953) (1.03503, 2.35714) (1.03997, 2.32569) (1.04882, 2.27261) (1.05375, 2.25) (1.05574, 2.23431) (1.07143, 2.15663) (1.07392, 2.14535) (1.0745, 2.14286) (1.07554, 2.13874) (1.09611, 2.0604) (1.10466, 2.03571) (1.11647, 1.99067) (1.1205, 1.97764) (1.13999, 1.92857) (1.14926, 1.89926) (1.17857, 1.82845) (1.18074, 1.8236) (1.18195, 1.82143) (1.18508, 1.81492) (1.21723, 1.75294) (1.24181, 1.71429) (1.25769, 1.68626) (1.28571, 1.64637) (1.30252, 1.62394) (1.31592, 1.60714) (1.33929, 1.57989) (1.35223, 1.56651) (1.39286, 1.52748) (1.40722, 1.51436) (1.42626, 1.5) (1.46773, 1.46773) (1.5, 1.44741) (1.53383, 1.42668) (1.60186, 1.39286) (1.60538, 1.3911) (1.60714, 1.39041) (1.61121, 1.38879) (1.68435, 1.36292) (1.71429, 1.35506) (1.76648, 1.34066) (1.76872, 1.34015) (1.82143, 1.33233) (1.86132, 1.32561) (1.89178, 1.32251) (1.92857, 1.32077) (1.96311, 1.32026) (2.03571, 1.31917) (2.10467, 1.3239) (2.14286, 1.32652) (2.19051, 1.33337) (2.25, 1.34192) (2.29202, 1.35084) (2.35714, 1.36465) (2.45998, 1.39286) (2.46329, 1.39385) (2.46429, 1.39422) (2.46633, 1.3949) (2.5419, 1.42238) (2.57143, 1.4352) (2.61797, 1.45346) (2.625, 1.45659) (2.64432, 1.46575) (2.67857, 1.48263) (2.71342, 1.5) (2.76015, 1.52556) (2.78571, 1.54157) (2.8271, 1.56575) (2.8891, 1.60714) (2.89127, 1.60873) (2.89286, 1.6099) (2.90634, 1.62063) (3., 1.69516) (3.01031, 1.70398) (3.02237, 1.71429) (3.10714, 1.79814) (3.11849, 1.81008) (3.12927, 1.82143) (3.21429, 1.92541) (3.21566, 1.92719) (3.21667, 1.92857) (3.22153, 1.93582) (3.29984, 2.05731) (3.34467, 2.14286) (3.37207, 2.19936) (3.42857, 2.34242) (3.43284, 2.35288) (3.43424, 2.35714) (3.43669, 2.36526) (3.47812, 2.52188) (3.48927, 2.57143) (3.50153, 2.64439) (3.51022, 2.70407) (3.51784, 2.78571) (3.52614, 2.88328) (3.52696, 2.90161) (3.52908, 3.) (3.52706, 3.09849) (3.52632, 3.11653) (3.51868, 3.21429) (3.51243, 3.29814) (3.50636, 3.35079) (3.49621, 3.42857) (3.48675, 3.48675) (3.46519, 3.60624) (3.45813, 3.64286) (3.45278, 3.66706) (3.42857, 3.7688) (3.41137, 3.83994) (3.40638, 3.85714) (3.39824, 3.88747) (3.36328, 4.00614) (3.34089, 4.07143) (3.31031, 4.16745) (3.29738, 4.20262) (3.26493, 4.28571) (3.25143, 4.32286) (3.21429, 4.4128) (3.1882, 4.47392) (3.17548, 4.5) (3.144, 4.57029) (3.12043, 4.62043) (3.10714, 4.64782) (3.07282, 4.71429) (3.04736, 4.76165) (3., 4.8473) (2.97011, 4.89868) (2.95023, 4.92857) (2.8883, 5.03116) (2.86792, 5.06065) (2.80978, 5.14286) (2.80035, 5.1575) (2.78571, 5.17749) (2.70636, 5.27779) (2.63354, 5.35714) (2.60553, 5.39124) (2.57143, 5.42683) (2.49626, 5.49626) (2.39792, 5.57143) (2.37613, 5.59041) (2.35714, 5.60392) (2.28181, 5.64676) (2.25, 5.66471) (2.24034, 5.66891) (2.21376, 5.67857) (2.16293, 5.69864) (2.14286, 5.70512) (2.10814, 5.71328) (2.07643, 5.71928) (2.03571, 5.72424) (1.98894, 5.72534) (1.97476, 5.72476) (1.92857, 5.72008) (1.89432, 5.71282) (1.83898, 5.69612) (1.82143, 5.68998) (1.8137, 5.6863) (1.80078, 5.67857) (1.74399, 5.64887) (1.71429, 5.62976) (1.68167, 5.60404) (1.6478, 5.57143) (1.60714, 5.53495) (1.57441, 5.49702) (1.55114, 5.46429) (1.52756, 5.43673) (1.50875, 5.41071) (1.5, 5.39845) (1.48421, 5.37294) (1.47443, 5.35714) (1.44643, 5.31131) (1.44367, 5.30633) (1.44229, 5.30357) (1.43767, 5.29481) (1.41545, 5.25) (1.40678, 5.23607) (1.39286, 5.20902) (1.37211, 5.1636) (1.36456, 5.14286) (1.33902, 5.08955) (1.33869, 5.08869) (1.32005, 5.03571) (1.30913, 5.01229) (1.28571, 4.95041) (1.28019, 4.9341) (1.27876, 4.92857) (1.27472, 4.91757) (1.25371, 4.85343) (1.24575, 4.82143) (1.2281, 4.7719) (1.22495, 4.76066) (1.21393, 4.71429) (1.20486, 4.688) (1.18475, 4.61332) (1.18335, 4.60714) (1.18216, 4.60356) (1.17857, 4.58965) (1.15894, 4.5) (1.15184, 4.47327) (1.14211, 4.42932) (1.13658, 4.39286) (1.12333, 4.34095) (1.12257, 4.33685) (1.11309, 4.28571) (1.10674, 4.2504) (1.0989, 4.20605) (1.09534, 4.17857) (1.09083, 4.15917) (1.07728, 4.07728) (1.0766, 4.07143) (1.07566, 4.06719) (1.07143, 4.04026) (1.06827, 4.01786) (1.06224, 3.97347) (1.061, 3.96429) (1.05944, 3.95229) (1.05003, 3.87854) (1.04737, 3.85714) (1.04413, 3.82985) (1.03868, 3.78275) (1.035, 3.75) (1.03067, 3.70924) (1.02827, 3.68601) (1.02397, 3.64286) (1.0191, 3.59053) (1.01889, 3.58825) (1.01786, 3.57624) (1.01472, 3.53571) (1.01136, 3.48864) (1.01042, 3.47471) (1.00975, 3.42857) (1.00507, 3.38778) (1.0036, 3.36075) (1.0016, 3.32143) (1.00003, 3.28569) (0.99854, 3.24854) (0.99729, 3.21429) (0.996364, 3.18221) (0.99527, 3.13813) (0.99462, 3.10714) (0.994237, 3.07719) (0.993831, 3.02955) (0.995279, 3.) (0.993834, 2.97045) (0.99426, 2.92283) (0.994665, 2.89286) (0.995379, 2.86176) (0.996591, 2.81802) (0.997649, 2.78571) (0.999145, 2.75086) (1.00086, 2.71514) (1.00281, 2.67857) (1.00547, 2.63739) (1.00709, 2.61423) (1.0128, 2.57143) (1.01477, 2.52094) (1.0153, 2.5153) (1.01786, 2.4898)
              };
          ]  \end{axis}
          \end{tikzpicture}}
    \end{image}
    From this graph, we can see the cyclic nature of the predator-prey
    relationship, where the population of the predators is on the
    $x$-axis, and the population of the prey is on the $y$-axis.
  \end{explanation}
\end{example}


%% \subsection{A substitution}

%% Sometimes equations might not look separable. While it is a bit beyond
%% the scope of this course, a subsitution can sometimes help.

%% \begin{example}
%%   Solve the differential equation
%%   \[
%%   y' = x+ y
%%   \]
%% \end{example}
\end{document}
